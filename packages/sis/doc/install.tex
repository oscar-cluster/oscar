% -*- latex -*-
%
% $Id: install.tex,v 1.5 2003/08/18 14:38:41 naughtont Exp $
%
% $COPYRIGHT$
%

\subsection{Managing machines and images in SIS}
\label{app:troubleshooting-machines-images}

During the life of your cluster, you may want to delete unused
images, create new images, or change the image that a
client uses.  Currently OSCAR doesn't have a direct interface to do
this, but you can use the SIS commands directly. Here are some useful
examples:

\begin{itemize}
\item To list all defined machines, run:
\begin{verbatim}
        mksimachine --List
\end{verbatim}
\item To list all defined images, run:
\begin{verbatim}
        mksiimage --List
\end{verbatim}
\item To delete an image, run:
\begin{verbatim}
        mksiimage --Delete --name <imagename>
\end{verbatim}
\item To change which image a machine will install, run:
\begin{verbatim}
        mksimachine --Update --name <machinename> --image <imagename>
\end{verbatim}
\end{itemize}

There is also a SIS GUI that is availble. Start it by running
\file{tksis}. 

More details on these commands can be obtained from their respective
man pages.

\subsubsection{Multicast Installs with SystemImager}
SIS now includes multicast install capability with SystemImager v3.2.x
and Flamethrower v1.0.x.  Working with multicast can prove very
beneficial, especially for large sites.  However, multicast can be a
tricky thing to get working reliably, based on networking equipment,
multicast tuning parameters, machine speed, etc.  If you are interested
in giving multicast a try, see the "HOWTO Use Flamethrower for Multicast
Installs" section in the SystemImager manual (http://www.systemimager.org/documentation/), and please provide us with feedback on your success!

