% -*- latex -*-
%
% $Id: user.tex,v 1.3 2002/07/19 02:28:56 jsquyres Exp $
%
% $COPYRIGHT$
%

\subsection{Using LAM/MPI}
\label{app:lam-mpi-overview}

LAM (Local Area Multicomputer) is an MPI programming environment and
development system for heterogeneous computers on a network. With
LAM/MPI, a dedicated cluster or an existing network computing
infrastructure can act as a single parallel computer.  LAM/MPI is
considered to be ``cluster friendly,'' in that it offers daemon-based
process startup/control as well as fast client-to-client message
passing protocols.  LAM/MPI can use TCP/IP and/or shared memory for
message passing.

LAM features a full implementation of MPI-1, and much of MPI-2.
Compliant applications are source code portable between LAM/MPI and
any other implementation of MPI.  In addition to providing a
high-quality implementation of the MPI standard, LAM/MPI offers
extensive monitoring capabilities to support debugging.  Monitoring
happens on two levels.  First, LAM/MPI has the hooks to allow a
snapshot of process and message status to be taken at any time during
an application run.  This snapshot includes all aspects of
synchronization plus datatype maps/signatures, communicator group
membership, and message contents (see the XMPI application on the main
LAM web site).  On the second level, the MPI library is instrumented
to produce a cummulative record of communication, which can be
visualized either at runtime or post-mortem.

\subsubsection{Notes about OSCAR's LAM/MPI Setup}

The OSCAR environment is able to have multiple MPI implementations
installed simultaneously -- see Section~\ref{app:switcher-overview}
(page~\pageref{app:switcher-overview}) for a description of the
\cmd{switcher} program.  

LAM/MPI is configured on OSCAR to use the Secure Shell (\cmd{ssh}) to
initially start processes on remote nodes.  Normally, using \cmd{ssh}
requires each user to set up cryptographic keys before being able to
execute commands on remote nodes with no password.  The OSCAR setup
process has already taken care of this step for you.  Hence, the LAM
command \cmd{lamboot} should work with no additional setup from the
user.

\subsubsection{Setting up \cmd{switcher} to use LAM/MPI}

In order to use LAM/MPI successfully, you must first ensure that
\cmd{switcher} is set to use LAM/MPI.  First, execute the following
command:

\begin{verbatim}
  $ switcher mpi --show
\end{verbatim}
  
If the result contains a line beginning with ``\file{default}''
followed by a string containing ``\file{lam}'' (e.g.,
``\cmd{lam-6.5.6}''), then you can skip the rest of this section.
Otherwise, execute the following command:

\begin{verbatim}
  $ switcher mpi --list
\end{verbatim}

This shows all the MPI implementations that are available.  Choose one
that contains the name ``\file{lam}'' (e.g., ``\cmd{lam-6.5.6}'') and
run the following command:

\begin{verbatim}
  $ switcher mpi = lam-6.5.6
\end{verbatim}

This will set all {\em future} shells to use LAM/MPI.  In order to
guarantee that all of your login environments contain the proper setup
to use LAM/MPI, it is probably safest to logout and log back in again.
Doing so will guarantee that all of your interactive shells will have
the LAM commands and man pages will be found (i.e., your \cmd{\$PATH}
and \cmd{\$MANPATH} environment variables set properly for LAM/MPI).

Hence, you will be able to execute commands such as ``\cmd{mpirun}''
and ``\cmd{man lamboot}'' without any additional setup.

\subsubsection{General Usage}

The general scheme of using LAM/MPI is as follows:

\begin{enumerate}
\item Use the \cmd{lamboot} command to start up the LAM run-time
  environment (RTE) across a specified set of nodes.  The
  \cmd{lamboot} command takes a single argument: the filename of a
  hostfile containing a list of nodes to run on.  For example:

\begin{verbatim}
$ lamboot my_hostfile
\end{verbatim}

\item Repeat the following steps as many times as necessary:

  \begin{enumerate}
  \item Use the MPI ``wrapper'' compilers (\cmd{mpicc} for C programs,
    \cmd{mpiCC} for C++ programs, and \cmd{mpif77} for fortran
    programs) to compile your application.  These wrapper compilers
    add in all the necessary compiler flags and then invoke the
    underlying ``real'' compiler.  For example:
    
\begin{verbatim}
$ mpicc myprogram.c -o myprogram
\end{verbatim}
    
  \item Use the \cmd{mpirun} command to launch your parallel
    application in the LAM RTE.  For example:

\begin{verbatim}
$ mpirun C myprogram
\end{verbatim}
    
    The \cmd{mpirun} command has many options and arguments -- see the
    man page and/or ``\cmd{mpirun -h}'' for more information.

  \item If your parallel program fails ungracefully, use the
    \cmd{lamclean} command to ``clean'' the LAM RTE and guarantee to
    remove all instances of the running program.

  \end{enumerate}

\item Use the \cmd{lamhalt} command to shut down the LAM RTE.  The
  \cmd{lamhalt} command takes no arguments.
\end{enumerate}

Note that the wrapper compilers are all set to use the corresponding
GNU compilers (\cmd{gcc}, \cmd{g++}, and \cmd{gf77}, respectively).
Attempting to use other compilers may run into difficulties because
their linking styles may be different than what the LAM libraries
expect (particularly for C++ and Fortran compilers).

\subsubsection{More Information}

The LAM/MPI web site (\url{http://www.lam-mpi.org/}) contains much,
much more information about LAM/MPI, including:

\begin{itemize}
\item A large Frequently Asked Questions (FAQ) list
\item Usage tutorials and examples
\item Access to the LAM user's mailing list, including subscription
  instructions and web archives of the list
\end{itemize}

\noindent Make today a LAM/MPI day!

% LocalWords:  Exp
