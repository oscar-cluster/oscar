% -*- latex -*-
%
% $Id: user.tex,v 1.6 2002/10/11 20:10:42 jsquyres Exp $
%
% $COPYRIGHT$
%

\subsection{An overview of \cmd{switcher}}
\label{app:switcher-overview}

Experience has shown that requiring untrained users to manually edit
their ``dot'' files (e.g., \file{\$HOME/.bashrc},
\file{\$HOME/.login}, \file{\$HOME/.logout}, etc.) can result in
damaged user environments.  Side effects of damaged user environments
include:

\begin{itemize}
\item Lost and/or corrupted work
\item Severe frustration / dented furniture
\item Spending large amounts of time debugging ``dot'' files, both by
  the user and the system administrator
\end{itemize}

The OSCAR \cmd{switcher} package is an attempt to provide a simple
mechanism to allow users to manipulate their environment.  The
\cmd{switcher} package provides a convenient command-line interface to
manipulate the inclusion of packages in a user's environment.  Users
are not required to manually edit their ``dot'' files, nor are they
required to know what the inclusion of a given package in the
environment entails.\footnote{Note, however, that it was a requirement
  for the OSCAR \cmd{switcher} package that advanced users should not
  be precluded -- in any way -- from either not using \cmd{switcher},
  or otherwise satisfying their own advanced requirements without
  interference from \cmd{switcher}.}  For example, if a user specifies
that they want LAM/MPI in their environment, \cmd{switcher} will
automatically add the appropriate entries to the \file{\$PATH} and
\file{\$MANPATH} environment variables.

Finally, the OSCAR \cmd{switcher} package provides a two-level set of
defaults: a system-level default and a user-level default.  User-level
defaults (if provided) override corresponding system-level defaults.
This allows a system administrator to (for example) specify which MPI
implementation users should have in their environment by setting the
system-level default.  Specific users, however, may decide that they
want a different implementation in their environment and set their
personal user-level default.

Note, however, that {\em \cmd{switcher} does not change the
  environment of the shell from which it was invoked.}  This is a
critical fact to remember when administrating your personal
environment or the cluster.  While this may seem inconvenient at
first, \cmd{switcher} was specifically designed this way for two
reasons:

\begin{enumerate}
\item If a user inadvertantly damages their environment using
  \cmd{switcher}, there is still [potentially] a shell with an
  undamaged environment (i.e., the one that invoked \cmd{switcher})
  that can be used to fix the problem.
  
\item The \cmd{switcher} package uses the \cmd{modules} package for
  most of the actual environment manipulation (see
  \url{http://modules.sourceforge.net/}).  The \cmd{modules} package
  can be used directly by users (or scripts) who wish to manipulate
  their current environment.
\end{enumerate}

The OSCAR \cmd{switcher} package contains two sub-packages:
\cmd{modules} and \cmd{env-switcher}.  The \cmd{modules} package can
be used by itself (usually for advanced users).  The
\cmd{env-switcher} package provides a persistent \cmd{modules}-based
environment.

%%%%%%%%%%%%%%%%%%%%%%%%%%%%%%%%%%%%%%%%%%%%%%%%%%%%%%%%%%%%%%%%%%%%%%%%%%

\subsubsection{The \cmd{modules} package}

The \cmd{modules} package (see \url{http://modules.sourceforge.net/})
provides an elegant solution for individual packages to install (and
uninstall) themselves from the current environment.  Each OSCAR
package can provide a modulefile that will set (or unset) relevant
environment variables, create (or destroy) shell aliases, etc.

An OSCAR-ized \cmd{modules} RPM is installed during the OSCAR
installation process.  Installation of this RPM has the following
notable effects:

\begin{itemize}
\item Every user shell will be setup for modules -- notably, the
  commands ``\cmd{module}'' and ``\cmd{man module}'' will work as
  expected.
  
\item Guarantee the execution of all modulefiles in a specific
  directory for every shell invocation (including corner cases such as
  non-interactive remote shell invocation by \cmd{rsh}/\cmd{ssh}).
\end{itemize}

Most users will not use any \cmd{modules} commands directly -- they
will only use the \cmd{env-switcher} package.  However, the
\cmd{modules} package can be used directly by advanced users (and
scripts).

%%%%%%%%%%%%%%%%%%%%%%%%%%%%%%%%%%%%%%%%%%%%%%%%%%%%%%%%%%%%%%%%%%%%%%%%%%

\subsubsection{The \cmd{env-switcher} package}

The \cmd{env-switcher} package provides a persistent
\cmd{modulues}-based environment.  That is, \cmd{env-switcher} ensures
to load a consistent set of modules for each shell invocation
(including corner cases such as non-interactive remote shells via
\cmd{rsh}/\cmd{ssh}).  \cmd{env-switcher} is what allows users to
manipulate their environment by using a simple command line interface
-- not by editing ``dot'' files.

It is important to note that {\em using the \cmd{switcher} command
  alters the environment of all {\bf future} shells / user
  environments.  \cmd{switcher} does not change the environment of the
  shell from which it was invoked.}  This may seem seem inconvenient
at first, but was done deliberately.  See the rationale provided at
the beginning of this section for the reasons why.

\cmd{env-switcher} manipulates four different kinds of entities: tags,
attributes, and values.  

\begin{itemize}
\item {\em Tags} are used to group similar software packages.  In
  OSCAR, for example, ``mpi'' is a commonly used tag.
  
\item {\em Names} are strings that indicate individual software
  package names in a tag.

\item Each tag can have zero or more {\em attributes}.  
  
\item An attribute, if defined, must have a single {\em value}.  An
  attribute specifies something about a given tag by having an
  assigned value.
\end{itemize}

There are a few built-in attributes with special meanings (any other
attribute will be ignored by \cmd{env-switcher}, and can therefore be
used to cache arbitrary values).  ``default'' is probably the
most-commonly used attribute -- its value specifies which package will
be loaded (as such, its value is always a name).  For example, setting
the ``default'' attribute on the ``mpi'' tag to a given value will
control which MPI implementation is loaded into the environment.

\cmd{env-switcher} operates at two different levels: system-level and
user-level.  The system-level tags, attributes, and values are stored
in a central location.  User-level tags, attributes, and values are
stored in each user's \file{\$HOME} directory.

When \cmd{env-switcher} looks up entity that it manipulates (for
example, to determine the value of the ``default'' attribute on the
``mpi'' tag), it attempts to resolves the value in a specific
sequence:

\begin{enumerate}
\item Look for a ``default'' attribute value on the ``mpi'' tag in
  the user-level defaults
  
\item Look for a ``default'' attribute value on the ``global'' tag in
  the user-level defaults
  
\item Look for a ``default'' attribute value on the ``mpi'' tag in
  the system-level defaults
  
\item Look for a ``default'' attribute value on the ``global'' tag in
  the system-level defaults
\end{enumerate}

In this way, a four-tiered set of defaults can be effected: specific
user-level, general user-level, specific system-level, and general
system-level.  

The most common \cmd{env-switcher} commands that users will invoke
are:

\begin{enumerate}
\item \cmd{switcher --list}
  
  List all available tags.

\item \cmd{switcher <tag> --list}
  
  List all defined attributes for the tag \cmd{<tag>}.

\item \cmd{switcher <tag> = <value> [--system]} 
  
  A shortcut nomenclature to set the ``default'' attribute on
  \cmd{<tag>} equal to the value \cmd{<value>}.  If the
  \cmd{--system} parameter is used, the change will affect the
  system-level defaults; otherwise, the user's personal user-level
  defaults are changed.

\item \cmd{switcher <tag> --show}

  Show the all attribute / value pairs for the tag \cmd{<tag>}.  The
  values shown will be for attributes that have a resolvable value
  (using the resolution sequence described above).  Hence, this output
  may vary from user to user for a given \cmd{<tag>} depending on the
  values of user-level defaults.

\item \cmd{switcher <tag> --rm-attr <attr> [--system]} 
  
  Remove the attribute \cmd{<attr>} from a given tag.  If the
  \cmd{--system} parameter is used, the change will affect the system
  level defaults; otherwise, the user's personal user-level defaults
  are used.
  
\end{enumerate}

Section~\ref{app:switcher-which-mpi-to-use} shows an example scenario
using the \cmd{switcher} command detailing how to change which MPI
implementation is used, both at the system-level and user-level.

See the man page for \cmd{switcher(1)} and the output of \cmd{switcher
  --help} for more details on the \cmd{switcher} command.

%%%%%%%%%%%%%%%%%%%%%%%%%%%%%%%%%%%%%%%%%%%%%%%%%%%%%%%%%%%%%%%%%%%%%%%%%%

\subsubsection{Which MPI do you want to use?}
\label{app:switcher-which-mpi-to-use}

% -*- latex -*-
%
% $Id: common.tex,v 1.3 2002/06/18 15:30:23 jsquyres Exp $
%
% $COPYRIGHT$
%

Starting with the OSCAR 1.3 series, there is a generalized mechanism
to both set a system-level default MPI implementation, and also to
allow users to override the system-level default with their own choice
of MPI implementation.

This allows multiple MPI implementations to be installed on an OSCAR
cluster (e.g., LAM/MPI and MPICH), yet still provide unambiguous MPI
implementation selection such that ``\cmd{mpicc foo.c -o foo}'' will
give deterministic results.

\subsubsection{Setting the system-level default}

The system-level default MPI implementation can be set in two ways:

\begin{enumerate}
\item During the OSCAR installation, the GUI will prompt asking which
  MPI should be the system-level default.  This will set the default
  for all users on the system who do not provide their own individual
  MPI settings.

\item As root, execute the command:

\begin{verbatim}
  % switcher mpi --list
\end{verbatim}

   This will list all the MPI implementations available.  To set the
   system-level default, execute the command:

\begin{verbatim}
  % switcher mpi = name --system
\end{verbatim}
   
   where ``name'' is one of the names from the output of the
   \cmd{--list} command.
\end{enumerate}

{\bf NOTE:} The current version of \cmd{switcher} is not
``cluster-aware.''  Hence, if you set the system-level default on one
machine in your cluster, it does {\em not} set the system-level
default on all nodes.  In order to propogate the change in any
\cmd{switcher} system-default settings, you must manually execute the
following command:

\begin{verbatim}
  $ cpush /opt/env-switcher-1.0.4/etc/switcher.ini \
       /opt/env-switcher-1.0.4/etc/switcher.ini
\end{verbatim}

The same is {\em not} true for user-level defaults.  Since user-level
defaults are stored in the home directory for each user, the fact that
user home directories are NFS-mounted on all nodes effectively takes
care of ensuring that \cmd{switcher}'s user-level defaults are
available everywhere.

{\bf NOTE:} Using the \cmd{switcher} command to change the default MPI
implementation will modify the \cmd{PATH} and \cmd{MANPATH} for all
{\em future} shell invocations -- it does {\em not} change the
environment of the shell in which it was invoked.  For example:

\begin{verbatim}
  % which mpicc
  /opt/lam-1.2.3/bin/mpicc
  % switcher mpi = mpich-4.5.6 --system
  % which mpicc
  /opt/lam-1.2.3/bin/mpicc
  % csh
  % which mpicc
  /opt/mpich-4.5.6/bin/mpicc
\end{verbatim}

\subsubsection{Setting the user-level default}

Setting a user-level default is essentially the same as setting the
system-level default, except without the \cmd{--system} argument.
This will set the user-level default instead of the system-level
default.  Using the special name \cmd{none} will remove the user-level
default and revert the user to the system-level default.

\begin{verbatim}
  # Set the user's default, overriding the system default:
  % switcher mpi = lam-1.2.3
  # Remove the user's default, and return to whatever the system
  # default is:
  % switcher mpi = none
\end{verbatim}

{\bf WARNING: The \cmd{switcher} command must be used with care!}  It
immediately affects the environment of all future shell invocations
(including the environment of scripts).  To get a full list of options
available, read the \cmd{switcher(1)} man page, and/or run
\cmd{switcher --help}.


%%%%%%%%%%%%%%%%%%%%%%%%%%%%%%%%%%%%%%%%%%%%%%%%%%%%%%%%%%%%%%%%%%%%%%%%%%

\subsubsection{Use \cmd{switcher} with care!}

\cmd{switcher} immediately affects the environment of all future shell
invocations (including the environment of scripts).  To get a full
list of options available, read the \cmd{switcher(1)} man page, and/or
run \cmd{switcher --help}.
