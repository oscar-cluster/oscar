% -*- latex -*-
%
% $Id: install.tex,v 1.7 2002/09/10 14:01:13 ngorsuch Exp $
%
% $COPYRIGHT$
%

\subsection{Packet Filtering with pfilter}
\label{app:pfilter-overview}

{\bf pfilter} is a firewall compiler.  The {\bf pfilter} package is
used to control the packet filtering capabilities available in the
Linux kernel.  It takes in high-level firewall directives, and
produces a complete firewall output commands file that can be turned
on or off like other Linux services.  Like other compilers, {\bf
  pfilter} adds appropriate ``glue'' code to the compiled output.
{\bf pfilter}'s added ``glue'' code consists of common things that are
done by any good firewall, including turning on TCP networking
protective features.

When OSCAR was installed, it merged any pre-existing server 
{\bf pfilter} installation configuration into a new configuration,
and created new client machine {\bf pfilter} configurations. 
The resulting {\bf pfilter} configurations do the following:

\begin{itemize}
\item any network connections that were specifically allowed in the
former server {\bf pfilter} installation configuration are still allowed 
to the main OSCAR server node
\item the main OSCAR server node and all client OSCAR nodes allow\cmd{ssh} logins from anywhere
\item the main OSCAR server node has http access enabled from anywhere
\item any remaining network connections from outside the cluster are blocked
\item the logging of bad network packets to syslog is turned off
\item all network communication of any kind between nodes in the cluster is enabled
\end{itemize}

If the main OSCAR server has two or more network interfaces, {\bf pfilter}
will attempt to determine which network interface is the public interface,
and then enable packet forwarding and network address translation for the 
remaining interfaces if it can.

{\bf pfilter} is turned on by default -- this is considered good
``defense in depth'' security for a cluster (see
Section~\ref{app:security} for more information on cluster security).
If for some reason you need to disable {\bf pfilter} packet filtering
(perhaps for debugging -- disabling it permanently is not recomended),
you can disable {\bf pfilter} subsequent system boots with the
following command:

\begin{verbatim}
  # chkconfig --level=2345 pfilter on
\end{verbatim}

To turn on {\bf pfilter} packet filtering immediately, execute the
following command:

\begin{verbatim}
  # service pfilter start
\end{verbatim}

For more information on how to use and configure {\bf pfilter}, see
the \cmd{pfilter(8)} and \file{pfilter.conf(5)} and
\file{pfilter.rulesets(5)} man pages. 
