% -*- latex -*-
%
% $Id: install.tex,v 1.4 2002/07/17 13:37:19 jsquyres Exp $
%
% $COPYRIGHT$
%

\subsection{Packet Filtering with pfilter}
\label{app:pfilter-overview}

{\bf pfilter} is a firewall compiler.  The {\bf pfilter} package is
used control the packet filtering capabilities available in the Linux
kernel.  It takes in high-level firewall directives, and produces a
complete firewall output commands file that can be turned on or off
like other Linux services.  Like other compilers, {\bf pfilter} adds
appropriate ``glue'' code to the compiled output.  In the case of {\bf
  pfilter}, the added ``glue'' code consists of common things that are
done by any good firewall, including turning on TCP networking
protective features.

When OSCAR was installed, it modified the {\bf pfilter} installation
by merging any pre-existing {\bf pfilter} configuration file and some
new cluster specific commands.  The new configuration file is common
to all machines in the cluster, using conditional expression
capabilities of {\bf pfilter} to allow different types of cluster
machines to have different packet filtering attributes.  OSCAR adds
these things to the {\bf pfilter} configuration file:

\begin{itemize}
\item the default for network interfaces as set to be untrusted
\item bad packet syslog logging is turned off
\item the main OSCAR server node has \cmd{ssh} logins enabled
\item the main OSCAR server node has http access enabled
\item all communication between nodes in the cluster is enabled
\end{itemize}

{\bf pfilter} is turned off by default. To enable {\bf pfilter} packet
filtering on subsequent system boots, execute the following command:

\begin{verbatim}
  # chkconfig --level=2345 pfilter on
\end{verbatim}

To turn on {\bf pfilter} packet filtering immediately, execute the
following command:

\begin{verbatim}
  # service pfilter start
\end{verbatim}

For more information on how to use and configure {\bf pfilter}, see
the \cmd{pfilter(8)} and \file{pfilter.conf(5)} and
\file{pfilter.rulesets(5)} man pages.

