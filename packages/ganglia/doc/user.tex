% -*- latex -*-
%
% $Id: user.tex,v 1.5 2002/09/11 04:32:59 jsquyres Exp $
%
% $COPYRIGHT$
%

\subsection{An overview of \cmd{ganglia}}
\label{app:ganglia-overview}

Ganglia is a real time cluster monitoring tool that uses a protocol
based on XML and XDR across a multicast network to provide a scalable
core tool set.  The cluster monitoring data can be accessed via a web
browser or via the command line (shortly available in OSCAR). The core
Ganglia monitoring toolkit also includes the ability to add user
defined data to the XML streams.

Ganglia was developed at the University of California, Berkeley
Computer Science Division as part of the ongoing clustering research
project named The Millennium Project (and its predecessor The NOW
Project).  It is being developed by Matt Massie
$<$\mailto{massie@cs.berkeley.edu}$>$ and Brent Chun
$<$\mailto{bnc@caltech.edu}$>$. For additional information about the
Ganglia project, see:

\vspace{10pt}
\centerline{\url{http://ganglia.sourceforge.net/}}
\vspace{10pt}

\rpmname{ganglia-php-rrd-client} is a package in OSCAR that sets up
the web page display on the head node of the data gathered by the
\rpmname{ganglia-monitor-core}.  This uses \cmd{rrdtool} and the
\rpmname{perl-XML-Parser} packages included with the OSCAR tool suite
as well as the \rpmname{apache}, \rpmname{php}, and \rpmname{expat}
packages installed with the core Linux distribution on the head node
to build an interactive useful display of the XML/XDR data being
passed across the multicast channel.

Once the cluster head node has the OSCAR tools installed a web browser
can be pointed at:

\vspace{10pt}
\centerline{\url{http://localhost/ganglia/}}
\vspace{10pt}

\noindent to see the ganglia web output. If nothing appears, check to
see if \cmd{apache} has been started along with the \cmd{gmond} and
\cmd{ganglia-php-rrd} processes. The startup init scripts can be found
in \file{/etc/init.d/}.

Future versions of OSCAR will also include the \cmd{ganglia} python
command line interface so access to the ganglia XML/XDR data can be
scripted. This will depend on having \rpmname{python} and the
\rpmname{PyXML} packages installed on the head node.

The current version of the Ganglia tools packaged for OSCAR is:

\begin{itemize}
\item \rpmname{ganglia-monitor-core-2.2.3-2oscar}
\item \rpmname{ganglia-monitor-core-lib-2.2.3-2oscar}
\item \rpmname{ganglia-php-rrd-client-1.0.4-4oscar}
\item \rpmname{ganglia-python-2.0-1oscar}
\item \rpmname{perl-XML-Parser-2.30-32oscar}
\item \rpmname{rrdtool-1.0.35-3oscar}
\item \rpmname{rrdtool-devel-1.0.35-3oscar}
\end{itemize}

The \cmd{ganglia} package has a command line tool based on a python
class. The addition of this tool makes it possible to access all of
the available multicasted ganglia monitoring data that is flowing
across the cluster from the command line or various perl, shell, or
other system admin created scripts.  The tool can be found in:

\vspace{10pt}
\centerline{\cmd{/usr/sbin/ganglia}}
\vspace{10pt}

\noindent and the actual python class file gets installed in:

\vspace{10pt}
\centerline{\cmd{/usr/lib/python2.1/site-packages/gmon/ganglia.py}}
\vspace{10pt}

Running this tool on the command line gives the following usage
message:

\begin{verbatim}
  usage: ganglia < metric > [ metric metric ... ]
\end{verbatim}

where ``metric'' is one of:

\begin{itemize}
\item \cmdarg{sys\_clock}
\item \cmdarg{cpu\_nice}
\item \cmdarg{proc\_run}
\item \cmdarg{boottime}
\item \cmdarg{cpu\_system}
\item \cmdarg{mem\_shared}
\item \cmdarg{os\_release}
\item \cmdarg{cpu\_aidle}
\item \cmdarg{load\_one}
\item \cmdarg{swap\_total}
\item \cmdarg{mem\_buffers}
\item \cmdarg{mem\_cached}
\item \cmdarg{proc\_total}
\item \cmdarg{os\_name}
\item \cmdarg{cpu\_speed}
\item \cmdarg{machine\_type}
\item \cmdarg{mem\_total}
\item \cmdarg{load\_five}
\item \cmdarg{load\_fifteen}
\item \cmdarg{cpu\_user}
\item \cmdarg{swap\_free}
\item \cmdarg{cpu\_num}
\item \cmdarg{cpu\_idle}
\item \cmdarg{mem\_free}
\end{itemize}

\noindent So for instance, the user can do:

\begin{verbatim}
  # ganglia swap_total cpu_speed cpu_num
\end{verbatim}

\noindent to produce the following output:

\begin{verbatim}
  startx          265064          851     1
  xcvs            1582384         733     2
\end{verbatim}

A small Ganglia PHP patch has been incorperated into the Ganglia web
display tool RPMs. This patch was developed by Goneri Le Bouderxi
$<$\mailto{glebouder@mandrakesoft.com}$>$ of MandrakeSoft. It provides
the following enhancements:

\begin{itemize}
\item You can  do selection range on \file{host.php}
\item Network graph on \file{host.php} and \file{index.php}
\item Network general load (in + out) in the ``Cluster Overview''
\item moved \cmdarg{\$rrd\_dir} and \cmdarg{\$rrdtool} from
  \file{graph.php} to \file{ganglia.php}
\end{itemize}
        
Goneri had also incluuded a \cmd{ganglia-network.sh} script to be run
periodically by the \cmd{cron} system to add network statistics to the
Ganglia data stream but this has not been packaged up yet for the
OSCAR cluster tool suite.

