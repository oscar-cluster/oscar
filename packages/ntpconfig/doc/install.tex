% -*- latex -*-
%
% $Id: install.tex,v 1.4 2003/03/06 23:08:45 brechin Exp $
%
% $COPYRIGHT$
%

\subsection{Managing NTP for the OSCAR Server and Clients}
\label{app:ntp-overview}

NTP is the Network Time Protocol which is used to synchronize the
computer clock to external sources of time.  The \file{ntpd} daemon
can operate as a client (by connecting to other NTP servers to get the
current time) and as a server (by providing the current time to other
NTP clients).

\bigskip

OSCAR uses NTP to synchronize:
\begin{itemize}
\item the server to an external source of time (if the server is
  connected to the internet), and
\item the clients to the OSCAR server (so that the clients do not need
  to be connected to the internet).
\end{itemize}

For the OSCAR server, NTP is configured to contact several public NTP
servers for time synchronization.  If none of these external servers
can be contacted (because the server is not connected to the internet
for example), it will still be used by the OSCAR clients as a source
for the current time.  Thus, your OSCAR cluster will be locally
synchronized even if it is not in-sync with the rest of the world.

For each OSCAR client, NTP is configured to contact only the OSCAR
server.  If you want to have your clients connect to other public NTP
servers, you will have to edit the NTP configuration files.

\subsubsection{Configuring NTP}

By default, the OSCAR server is set up to use several public NTP
servers for its time source, and the OSCAR clients are set up to use
the OSCAR server for their time sources.  If you want to change the
servers used for time synchronization, you will need to edit two
configuration files: \file{/etc/ntp.conf} and
\file{/etc/ntp/step-tickers}.

For the \file{/etc/ntp.conf} file, place entries of the form ``{\tt
  server SERVER\_NAME\_OR\_IP}'' at the top of the file, one entry per
line.  You can have as many {\tt server} lines as you want.  However,
ALL of the listed servers are consulted for time synchronization, so
you may want to limit the number of servers to three or less.  Here is
an example of the first few lines of a typical \file{/etc/ntp.conf}
file:

\begin{verbatim}
     # These are some servers for use by the ntpd daemon.
     server 130.126.24.24
     server ntp0.cornell.edu
     server ntp.cmr.gov
\end{verbatim}

Notice that you may use either FQDNs (fully qualified domain names) or
IP addresses.  Most NTP servers prefer that you use FQDNs in case they
change the server IP address, but typically these IP addresses are
fairly static.

For the \file{/etc/ntp/step-tickers} file, place these NTP servers on
a single line separated by spaces.  Here is an example of the
\file{/etc/ntp/step-tickers} file using the same servers:

\begin{verbatim}
     130.126.24.24 ntp0.cornell.edu ntp.cmr.gov
\end{verbatim}

The servers used in the \file{/etc/ntp.conf} file and the
\file{/etc/ntp/step-tickers} file do not have to be the same.  The
\file{/etc/ntp/step-tickers} file is used to force the clock to be set
correctly at boot time, while the \file{/etc/ntp.conf} file is used to
adjust the clock in small increments while the system is running.

\subsubsection{Enabling/Disabling the NTP Service}

By default, the \file{ntpd} daemon is configured to start at boot time
in run levels 2 through 5.  If for some reason you want to disable NTP
without actually uninstalling it, execute the following commands:

\begin{verbatim}
 # /etc/init.d/ntpd stop
 # /sbin/chkconfig --level 2345 ntpd off
\end{verbatim}

This will not only stop any currently running \file{ntpd} daemon, but
also prevent NTP from starting up at boot time.

\bigskip 

{\bf NOTE:} You must be {\tt root} to execute these commands.

\bigskip 

To restart NTP and make NTP start up at boot time, execute the following
commands:

\begin{verbatim}
 # /etc/init.d/ntpd restart
 # /sbin/chkconfig --level 2345 ntpd on
\end{verbatim}

For more information on NTP, see the (rather lengthy) documentation at
\url{http://www.ntp.org/}.


