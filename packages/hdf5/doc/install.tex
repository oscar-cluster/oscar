% -*- latex -*-
%
% $Id: install.tex,v 1.1 2002/08/28 04:39:03 jenos Exp $
%
% $COPYRIGHT$
%

\subsection{HDF5}
\label{app:hdf5-overview}

HDF5 is a Hierarchical Data Format product consisting of a data format 
specification and a supporting library implementation. The HDF5 library
is included with the OSCAR distribution.


HDF5 includes the following features which make it a widely-used 
scientific data format:
   --supports a user-defined hierarchical grouping structure with 
     varied datatypes and attributes to organize large and varied data 
         in research, development, and production environments,
   --runs on parallel computing platforms using MPI I/O,
   --supports an all-encompassing variety of datatypes, including 
         compound and user-defined datatypes,
   --supports very large files (theoretically terabyte and larger, 
         though the outer limits have not been tested), 


HDF5 is fully documented.  See
   <<<JEREMY>>>/hdf5/doc/index.html
   for the documentation that accompanied the release of HDF5 that is
   installed with this OSCAR distribution. 
For documentation of the current release of HDF5, as served from the 
   HDF Group's website, see
   http://hdf.ncsa.uiuc.edu/HDF5/doc/


For further information regarding HDF5, such as lists of current users,
supporting agencies, and applications that employ the library, see
    http://hdf.ncsa.uiuc.edu/HDF5/


OSCAR installs the HDF5 library at /opt/hdf5-oscar-1.4.4-post2/lib/

When compiling HDF5, two compiler issues must be considered:

First, there is a bug in the gcc 2.96 compiler which affects HDF5. 
To work properly, both MPICH and HDF5 must be compiled with a compiler 
other than 2.96. The HDF5 Group suggests using gcc 2.95.3 for both.

Second, when compiling MPICH on a Linux system with kernels 2.4 or
greater, you must specify that it should support >2GB file sizes. To
do so, configure the installation of MPICH with the following
configuration command-line option:
-cflags="-D_LARGEFILE_SOURCE -D_LARGEFILE64_SOURCE -D_FILE_OFFSET_BITS=64"

