\documentclass{letter}
\IfFileExists{TeXmacs.sty}
  {\usepackage{TeXmacs}}
  {\usepackage{/usr/local/share/TeXmacs-1.0/misc/latex/TeXmacs}}
\newcommand{\chapter}[1]{\tmsection{\begin{center}\huge #1\end{center}}}
\newcommand{\section}[1]{\tmsection{\LARGE #1}}
\newcommand{\subsection}[1]{\tmsection{\Large #1}}
\newcommand{\subsubsection}[1]{\tmsection{\large #1}}
\newcommand{\paragraph}[1]{\tmparagraph{#1}}
\newcommand{\subparagraph}[1]{\tmparagraph{#1}}

\begin{document}

\subsection{Using The Portable Batch System (PBS)}\label{app:pbs-overview}

All PBS commands can be found under /usr/local/pbs/bin on the OSCAR head node.
There are man pages available for these commands, but here are the most
popular and some basic options:

\%Need bullets on commands

qsub             submits jobs to PBS

qdel              deletes PBS jobs

qstat (-n)       displays current job status and node associations

pbsnodes (-a)  displays node status

pbsdsh           distributed process launcher{\hspace*{\fill}}

Submitting a Job

The PBS client daemons are called "moms".  They are each polled by the PBS
server.

qsub:  Not necessarily intuitive.  here are some things to know:

\%Need bullets until "A sample qsub..."

only accepts a script for a target

the target script cannot have arguments

target script is only launched on one node, therefore the script is
responsible for launching all processes

pbsdsh can be used within the script to launch on all allocated processors and
nodes (specified as arguments to qsub).  Other methods of parallel launch
exist, such as mpirun.

Job parameters can be specifed to qsub on the command line, or within the
submitted script.  You can get a good start by looking at examples provided by
the OSCAR test suite.  Ask your sysadmin if you would like to see these.  They
can likely be found in the home directory of the "oscartst" user.

A sample qsub line and target script:

\% I tried to put a backslash in on the wrapped command line, but couldn't

qsub -N my\_jobname -e my\_stderr.txt -o my\_stdout.txt -q workq -l
nodes=X:ppn=Y:all,walltime=1:00:00 my\_script.sh

script.sh:

\#!/bin/sh

echo Launchnode is $\tmop{hostname}$

pbsdsh /my\_path/my\_executable

\#done



Or... an equivalent alternative is to include the qsub parameters in the
script:

qsub -l nodes=X:ppn=Y:all,walltime=1:00:00 script.sh

script.sh:

\#!/bin/sh

\#PBS -N my\_jobname

\#PBS -o my\_stdout.txt

\#PBS -e my\_stderr.txt

\#PBS -q workq

\#echo Launchnode is $\tmop{hostname}$

\#pbsdsh /my\_path/my\_executable

\#done

*    "all" is an optional specification of a node attribute, or "resource"

**  "workq" is a default queue name that is used in OSCAR clusters

*** 1:00:00 is in HH:MM:SS

\%More to come later

\end{document}
