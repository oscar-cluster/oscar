% -*- latex -*-
%
% $Id: user.tex,v 1.3 2002/06/14 05:15:45 jsquyres Exp $
%
% $COPYRIGHT$
%

\subsection{Using The Portable Batch System (PBS)}
\label{app:pbs-overview}

All PBS commands can be found under \file{/usr/local/pbs/bin} on the
OSCAR head node.  There are man pages available for these commands,
but here are the most popular and some basic options:

\begin{itemize}
\item \cmd{qsub}: submits jobs to PBS

\item \cmd{qdel}: deletes PBS jobs
  
\item \cmd{qstat [-n]}: displays current job status and node
  associations

\item \cmd{pbsnodes [-a]}: displays node status
  
\item \cmd{pbsdsh}: distributed process launcher
\end{itemize}

\subsubsection{PBS Architecture}

The PBS architecture consists of three major components:

\begin{itemize}
\item The PBS server.  This runs on the OSCAR head node.  It controls
  the submission and running of jobs.
  
\item The Maui scheduler.  This takes care of scheduling jobs across
  the cluster according sophisticated algorithms.

\item A ``mom'' daemon on each cluster node.  The moms are responsible
  for actually starting and stopping jobs on the client nodes.
\end{itemize}

\subsubsection{Submitting a PBS Job}

The \cmd{qsub} command is not necessarily intuitive.  Here are some
handy tips to know:

\begin{itemize}
\item Be sure to read the \cmd{qsub} man page.

\item \cmd{qsub} only accepts a script filename for a target.
  
\item The target script cannot take any command line arguments.
  
\item For parallel jobs, the target script is only launched on one
  node.  Therefore the script is responsible for launching all
  processes in the parallel job.
  
\item Alternatively, the \cmd{pbsdsh} command can be used within the
  script to launch an executable on all allocated processors and nodes
  (specified as arguments to \cmd{qsub}).  Other methods of parallel
  launch exist, such as commands like \cmd{mpirun}.
  
\item Job parameters can be specifed to \cmd{qsub} on the command
  line, or within the submitted script.  You can get a good start by
  looking at examples provided by the OSCAR test suite.  Ask your
  system administrator if you would like to see these.  They can
  likely be found in the home directory of the ``\user{oscartst}''
  user.
\end{itemize}

Here is a sample \cmd{qsub} command line:

\begin{verbatim}
$ qsub -N my_jobname -e my_stderr.txt -o my_stdout.txt -q workq -l \
       nodes=X:ppn=Y:all,walltime=1:00:00 my_script.sh
\end{verbatim}

Here is the contents of the \file{script.sh} file:

\begin{verbatim}
#!/bin/sh

echo Launchnode is `hostname`
pbsdsh /path/to/my_executable

# All done
\end{verbatim}

Alternatively, you can specify most of the qsub parameters in
\file{script.sh} itself:

\begin{verbatim}
$ qsub -l nodes=X:ppn=Y:all,walltime=1:00:00 script.sh
\end{verbatim}

Here is the contents of the \file{script.sh} file:

\begin{verbatim}
#!/bin/sh
#PBS -N my_jobname
#PBS -o my_stdout.txt
#PBS -e my_stderr.txt
#PBS -q workq

echo Launchnode is `hostname`
pbsdsh /path/to/my_executable

# All done
\end{verbatim}

\subsubsection{Other Points to Cover...}

\begin{discuss}
  Jeremy will write more stuff here -- major topics to write about are
  listed below.
\end{discuss}

\begin{itemize}
\item ``all'' is an optional specification of a node attribute, or
  ``resource''
  
\item ``workq'' is a default queue name that is used in OSCAR clusters

\item 1:00:00 is in HH:MM:SS
\end{itemize}

% LocalWords:  Exp
