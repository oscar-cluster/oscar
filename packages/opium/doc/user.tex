% -*- latex -*-
%
% $Id: user.tex,v 1.6 2002/07/19 02:28:56 jsquyres Exp $
%
% $COPYRIGHT$
%

\subsection{The OSCAR Password Installer and User Management (OPIUM)}
\label{app:opium-overview}

The OPIUM package includes facilities which synchronize the cluster's
accounts and configures \cmd{ssh} for users.  The user account
synchronization may only be run by root, and is automatically
triggered each time your sysadmin adds or removes users or groups.
OPIUM configures \cmd{ssh} such that every user can traverse the
cluster securely without entering a password, once logged on to the
head node.  This is done using \cmd{ssh} user keys, in the \file{.ssh}
folder in your home directory.  It is not recommended that you make
changes here unless you know what you are doing.

\subsubsection{Caveats for OSCAR \oscarversion}

\begin{itemize}
\item Your \cmd{ssh} keys are automatically generated the first time
  you log on to the OSCAR cluster's head node.  Your first log on must
  be on the head node if you wish for your \cmd{ssh} keys to be
  auto-generated.  This will change in future versions of OSCAR.
  
\item \cmd{chsh} and \cmd{passwd} commands will not trigger a
  synchronization, so you may have to ask your sysadmin to invoke a
  sync, or just wait until the next user is created, which will
  trigger a sync anyway.  Keeping your password synced shouldn't be as
  much of an issue anyway, since \cmd{ssh} makes it so they're not
  really required.  If synchronization becomes a troublesome area for
  you, you may want to ask your sysadmin to add \cmd{sync\_users} to
  \user{root}'s crontab.
\end{itemize}

