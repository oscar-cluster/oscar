% -*- latex -*-
%
% $Id: install.tex,v 1.2 2002/06/14 23:36:12 jenos Exp $
%
% $COPYRIGHT$
%

\subsection{The OSCAR Password Installer and User Management (OPIUM)}
\label{app:opium-overview}

OPIUM is OSCAR's facility to manage user accounts on the cluster.  Currently,
OPIUM does this by replicating files on a triggered basis.  The user account
files are copied to each node each time certain commands are invoked.
The files:
\begin{itemize}
/etc/passwd
/etc/shadow
/etc/group
/etc/gshadow
\end{itemize}
It is possible to add your own files to this list, by editing the command
script which does the work.
The command is \cmd{/usr/sbin/sync_users}.  It is triggered by wrapper
scripts around the user management commands in Linux.  The wrapped scripts:
\begin{itemize}
/usr/sbin/useradd
/usr/sbin/userdel
/usr/sbin/usermod
/usr/sbin/groupadd
/usr/sbin/groupdel
/usr/sbin/groupmod
\end{itemize}

Problems with sync_users:
\begin{itemize}
chsh and passwd commands don't trigger a sync (a cron entry is a recommended
workaround)
If shadow utils rpms are updated (useradd, userdel, etc), the wrapped scripts
are blown away, thus removing the sync_users triggers.
The ''-c'' option on \cmd{useradd}(used for comments) doesn't work due to
the wrapped script not handling quotation marks properly.
\end{itemize}
In OSCAR 1.3.1, these issues will all be fixed.  The shadow utils commands
will no longer be wrapped, and a cron job will be watching a given list of
files and syncing only if that file changes.

SSH Configuration
The OPIUM package also handles the ssh key setup.  This is done by placing
startup scripts in /etc/profile.d/ which generate ssh user key files if 
they don't exist.  It also generates authorized_keys files which enable
users to traverse the cluster securely without entering passwords.  Since
these files reside in the users' home directories, it is necessary for
the home directory to be mounted on a node in order for a user to ssh to
it.  Also, a user must log on to the head node in order for the keys to
be generated, as the /etc/profile.d/ scripts are not installed on the
compute nodes in OSCAR 1.3.  SSH1 and SSH2 key files are supported.  



