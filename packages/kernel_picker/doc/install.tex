%% -*- latex -*-
%%
%% $Id: install.tex,v 1.4 2002/09/11 07:17:19 jsquyres Exp $
%%
%% $COPYRIGHT$
%%

\subsection{Selecting a Different Kernel with \rpmname{kernel\_picker}}
\label{app:kernel-picker-overview}

\cmd{kernel\_picker} is a Perl script which allows a user to install a
given kernel into an OSCAR image different from the one which is
installed by default.  After step 1, but before step 2, you can run
\cmd{kernel\_picker} to substitute a given kernel into your OSCAR
(SIS) image.  If executed with no command line options, you will be
prompted for all information.  If you use any command line options,
the program will assume that you know what you are doing and prompt
you {\em only} for information which is required for correct
execution.

The \cmd{kernel\_picker} program assumes that the optional OSCAR image
files you wish to use reside in a subdirectory in the
\file{/var/lib/systemimager/images} directory.  By default, the
original OSCAR image is in a subdirectory named \file{oscarimage}.

\cmd{kernel\_picker} is installed in the
\file{/opt/kernel\_picker/bin} directory.  Documentation is available
in HTML, PostScript, PDF, plain text, and manpage formats.  To see the
manpage documentation, type the following at a Unix command prompt:

\begin{verbatim}
  $ man /opt/kernel_picker/man/man1/kernel_picker.1
\end{verbatim}
