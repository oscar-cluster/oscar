% $Id: pkg-layout.tex,v 1.2 2003/09/08 22:31:34 naughtont Exp $

\section{Package Layout}
\label{sect:pkg-layout}

As OSCAR evolved it became obvious that the mechanism to configure and
install a cluster needed to be cleanly seperated from the software that was
to be installed.  The approach taken was to create \emph{OSCAR Packages}.
The OSCAR Package layout is geared toward making things as simple as
possible for package authors.  So, in its simplest form an OSCAR Package is
an RPM~\footnote{A binary RPM compiled for an OSCAR supported Linux
distribution.}.  However, most software requires further configuration for
a cluster environment so additional scripts, documentation, etc. may be
added.  The basic directory structure for an OSCAR Package is as follows:

\begin{quote}
\begin{description}
    \item[\file{config.xml}] -- meta file with description, version, etc.
    \item[\file{RPMS/}] -- directory containing binary RPM(s) for the package
    \item[\file{SRPMS/}] -- directory containing source RPM(s) used to build
                            the package
    \item[\file{scripts/}] --  set of scripts that run at particular times
                     during the installation/configuration of the cluster
    \item[\file{testing/}] -- unit test scripts for the package
    \item[\file{doc/}] -- documentation and/or license information
\end{description}
\end{quote}


The packaging API provides authors the ability to make use of scripts to
configure the cluster software outside of the RPM itself.  The scripts fire
at different stages of the installation process and test scripts can be
added to verify the process. Additionally, an OSCAR Package Downloader
(OPD) (see Section~\ref{sect:opd}) is provided to simplify acquisition of
new packages.



\subsection{\file{config.xml}}
This file provides the version/description information similar to that
obtained from an \cmd{rpm --query --info \emph{RPM\_NAME}}.  Additionally,
the list of RPMS and where they are to be installed, e.g., server or clients.
There are also XML fields for expressing details about the distribution the 
binary RPMS were compiled for as well as simple dependencies on other OSCAR
Packages, e.g., Env-Switcher.  If this meta file is not included a
simplistic default is used--install all files in \file{RPMS/} on all
machines in cluster.  The available XML tags for use in this file are
listed in Table~\ref{tab:pkg-xml-tags}.

%  $Id: pkg-xml-table.tex,v 1.1 2003/09/08 22:31:35 naughtont Exp $

% TJN: Note I'm using the "(" symbol for the \verb delimiters b/c
%      I don't believe this symbol will occur in any of the XML elements
%      or attributes.  Any other non-occuring char could be used.
%
% TJN: I need some way to convey heirarchy, currently using multi-columns
%      xml   &      &  descr... \\
%      dtd   &      &  descr... \\
%      oscar &      &  descr... \\
%            & name &  descr... \\
%
% TJN: Somehow I need to display <tag></tag>  VS. <tag/>
%      Also, certain things like <description></description> must be
%      on same line as closing multi-line text.  Need to check this out
%      I believe it's a product of using XML::Simple.
%
% TJN: Also, I need to denote optional vs. required elements

\begin{table}[htbp]
  \begin{center}
  \begin{tabular}{|l|l|l|} \hline
  {\bfseries Top-Level Tag} & {\bfseries Sub Tag} & {\bfseries Description}\\\hline
  \hline
  \verb(<?xml version="1.0"?>( & &  Opening XML Version string         \\ \hline
  %\verb(<!DOCTYPE package SYSTEM "../package.dtd">( & & DTD Include line\\\hline

  \verb(<!DOCTYPE ...>( & & Document Type Definition (DTD) Include line\\ \hline
  \verb(<oscar>(   &   & OSCAR Package Element                         \\ \hline
  \verb(<name>(    &   & Name of the software package                  \\ \hline

  %\verb(<version>( &   & Version of the software package               \\ \hline
  \verb(<version>( &   & Version of the software package           \\\cline{2-3}
     & \verb(<major>(  & Major version of the software package     \\\cline{2-3}
     & \verb(<minor>(  & Minor version of the software package     \\\cline{2-3}
     & \verb(<subversion>( & Sub-version of the software package   \\\cline{2-3}
     & \verb(<release>(    & Release number of the software package\\\cline{2-3}
     & \verb(<epoch>(      & Release number of the software package\\\hline

  \verb(<class>(   &   & Classification of package, e.g. third-party   \\ \hline
  \verb(<summary>( &   & Brief one-line description of package         \\ \hline
  \verb(<license>( &   & Software licence for the package              \\ \hline
  \verb(<group>(   &   & RPM-style software group, e.g., Application/System\\\hline
  \verb(<url>(     &   & Homepage for software package                 \\ \hline

  %\verb(<packager>(&   & OSCAR package author                         \\ \hline
  \verb(<packager>(&   & OSCAR package author                      \\\cline{2-3}
     & \verb(<name>(   & Name of package author                    \\\cline{2-3}
     & \verb(<email>(  & Email address of package author           \\\hline

  %\verb(<maintainer>(& & OSCAR package author                         \\ \hline
  \verb(<maintainer>(& & OSCAR package author                      \\\cline{2-3}
     & \verb(<name>(   & Name of package author                    \\\cline{2-3}
     & \verb(<email>(  & Email address of package author           \\\hline

  \verb(<description>(&& Brief description of the package              \\ \hline

  %\verb(<rpmlist>(&    & List RPMS included in the rpmlist             \\ \hline
  \verb(<rpmlist>(&    & List RPMS included in the rpmlist         \\\cline{2-3}
     & \verb(<filter>( & Filter applied to all RPM in list         \\\cline{2-3}
     & \verb(<rpm>(    & Name of RPM (without version)             \\\hline

  %\verb(<filter>(&     & Filter Attributes used on RPMLIST              \\\hline
  \verb(<filter>(&     & Filter Attributes used on RPMLIST            \\\cline{2-3}
     & \verb(group=(   & Attribute specifying target location for RPMS\\
     &                 & e.g., ``oscar\_server'' or ``oscar\_clients''\\\cline{2-3}
     & \verb(distribution=(& Attribute specifying supported distribution   \\
     &                 & e.g., ``redhat'', ``mandrake'', ``suse'', ``rhas''\\
     &                 & see also: \emph{lib/OSCAR/Distro.pm}              \\\cline{2-3}
     & \verb(distribution_version=(& Attribute specifying supported distribution version  \\
     &                 & e.g., ``7.3'', ``9'', ``2.1AS''                   \\\hline

  %\verb(<requires>(&   & List requirement/dependency                    \\\hline
  \verb(<requires>(&   & List requirement/dependency                \\\cline{2-3}
     & \verb(type=(    & Tag specifying type of requirement         \\
     &                 &  e.g., ``package'' (OSCAR Package),        \\
     &                 &  \emph{``rpm'' (not implemented yet)}      \\\hline

  %\verb(<package>( &   & Package specific ODA material                \\\hline 
  \verb(<package>( &    & Package specific ODA material            \\\cline{2-3}
     & \verb(<shortcut>(& ODA shortcut defined by a package            \\\hline

  \end{tabular}
  \caption{Overview of existing XML tags for OSCAR Packages} 
  \label{tab:pkg-xml-tags}
  \end{center}  
\end{table}



\subsection{RPMS \& SRPMS}

