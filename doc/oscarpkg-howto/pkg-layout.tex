% $Id: pkg-layout.tex,v 1.3 2003/09/09 06:21:17 naughtont Exp $

\section{Package Layout}
\label{sect:pkg-layout}

As OSCAR evolved it became obvious that the mechanism to configure and
install a cluster needed to be cleanly seperated from the software that was
to be installed.  The approach taken was to create \emph{OSCAR Packages}.
The OSCAR Package layout is geared toward making things as simple as
possible for package authors.  So, in its simplest form an OSCAR Package is
an RPM\footnote{A binary RPM compiled for an OSCAR supported Linux
distribution.}.  However, most software requires further configuration for
a cluster environment so additional scripts, documentation, etc. may be
added.  The basic directory structure for an OSCAR Package is as follows
along with a list of the available environment variables in 
Table~\ref{tab:oscar-envvars}.

\begin{quote}
\begin{description}
  \item[\file{config.xml}] -- meta file with description, version, etc.
  \item[\directory{RPMS/}] -- directory containing binary RPM(s) for the package
  \item[\directory{SRPMS/}] -- directory containing source RPM(s) used to build
                            the package
  \item[\directory{scripts/}] --  set of scripts that run at particular times
                     during the installation/configuration of the cluster
  \item[\directory{testing/}] -- unit test scripts for the package
  \item[\directory{doc/}] -- documentation and/or license information
\end{description}
\end{quote}

% Table with all currently supported/recognized Environment Variables
%  $Id: oscar-envvars-table.tex,v 1.2 2003/09/10 03:31:28 naughtont Exp $


\begin{table}[htbp]
  \begin{center}
  \begin{tabular}{|l|l|} \hline
  {\bfseries Environment Variable} & {\bfseries Description} \\\hline
  \hline
  \envvar{OSCAR\_HOME} & Defines top-level directory for OSCAR installation \\\hline

  \envvar{OSCAR\_PACKAGE\_HOME} & Points to a packages directory (used with OPD) \\\hline

  \envvar{OSCAR\_HEAD\_INTERNAL\_INTERFACE} & Contains the value provided to 'install\_cluster \emph{ethX}' \\\hline

  \envvar{OSCAR\_RPMPOOL} & Defines an alt. location for RPM (default = \directory{/tftpboot/rpm})  \\\hline

  \end{tabular}
  \caption{Environment variables currently recognized/used by OSCAR.} 
  \label{tab:oscar-envvars}
  \end{center}  
\end{table}


The packaging API provides authors the ability to make use of scripts to
configure the cluster software outside of the RPM itself.  The scripts fire
at different stages of the installation process and \directory{testing/}
scripts can be added to verify the process.  These are detailed in
Table~\ref{tab:pkg-scripts} and Section~\ref{sect:pkg-testing}.  Lastly, an
OSCAR Package Downloader (OPD) tool is provided to simplify acquisition of
new packages (see Section~\ref{sect:opd}).




\subsection{\file{config.xml}}

This file provides the version/description information for the
package\~footnote{This information is similar to that obtained from an
\cmd{rpm --query --info \emph{RPM\_NAME}}}.  Additionally, the list of RPMS
and where they are to be installed, e.g., server or clients.  There are
also XML fields for expressing details about the distribution the binary
RPMS were compiled for as well as simple dependencies on other OSCAR
Packages, e.g., Env-Switcher.  If this meta file is not included a
simplistic default is used--install all files in \directory{RPMS/} on all
machines in cluster.  The available XML tags for use in this file are
listed in Table~\ref{tab:pkg-xml-tags}.


% Table with all currently supported XML tags & brief description
%  $Id: pkg-xml-table.tex,v 1.2 2003/09/10 03:31:28 naughtont Exp $

% This contains the available XML tags in a tabular form.  
% Note, those editing this file should expand their display beyond 
% 80 chars b/c of the length of each table row...deal with it.

% TJN: Note I'm using the "(" symbol for the \verb delimiters b/c
%      I don't believe this symbol will occur in any of the XML elements
%      or attributes.  Any other non-occuring char could be used.
%
% TJN: I need some way to convey heirarchy, currently using multi-columns
%      xml   &      &  descr... \\
%      dtd   &      &  descr... \\
%      oscar &      &  descr... \\
%            & name &  descr... \\
%
% TJN: Somehow I need to display <tag></tag>  VS. <tag/>
%      Also, certain things like <description></description> must be
%      on same line as closing multi-line text.  Need to check this out
%      I believe it's a product of using XML::Simple.
%
% TJN: Also, I need to denote optional vs. required elements

\begin{table}[htbp]
  \begin{center}
  \begin{tabular}{|c|l|l|l|} \hline
  {\bfseries Require} & {\bfseries Elements} & {\bfseries Sub Elements} & {\bfseries Description}\\\hline
  \hline

% TJN: These are not particular to OSCAR elements, just XML-ness.
%  Y & \verb(<?xml version="1.0"?>( & &  Opening XML Element/tag           \\ \hline
%  N & \verb(<!DOCTYPE oscar ...>( & & Document Type Definition (DTD) include \\ \hline


  Y & \verb(oscar(   &   & Top-level OSCAR Package element               \\ \hline
  Y & \verb(name(    &   & Name of the software package                  \\ \hline


  Y & \verb(version(& & Version of the software package           \\\cline{3-4}
    & & \verb(major(  & Major version of the software package     \\\cline{3-4}
    & & \verb(minor(  & Minor version of the software package     \\\cline{3-4}
    & & \verb(subversion( & Sub-version of the software package   \\\cline{3-4}
    & & \verb(release(    & Release number of the software package\\\cline{3-4}
    & & \verb(epoch(      & Release number of the software package\\\hline


  Y & \verb(class(   &   & Classification of package, e.g. third-party   \\ \hline
  Y & \verb(summary( &   & Brief one-line description of package         \\ \hline
  N & \verb(license( &   & Software licence for the package              \\ \hline
  N & \verb(group(   &   & RPM-style software group, e.g., Application/System\\\hline
  N & \verb(url(     &   & Homepage for software package                 \\ \hline


  N & \verb(packager(&   & OSCAR package author                      \\\cline{3-4}
    & & \verb(name(   & Name of package author                    \\\cline{3-4}
    & & \verb(email(  & Email address of package author           \\\hline


  N & \verb(maintainer(& & OSCAR package author                      \\\cline{3-4}
    & & \verb(name(   & Name of package author                    \\\cline{3-4}
    & & \verb(email(  & Email address of package author           \\\hline


  Y & \verb(description(&& Brief description of the package              \\ \hline


  N & \verb(rpmlist(&    & List RPMS included in the rpmlist         \\\cline{3-4}
    & & \verb(filter( & Filter applied to all RPM in list         \\\cline{3-4}
    & & \verb(rpm(    & Name of RPM (without version)             \\\hline


  N & \verb(filter(&     & Filter Attributes used on RPMLIST                 \\\cline{3-4}
    & & \verb(group=(   & Attribute specifying target location for RPMS     \\
    & &                 & e.g., ``oscar\_server'' or ``oscar\_clients''     \\\cline{3-4}
    & & \verb(distribution=(& Attribute specifying supported distribution   \\
    & &                 & e.g., ``redhat'', ``mandrake'', ``suse'', ``rhas''\\
    & &                 & see also: \emph{lib/OSCAR/Distro.pm}              \\\cline{3-4}
    & & \verb(distribution_version=(& Attribute specifying supported distribution version  \\
    & &                 & e.g., ``7.3'', ``9'', ``2.1AS''                   \\\cline{3-4}
    & & \verb(arch=(    & Architecture the RPM was compiled to support      \\
    & &                 & e.g., ``ia32'', ``ia64''                          \\\hline


  N & \verb(requires(&   & List requirement/dependency                \\\cline{3-4}
    & & \verb(type=(    & Tag specifying type of requirement         \\
    & &                 &  e.g., ``package'' (OSCAR Package),        \\
    & &                 &  \emph{``rpm'' (not implemented yet)}      \\\hline



  N & \verb(package( &    & Package specific namespace (see detail text) \\\hline

  N & \verb(oda(     &    & OSCAR Database (ODA) defines for package     \\\cline{3-4}
    & & \verb(shortcut(   & ODA shortcut defined by a package            \\\hline


  \end{tabular}
  \caption{Overview of existing XML tags for OSCAR Packages} 
  \label{tab:pkg-xml-tags}
  \end{center}  
\end{table}



\subsection{RPMS \& SRPMS}

The binary RPMS are placed obviously enough in the \directory{RPMS}
directory.  The OSCAR Wizard copies all files listed in the \xmltag{rpmlist} 
from this \directory{RPMS} directory to the \directory{/tftpboot/rpm} directory 
or alternately the directory specified by the environment variable 
\envvar{OSCAR\_RPMPOOL}. 

\begin{verse}
   {\bfseries Notice: } As of OSCAR-2.3 the \xmltag{filter} tag does not yet
   support the \xmlval{subdir} attribute, which is used to specify a
   directory for the \xmltag{rpm}'s in the \xmltag{rpmlist}.  Due to this
   limitation, some packages use a \file{setup} script to copy the
   appropriate RPMS to this \directory{RPMS} area based on the value
   returned from Perl method \verb={OSCAR::Distro::which_distro_server();=.
\end{verse}



% TJN: finish this section
\subsection{scripts}

The currently available API scripts are listed in Table~\ref{tab:pkg-scripts}.
% Table with all currently supported API script & brief description.
%  $Id: pkg-scripts-table.tex,v 1.2 2003/09/26 06:52:32 naughtont Exp $


\begin{table}[htbp]
  \begin{center}
  \begin{tabular}{|c|l|l|} \hline
%%%
  {\bfseries Seq\# } & {\bfseries Script Name} 
		& {\bfseries Description} 
		\\\hline
%%%
  \hline
  1 & \file{setup}                      
		& Perform any package setup 
		\\ \hline
%
  2 & \file{pre\_configure}             
		& Prepare package configuration (dynamic user input)
		\\ \hline
%
  3 & \file{post\_configure}            
		& Process results from package configuration (user input results)
		\\ \hline
%
  4 & \file{post\_server\_rpm\_install} 
		&  Perform ``out of RPM'' operations on server (limited cluster knowledge)
		\\ \hline
%
  5 & \file{post\_client\_rpm\_install} 
		&  Perform ``out of RPM'' operations on client (limited cluster knowledge)
		\\ \hline
%
  6 & \file{post\_clients}              
		&  Perform configurations with knowledge about cluster nodes (pre node install)
		\\ \hline
%
  7 & \file{post\_install}              
		&  Perform final configurations with fully install/booted cluster nodes
		\\ \hline
%
  \end{tabular}
  \caption{The set of available OSCAR API scripts listed in order of execution.}
  \label{tab:pkg-scripts}
  \end{center}  
\end{table}




% TJN: finish this section
\subsection{testing}
\label{sect:pkg-testing}

Basic tests are run for each package.  The two scripts that are available
for this testing are: \file{test\_root} and \file{test\_user}.  When tests
are run for the cluster, all \file{test\_root} scripts are sourced which
perform any root level package tests.  

\begin{verse}
   {\bfseries Notice: } There are obvious security issues with this
   but currently all operations in the cluster installation are being
   performed by {\tt root} so care is expected at all phases.  The user
   tests are run as an actual user so those tests are somewhat less 
   dangerous and therefore most packages are using \file{test\_user}.
\end{verse}


The tests typically have PBS available and most of teh \file{test\_user}
scripts simply setup and run a simple PBS jobs for the installed software,
e.g., PVM, MPI's.  The testing framework is currently pretty simple with
display functions provided which show boolean results of ``PASSED'' or
``FAILED''.




 
\subsection{doc}

This directory contains supplemental documentation for the package.  There
are a few pre-defined \LaTeX\ files that may be incorporated into the
overall OSCAR documentation if the package's classification is either
\emph{core} or \emph{selected}~\footnote{That is to say the package is
included in the main distribution tarball -- not obtained via OPD.}.  These
files are: \file{install.tex}, \file{user.tex} and \file{license.tex}.  The
first is added to the overall \file{install.pdf} and contains information
related to the installation of the particular software package.  The latter
two files are incorporated into the \file{user.pdf}.   The user information
can be complete or simply pointers to obtaining more thorough documentation
for the particular package.  The license for all packages are listed in
this document based on the contents of this \file{license.tex} file.


