%---------------------------------------------------------------------------
% $Id: env-switcher.tex,v 1.1 2003/09/26 06:52:32 naughtont Exp $
% (FROM: env-switcher.tex,v 1.4 2003/05/18 02:48:28 tjn)
%
%  Ottawa Linux Symposium 2003 (OLS'03) Paper
%  July 23-26, 2003,  Ottawa Canada
%  http://www.linuxsymposium.org/2003/
%
% Information on the env-switcher package.
%
%---------------------------------------------------------------------------
%
% TJN: I have removed the cites{} since we don't do a Reference section.
%
% TJN: (TODO) Add comments on how to install a switcher module & create
%      a switcher RPM.
%

\section{Env-Switcher}
\label{sect:switcher}

Managing the shell environment -- both at the system-wide level as
well as on a per-user basis -- has historically been a daunting task.
For cluster-wide applications, system administrations typically need
to provide custom, shell-defendant startup scripts that, create and/or
augment \envvar{PATH}, \envvar{LD\_\-LIBRARY\_\-PATH}, and
\envvar{MANPAGE} environment variables.
%
Alternatively, users could hand-edit their ``dot'' files (e.g.,
\file{\$HOME/.profile}, \file{\$HOME/.bashrc}, and/or
\file{\$HOME/.cshrc}) to create/augment the environment as necessary.
%
Both approaches, while functional and workable, typically lead to
human error -- sometimes with disastrous results, such as users being
unable to login due to errors in their ``dot'' files.

Instead of these models, OSCAR provides the \package{env-switcher}
OSCAR package. \package{env-switcher} forms the basis for simplified
environment management in OSCAR clusters by providing a thin layer on
top of the Environment Modules
%package~\cite{furlani91:_modul,furlani96:_abstr_yours_with_modul}.
package (Furlani \& Osel).
Environment Modules provide an efficient, shell-agnostic method of
manipulating the environment.  Basic primitives are provided for
actions such as: add a directory to a \envvar{PATH}-like environment
variables, displaying basic information about a package, and setting
arbitrary environment variables.  For example, a module file for
setting up a given application may include directives such as:
%
\begin{verbatim}
  setenv FOO_OUTPUT $HOME/results
  append-path PATH /opt/foo-1.2.3/bin
  append-path MANPATH /opt/foo-1.2.3/man
\end{verbatim}
% Stupid emacs mode: $
%
The \package{env-switcher} package installs and configures the base
Modules package and creates two types of modules: those that are
unconditionally loaded, and those that are subject to system- and
user-level defaults.

Many OSCAR packages use the unconditional modules to append the
\envvar{PATH}, set arbitrary environment variables, etc.  Hence, all
users automatically have these settings applied to their environment
(regardless of their shell) and guarantee to have them executed even
when executing on remote nodes via \cmd{rsh}/\cmd{ssh}.

Other modules are optional, or a provide one-of-many selection
methodology between multiple equivalent packages.  This allows the
system to provide a default set of applications, that optionally can
be overridden by the user ({\em without} hand-editing ``dot'' files).
%
A common example in HPC clusters is having multiple Message Passing
%Interface (MPI)~\cite{geist96:_mpi2_lyon,mpi_forum93:_mpi}
Interface (MPI)
implementations installed.  OSCAR installs both the
%LAM/MPI~\cite{burns94:_lam} and MPICH~\cite{Gropp:1996:HPI}
LAM/MPI and MPICH
implementations of MPI.  Some users prefer one over the other, or have
requirements only met by one of them.  Other users wish to use both,
switching between them frequently (perhaps for performance
comparisons).

The \package{env-switcher} package provides trivial syntax for a user
to select which MPI package to use.  The \cmd{switcher} command is
used to select which modules are loaded at shell initialization time.
For example, the following command shows a user selecting to use
LAM/MPI:
%
\begin{verbatim}
  shell$ switcher mpi = lam-6.5.9
\end{verbatim}
% Stupid emacs mode: $

\subsection{Example Switcher Module file}

This example was taken from the PVM module file, which gets loaded
unconditionally for all shells.

\begin{footnotesize}
\begin{verbatim}
# PVM modulefile for OSCAR clusters (based on LAM modulefile)

proc ModulesHelp { } {
  puts stderr "\tThis module adds PVM to the PATHand MANPATH."
  puts stderr "\tAdditionally the PVM_ROOT and PVM_ARCH are set."
}

module-whatis   "Sets up the PVM environment for an OSCAR cluster."

setenv PVM_RSH  ssh
setenv PVM_ROOT /opt/pvm
setenv PVM_ARCH LINUX

append-path MANPATH /opt/pvm/man

append-path PATH /opt/pvm/lib
append-path PATH /opt/pvm/lib/LINUX
append-path PATH /opt/pvm/bin/LINUX
\end{verbatim}
\end{footnotesize}
