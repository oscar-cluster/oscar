% -*- latex -*-
%
% $Id: sourceforge.tex,v 1.1 2002/02/09 19:00:36 jsquyres Exp $
%
% $COPYRIGHT$
%

\section{SourceForce Usage}
\label{sec:sf}

SourceForce provides a number of development tools for the OSCAR
project.  As with all computer system administration, having some
``common sense'' policies ease the administrative burden and ensure
the orderly progress of development.

%%%%%%%%%%%%%%%%%%%%%%%%%%%%%%%%%%%%%%%%%%%%%%%%%%%%%%%%%%%%%%%%%%%%%%%%%%

\subsection{Roles}

The OSCAR working group uses several roles at SourceForge.  Sweat
equity holders can fall into one or more roles.  Assuming one role or
another is mainly an administrative distinction, and meant to ensure
that common system-administrative tasks are carried out in a
coordinated fashion.  {\em Roles do not indicate a member's relative
  importance or status in the OSCAR working group}.  The OSCAR working
group chair is the only member who automatically assumes a role -- all
others are considered equal in terms of SourceForge roles.  Decisions
on who is in each role should be achieved by group consensus whenever
possible.

The following roles are defined:

\begin{itemize}
\item Administrator: The OSCAR chair is, by default, the SourceForge
  administrator for the OSCAR project.  The SourceForge administrator
  can add and delete members to any of the other roles.  It is
  strongly recomended to have at least one other administrator
  (designated by the chair) as a backup.

\item CVS committer: Any sweat equity holder who actively contributes
  to the CVS tree.
  
\item Bug tracker administrators: This should be the same group of
  members as CVS committers; because of the way that SourceForge
  permissions work, only bug track administrators can change fields of
  a bug.  All others can only add notes to a bug.  Since a bug can be
  assigned to any CVS committer, that bug owner should be able to
  change the fields on that bug.  Hence, all CVS committers need to be
  bug tracker administrators.
  
\item Release technicians: This should be a small group of sweat
  equity holders (at least two).  All releases must be approved by the
  Release Manager.  See Section~\ref{sec:release} for procedures on
  releasing on SourceForge -- it's not trivial.

\item Mailing list administrators: This should be a small group of
  sweat equity holders (at least two).  The mailing list
  administrators should chosen so that at least one is generally
  available to respond to mailing list administrative functions in a
  timely manner.
  
\item Web page updaters: This group of sweat equity holders that are
  directly responsible for keeping the OSCAR web pages up-to-date with
  accurate information.  Since the OSCAR web pages are in the CVS
  repository, any CVS committer has access to modify them, but if
  possible, only this group should have permissions to actually modify
  the ``live'' copy of the pages on the SourceForge site.
\end{itemize}

The members of each role should be listed on the OSCAR web site to
make it easy for members to find out who to contact for any given
SourceForge administrative actions.

% LocalWords:  Exp
