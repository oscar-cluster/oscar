\section{Provided Tools}

In this section, the avalaible tools for package creators and maitainers
are described.  As this is the first draft of this section, 
please report all documentation bug via the regular sourceforge bug interface.

\subsection{PERL syntax and location of the tools}

The OSCAR library is in \file{lib/OSCAR} and contain a lot of PERL packages.
In order to use a package in your script, you have to use the following
syntax~:\verb+ use OSCAR::PackageName+. The corresponding file will be
\file{lib/OSCAR/PackageName.pm}.

If you find a discrepancy with this documentation and the expected behavior,
please have a look at the package file~: they are generally commented.

\subsubsection{PERL invocation and mode}

As OSCAR is a big project, you are strongly encourage to use the warning
option, i.e.  \verb+/usr/bin/perl -w+ as well as the strict mode 
\verb+use strict;+. Please check the existing OSCAR perl ressources so that you don't
reinvent the wheel. If you think that some functionnality should be provided
to all packages, please make a separate module in the \file{lib/OSCAR/}
directory.

\subsection{OSCAR logging~: \file{OSCAR::Logger}}

This is an important package ! Please use it so that all logged information
has the same look. It helps people to find their way inside the problems and
their solutions !

\subsubsection{oscar\_log\_section(section\_name)}

It is used to produce a "section" title. Use it with your module name at the
begining of your module script(s) so that it is clear where the problem
happened !

\subsubsection{oscar\_log\_subsection(log\_message)}

Use this so send a standard log message.

\subsubsection{Complete Example}

\begin{verbatim}
use OSCAR::Logger;
my $oscar_version="12.2.3";
oscar_log_section("Running OSCAR install_cluster script");
oscar_log_subsection("OSCAR version: $oscar_version");
\end{verbatim}

\subsection{Distribution type and version~:\file{OSCAR::Distro}}

The \file{OSCAR::Distro} provide the functions to obtain the distro name and
version. For the moment, the only distro detected are "RedHat" and "Mandrake".

\begin{verbatim}
use OSCAR::Distro;
my ($distro_name, $distro_version) = which_distro_server();
\end{verbatim}

\subsection{Package actions~: \file{OSCAR::Package}}

The \file{OSCAR::Package} provide useful function to deal with all package
related stuff. It is very well documented and you should have a serious look
at it if you want to make an OSCAR package.

Here is are some particularly usefull functions~:

\subsubsection{install\_rpms(@rpm\_hash)}

This routine install the best rpms on the server, only if
they don't already exist at a high enough version.

\subsubsection{list\_pkg(core or noncore or all)}

Return a list (array) of packages that are avalaible for install.

\subsubsection{isPackageSelectedForInstallation(package\_name)}

Return 1 if selected, 0 otherwise. Usefull if your package actions depend of
another package !

\subsubsection{Other exported functions}

\begin{verbatim}
@EXPORT = qw(list_pkg run_pkg_script run_pkg_user_test
             run_pkg_script_chroot rpmlist distro_rpmlist install_rpms
             pkg_config_xml list_install_pkg getSelectionHash
             isPackageSelectedForInstallation getConfigurationValues);
\end{verbatim}

\subsection{Network Interface Information on the OSCAR server~: \file{OSCAR::Network}}

The \file{OSCAR::Network} provides the ip address, broadcast and netmask
of an interface. There is only one function~:\file{interface2ip}.

\subsubsection{Complete Example}

\begin{verbatim}
use OSCAR::Network;
my ($ip, $broadcast, $net) = interface2ip(eth0);
\end{verbatim}

\subsection{Environment variables}

Two useful environment variable are used by OSCAR. 

Note~: those variables are only avalaible when installing the cluster (i.e. running
\file{install\_cluster}).

\subsubsection{Interface of the cluster (private interface)}

If your package configure some network parameter, you will certainly need this
parameter~:
\begin{itemize}
\item \verb+OSCAR_HEAD_INTERNAL_INTERFACE+ : The interface where the cluster
is installed (usually ethx).
\end{itemize}

\subsubsection{OSCAR tree root}

This is useful in order to specify absolute path from anywhere !
\begin{itemize}
\item \verb+OSCAR_HOME+ : The localisation of OSCAR tree (i.e. the location of
the file \file{install\_cluster}).
\end{itemize}

\subsubsection{Usage in PERL}

All the environment variable are in the hash \verb+ENV+. You can access those
variables this way~:
\begin{itemize}
\item \verb+$ENV{OSCAR_HOME}+
\item \verb+$ENV{OSCAR_HEAD_INTERNAL_INTERFACE}+
\end{itemize}


