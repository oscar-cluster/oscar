% -*- latex -*-
%
% $Id: cvs.tex,v 1.5 2002/08/07 02:43:13 jsquyres Exp $
%
% $COPYRIGHT$
%

\section{CVS Usage}
\label{sec:cvs}

Portions of the OSCAR package are maintained in CVS.  Checking out the
``{\tt oscar}'' module from the OSCAR CVS repository will create a
copy of a portion of the OSCAR development tree.  The remainder of the
development tree is obtained by invoking a top-level ``{\tt make}''
command.  This will trigger the automatic downloads of any missing
binary packages from various web sites, which will complete the
development tree.

%%%%%%%%%%%%%%%%%%%%%%%%%%%%%%%%%%%%%%%%%%%%%%%%%%%%%%%%%%%%%%%%%%%%%%%%%%

\subsection{The Golden Rules}

The following Golden Rules will be followed at all times:

\begin{enumerate}
\item {\bf Only check in source code.}  This means not checking in any
  automatically-generated files, such as DVI or postscript files
  generated from \LaTeX, or {\tt configure} scripts generated by {\tt
    autoconf}, etc.
  
\item {\bf Never knowingly commit buggy or untested code.}  While bugs
  are inevitable and unavoidable, please take care to test code before
  committing it back to CVS.  There is nothing more frustrating than
  checking out someone else's code that is blatantly broken.
  
\item {\bf Give meaningful log messages.}  When committing files,
  write meaningful log messages so that not only do other developers
  know what you did, a permanent and easily human-readable history of
  what has happened to each file is maintained.  If the commit fixes a
  specific bug, indicate the bug ID in the log message.
  
\item {\bf Group related patches into a single commit.}  If a specific
  feature or bug fix spans multiple files, isolate the patches related
  to just that feature/bug fix and commit them all at once.  This
  provides modular patches, and helps the developer ensure that {\em
    all} related code is committed.

\item {\bf Do not commit binary files.}  While CVS purportedly has the
  capability to handle binary files, there have been reports of
  inconsistent behavior.  As such, all of OSCAR's binary files will be
  stored outside of the CVS tree.
\end{enumerate}

%%%%%%%%%%%%%%%%%%%%%%%%%%%%%%%%%%%%%%%%%%%%%%%%%%%%%%%%%%%%%%%%%%%%%%%%%%

\subsection{Text Files}

Per Section~\ref{sec:file-conventions}, all text files will include
the CVS token ``{\tt \$Id\$}'' within the first few lines.  This will
automatically maintain the last modification time, who committed last,
etc.  

The CVS token ``{\tt \$Log\$}'' will not be used for text files; this
is redundant with the ``{\tt cvs log}'' command.

%%%%%%%%%%%%%%%%%%%%%%%%%%%%%%%%%%%%%%%%%%%%%%%%%%%%%%%%%%%%%%%%%%%%%%%%%%

\subsection{Development Tree Update Procedure}

Updating an OSCAR development tree is simple: run \cmd{cvs update}.
This will update the entire development tree to match the current
state of CVS.

% LocalWords:  Exp oscar DVI autoconf og RPMs gzipped tarballs kb logfile
