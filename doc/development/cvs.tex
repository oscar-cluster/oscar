% -*- latex -*-
%
% $Id: cvs.tex,v 1.1 2001/08/14 18:17:53 jsquyres Exp $
%
% $COPYRIGHT$
%

\section{CVS Usage}
\label{sec:cvs}

The OSCAR package is maintained entirely in CVS; checking out the
``{\tt oscar}'' module from the OSCAR CVS repository will create a
copy of the OSCAR development tree.

%%%%%%%%%%%%%%%%%%%%%%%%%%%%%%%%%%%%%%%%%%%%%%%%%%%%%%%%%%%%%%%%%%%%%%%%%%

\subsection{The Golden Rules}

The following Golden Rules will be followed at all times:

\begin{enumerate}
\item {\bf Only check in source code.}  This means not checking in any
  automatically-generated files, such as DVI or postscript files
  generated from \LaTeX, or {\tt configure} scripts generated by {\tt
    autoconf}, etc.
  
\item {\bf Never knowingly commit buggy or untested code.}  While bugs
  are inevitable and unavoidable, please take care to test code before
  committing it back to CVS.  There is nothing more frustrating than
  checking out someone else's code that is blatantly broken.
  
\item {\bf Give meaningful log messages.}  When committing files,
  write meaningful log messages so that not only do other developers
  know what you did, a permanent and easily human-readable history of
  what has happened to each file is maintained.  If the commit fixes a
  specific bug, indicate the bug ID in the log message.
  
\item {\bf Group related patches into a single commit.}  If a specific
  feature or bug fix spans multiple files, isolate the patches related
  to just that feature/bug fix and commit them all at once.  This
  provides modular patches, and helps the developer ensure that {\em
    all} related code is committed.
\end{enumerate}

%%%%%%%%%%%%%%%%%%%%%%%%%%%%%%%%%%%%%%%%%%%%%%%%%%%%%%%%%%%%%%%%%%%%%%%%%%

\subsection{Text Files}

Per Section~\ref{sec:file-conventions}, all text files will include
the CVS token ``{\tt \$Id\$}'' within the first few lines.  This will
automatically maintain the last modification time, who committed last,
etc.  

The CVS token ``{\tt \$Log\$}'' will not be used for text files; this
is redundant with the ``{\tt cvs log}'' command.

%%%%%%%%%%%%%%%%%%%%%%%%%%%%%%%%%%%%%%%%%%%%%%%%%%%%%%%%%%%%%%%%%%%%%%%%%%

\subsection{Binary Files}

Since the OSCAR package includes binary files such as RPMs and gzipped
tarballs, CVS must be used to maintain these as well.  However, CVS
has some well-known deficiencies in maintaining binary
files.\footnote{Indeed, there actually is no good way to maintain
version control of binary files short of saving a copy of each old
version of the file in its entirety.}  As such, text files should be
preferred whenever possible.  For example, none of the core packages
developed specifically for OSCAR should be stored in CVS in binary
form.

When adding binary files to CVS, the ``{\tt -kb}'' command line switch
{\em must} be used to prevent CVS from attempting to treat it as a
text file.

Binary packages will likely have version numbers as part of their
filename (see Section~\ref{sec:file-conventions}).  As such, each new
version that is checked in will likely have a new filename.  When this
occurs, the old file must be removed from CVS.  

This somewhat breaks one of the fundamental ideas of CVS -- that the
history associated with a file is kept with that file.  Since each
version of the binary will have a new filename, the log history of
that binary package will be split across multiple files.  

\begin{discuss}
  For this reason, it may be desirable for the small number of third
  party packages that are maintained by the OSCAR equity holders
  (namely, LAM/MPI, PVM, MPICH, PBS) to maintain an associated ``{\tt
    ChangeLog}'' file in the package directory that contains the CVS
  ``{\tt \$Log\$}'' token.  This will create a comprehensive version
  history, rather than having to query the log comments from multiple
  files.
\end{discuss}

% LocalWords:  Exp oscar DVI autoconf og RPMs gzipped tarballs kb logfile
