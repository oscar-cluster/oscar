% -*- latex -*-
%
% $Id: intro.tex,v 1.1 2001/08/14 18:17:53 jsquyres Exp $
%
% $COPYRIGHT$
%

\section{Introduction}

The purpose of this document is to familiarize developers of the OSCAR
packages with both the conventions used in the OSCAR package as well
as the requirements for core and third party packages that are
included in OSCAR.  For the purposes of this document, the term
``OSCAR package'' refers to the directory structure and all files
contained within an OSCAR distribution package.  This package may be a
tarball, ISO, RPM, Debian package, etc.  Initial versions will almost
certainly be gzipped tarballs, but alternate packages may also be
released.  The OSCAR package includes binary distributions of core and
third party packages.

The OSCAR package is essentially divided into three components:

\begin{enumerate}
\item The OSCAR core: a small number of packages that provide the core
  functionality of OSCAR.
  
\item Third party packages: self-contained packages that conform to
  the OSCAR installation API, and can be ``dropped in'' to an OSCAR
  package such that they can be installed / maintained / uninstalled
  by the OSCAR framework.

\item ``Glue'' utility code: some helper scripts and additional
  access functionality that is used by multiple parts of the OSCAR
  core and/or third party packages.
\end{enumerate}

Note that the conventions and guidelines described in this document
are intended for OSCAR-specific files.  That is, these conventions
apply to:

\begin{itemize}
\item Directories in the OSCAR package.
\item Third party binary package files contained in the OSCAR package.
\item All other files (source code, text, scripts, etc.) in the OSCAR
  package.
\end{itemize}

\noindent These conventions do {\em not} apply to:

\begin{itemize}
\item Files contained {\em within} binary package files in the OSCAR
  package.
\item Files created by the installation of any package (core or third
  party).  We recommend that files installed by OSCAR core packages
  adhere to these conventions, but do not require it.
\end{itemize}

For example, the filename of a third party binary package is covered
by these conventions (e.g., the OSCAR version of the LAM/MPI RPM), but
individual files contained in the third party binary package do not
fall under the jurisdiction of these conventions (e.g., the files
installed by or used in the OSCAR version of the LAM/MPI RPM), nor do
any of the files that it installs.

% LocalWords:  Exp gzipped tarballs
