% -*- latex -*-
%
% $Id: directories.tex,v 1.3 2002/01/23 05:13:41 jsquyres Exp $
%
% $COPYRIGHT$
%

\section{Directory Structure}
\label{sec:directory-structure}

The following describes the directory structure that is used in the
OSCAR CVS tree (for version 2.0 and beyond) as well as OSCAR
distribution packages.  Note that this tree has little (if any)
relation to the installation trees that are installed by the various
core and third party packages that OSCAR installs.

%%%%%%%%%%%%%%%%%%%%%%%%%%%%%%%%%%%%%%%%%%%%%%%%%%%%%%%%%%%%%%%%%%%%%%%%%%

\subsection{Conventions}

The directory structure of the OSCAR development tree (stored in CVS,
see Section~\ref{sec:cvs}) is intended to be almost identical to the
directory tree of the OSCAR distribution package.  This allows the
OSCAR framework to function identically in both cases, and makes less
for OSCAR developers to debug.

An OSCAR distribution package is generated from the development tree
by applying a minimalistic set of transformations and then bundling up
the resulting tree into a binary package.  These transformations may
include:

\begin{itemize}
\item Adding copyright notices
\item Substituting in the current OSCAR version number
\item Substituting in the release date
\item Removing maintainer mechanisms to build distribution packages
\end{itemize}

As such, OSCAR developers may develop and test code directly in a
development tree without requiring the interim steps of creating a
distribution package, expanding that package in a temporary directory,
and then installing from that temporary directory.

%%%%%%%%%%%%%%%%%%%%%%%%%%%%%%%%%%%%%%%%%%%%%%%%%%%%%%%%%%%%%%%%%%%%%%%%%%

\subsection{Tree Layout}

The top-level directory name is {\tt oscar}; it is the one and only
top-level directory for OSCAR.  All OSCAR directories and files will
be included below this directory.  This directory is also special in
that in the CVS tree, its name is ``{\tt oscar}'', but in a
distribution package, its name will include the OSCAR version number
(this is automatically changed when the distribution package is made).
For example, ``{\tt oscar-2.0}''.\footnote{For brevity, the rest of
  this document refers to this directory as ``{\tt oscar}'' with the
  understanding that in a distribution package the directory name will
  actually be ``{\tt oscar-VERSION}''.}

This top-level directory will include at least the following files:

\begin{itemize}
\item {\tt COPYING}: The GNU public license file, version 2.0.

\item {\tt Makefile.am}: A top-level file used for building an OSCAR
  distribution package.

\item {\tt configure.in}: Another top-level file used for building an
  OSCAR distribution package.  Note that this file will probably
  eventually change names to {\tt configure.ac} as newer versions of
  the GNU {\tt autoconf} package become more commonly installed.
  
  Also note that {\tt configure} (and any other
  automatically-generated files) is {\em not} maintained in CVS.  It
  is the developer's responsibility to invoke the right commands to
  generate these files.  {\tt configure} (and other
  automatically-generated files) {\em will} be included in OSCAR
  distribution packages.

\item {\tt README}: A text file describing generally what OSCAR is,
  listing the main OSCAR URL, and brief directions on where to find
  the OSCAR installation guide.
\end{itemize}

\noindent The subdirectories of {\tt oscar} are:

\begin{itemize}
  
\item {\tt core-packages}\footnote{Note that it would be a Bad Idea to
    name this directory ``{\tt core}'', because at least some linux
    distributions have cron jobs that actively seek out and {\tt rm
      -rf} files named {\tt core}.}: This directory contains
  subdirectories for each of the core packages in OSCAR.  Just as with
  the {\tt packages} directory, there may be some top-level scripts in
  this directory to assist in installing/uninstalling the core
  packages.

  Each core package will be entirely self-contained in its own
  directory, the name of which will reflect the name of the core
  package.  The subdirectory name will {\em not} include a version
  number.

  The core packages of OSCAR are:

  \begin{itemize}
  \item {\tt c3}: The C3 package.
  \item {\tt odr}: The OSCAR data repository.
  \item {\tt sis}: The System Installation Suite.
  \item {\tt wizard}: The OSCAR wizard.
  \end{itemize}
  
  The contents of these directories are discussed in
  Section~\ref{sec:core-directories}.
  
\item {\tt dist}: This directory contains scripts and additional
  maintainer-level files that are used for making an OSCAR
  distribution.  This may include a top-level OSCAR RPM spec file,
  configuration or other helper shell scripts, etc.  This directory
  may or may not be included in a distribution OSCAR package.
  
\item {\tt doc}: This directory contains various forms of
  documentation.  Top-level how-to's, text files, etc., are eligible
  to be included in this directory.  Any document that consists of
  multiple parts (e.g., documents that are made from \LaTeX\ source)
  will be in their own subdirectory.  The following three documents,
  for example, have their own subdirectories, and can be built as
  postscript and/or PDF files:

  \begin{itemize}
  \item {\tt development}: This document; outlining the standards and
    guidelines for OSCAR developers.

  \item {\tt information}: A document containing general information
    about OSCAR.
    
  \item {\tt installation}: The installation guide for the OSCAR
    software package.
  \end{itemize}
  
  Distribution packages will include pre-made postscript and PDF
  versions of all documents that can be made into postscript/PDF.

\item {\tt packages}: This directory contains subdirectories for each
  of the third party packages that the OSCAR framework may install.
  There may be some top-level scripts in this directory to assist in
  installing/uninstalling third party packages.

  The directory naming scheme is the same as with the {\tt
    core-packages} directory.  Some examples of names include:

  \begin{itemize}
  \item {\tt lam}
  \item {\tt mpich}
  \item {\tt openssh}
  \item {\tt pbs}
  \item {\tt pvm}
  \end{itemize}
  
  The contents of these directories are discussed in
  Section~\ref{sec:third-directories}.

\end{itemize}

% LocalWords:  Exp oscar PL dist doc to's ource PDF lib lam mpi mpich pbs linux
% LocalWords:  pvm odr minimalistic rm rf openssh
