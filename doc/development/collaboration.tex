% -*- latex -*-
%
% $Id: collaboration.tex,v 1.2 2002/09/09 14:10:52 jsquyres Exp $
%
% $COPYRIGHT$
%

\section{Distributed Development / Collaboration}
\label{sec:collaboration}

The OSCAR working group is comprised of developers belonging to
different organizations from around the world.  As such, most
developers are physically separated from each other, which must be
managed properly in order to avoid development chaos and degrdation
into flame wars between developers.
% 
OSCAR embodies the nature and spirit of open source development that
is both rewarding and challenging; the OSCAR group works well together
because its members all share a desire to create high-quality cluster
software, yet every member has their own specific goals and
requirements that they wish to see in OSCAR.

Generally, this model works well.  However, after working in the OSCAR
core development group for quite a while, we have learned a few things
about distributed open source development, and have adopted a loose
set of standard operating procedures which, while they are continually
updated to reflect ``best development practices'', are generally
stable enough that new OSCAR developers should be aware of them.

%%%%%%%%%%%%%%%%%%%%%%%%%%%%%%%%%%%%%%%%%%%%%%%%%%%%%%%%%%%%%%%%%%%%%%%%%%

\subsection{Rules of Thumb}

The following rules of thumb describe how the OSCAR group has been
successful in terms of distributed, open source code development:

\begin{itemize}
\item OSCAR is a consensus-driven group.  There are formal mechanisms
  for resolving conflicts, but the emphasis to get group agreement
  before emarking on major new directions.  Having group-level
  consensus -- especially given OSCAR's distributed development -- is
  a major reason that OSCAR development ``works''.
  
\item OSCAR developers primarily work ``in their spare time.''  As
  such, most suggestions for new features and/or directions are
  usually most appreciated when they are accompanied by concrete
  examples (such as code, or some form of proposal describing
  specifics, such as requirements and implementation plans).
  
\item Read and updated this document.  A lot of our development
  guidelines are contained within this document.  As a group, we
  generally try to abide by everything here -- it's a lot easier when
  all developers adhere to the same guidelines.
  
\item Use collaboration tools are much as possible.  Communication is
  vital in distributed development.  Don't blindly make code changes
  without communicating with other OSCAR developers.  There may be
  non-obvious reasons why code is written in a specific way.  When in
  doubt -- ask.
  
\item It's bad form to modify a specific piece of code that
  ``belongs'' to another developer without prior notice and/or
  coordination with the ``owner'' of the code.  This is usually most
  important with respect to individual packages, since those are
  generally developed by only one or two developers (vs.\ the general
  OSCAR framework, which is generally developed by the whole group).
\end{itemize}

%%%%%%%%%%%%%%%%%%%%%%%%%%%%%%%%%%%%%%%%%%%%%%%%%%%%%%%%%%%%%%%%%%%%%%%%%%

\subsection{Collaboration Tools}

Because of the distributed nature of OSCAR development, the OSCAR
developers have made heavy use of various collaboration tools.  These
include the following:

\begin{itemize}
\item E-mail.  E-mail is probably the most heavily used tool for
  collaboration and coordination.  OSCAR has several lists hosted at
  SourceForge, each of which has a specific purpose.  The key lists
  for developers to be subscribed to are:

  \begin{itemize}
  \item \listname{oscar-devel}: The developer's list.  The majority of
    developer-related traffic goes across this list.
    
  \item \listname{oscar-users}: The general user's list.  This list is
    where users submit their problems and questions.  Developers
    should be on this list to answer questions about their portions of
    OSCAR, as well as to stay informed about the current OSCAR user
    base.
    
  \item \listname{oscar-checkins}: This list is where all CVS commit
    notices are sent.  Developers should be in the habit of at least
    scanning each CVS commit for sanity checking purposes.
  \end{itemize}
  
  Much mail is also sent outside of the lists for addressing specific
  issues, coordination between developers, etc.  However, developers
  are encouraged to use the lists whenever possible so that messages
  and issues are archived in the SourceForge web repository for each
  list.
  
\item Weekly teleconferences.  This is currently only open to the
  ``core'' group of developers.  Ths weekly teleconference is used
  primarily to discuss administravia, technical issues, release
  schedules, planning for versions, features, and bug fixes, etc.  It
  is essentially a weekly synchronization between the core group of
  developers.

\item SourceForge bug and feature trackers.  All bugs, great and
  small, should be logged to the SF bug tracker, and all ideas for new
  features should be logged to the SF feature tracker.  This is not
  only to ensure that the OSCAR group doesn't collectively forget
  about a given issue, but also a coordination tool so that specific
  bugs are assigned to specific developers.
  
  Developers should check their personal SF pages frequently to ensure
  that they are aware of bugs that have been assigned to them, ensure
  that old bugs have been closed, etc.
  
\item Internet Relay Chat (IRC).  OSCAR has a dedicated IRC channel on
  \hostname{irc.openprojects.net} named
  ``\channelname{\#oscar-cluster}''.  This channel is frequently used
  ``on demand'', or when multiple developers want to meet to discuss a
  given topic.  It has also been used to real-time chat with OSCAR
  users to step them through problems, etc.  For example, if a user
  submits a problem on the \listname{oscar-users} list that cannot be
  resolved easily through e-mail, a developer may send a off-list
  message to the user asking them to meet on
  \channelname{\#oscar-cluster} for some interactive troubleshooting.
  
\item Instant messaging (IM).  Most OSCAR developers have IM accounts
  of various flavors (most use Yahoo! IM, but some have other kinds of
  accounts, such as AOL IM, MSN, etc.).  IM is used quite frequently
  for spur-of-the-moment collaboration and questions between core
  developers.  
  
  IM account names are generally treated as ``semi-private.''  As
  such, they are not published in this document, nor are they
  generally given out to random OSCAR users or ``fringe'' OSCAR
  developers.  That is, IM is generally used between core developers
  (i.e., developers who will be involved in OSCAR development for long
  periods of time); new core developers should feel free to ask for
  the IM account names of other OSCAR core developers (perhaps on the
  weekly teleconference).

  IRC is generally used for all other real-time messaging.
\end{itemize}
