% -*- latex -*-
%
% $Id: included-packages.tex,v 1.2 2001/08/30 22:30:51 jsquyres Exp $
%
% $COPYRIGHT$
%

\section{Included Packages}

The following sections include a brief description of the included
packages in a full OSCAR distribution.

\subsection{LAM/MPI}

Blah.

\subsection{MPICH}

Blah.

\subsection{PVM}

Blah.

\subsection{OpenSSH}

Blah.

\subsection{OSCAR Password Installer and User Management (OPIUM)}

\begin{discuss}
  Note that this is a conceptual description -- details and
  implementation still remain to be worked out.
\end{discuss}

This package is a method for maintaining a consistent view of users,
passwords, and groups across an OSCAR cluster.  The OPIUM model is
that all user, password, and group management is performed on a
specified node (most likely the cluster head); changes are
pushed/pulled out to the rest of the nodes in the cluster.  OPIUM is
{\em not} meant to be used in conjunction with Kerberos, NIS, NIS+, or
any other user/group/password management system.

OPIUM replaces commands such as \cmd{passwd}, \cmd{adduser}, and
\cmd{deluser} with its own versions.  If users run these commands on
nodes other than the specified OPIUM server, they will be told to run
them on the OPIUM server.  The commands perform their specified
functionality only on the OPIUM server.

This is not a required package.  However, if it is not installed, some
other user/group/password management system is highly recommended
(such as Kerberos, NIS, NIS+, etc.), or OSCAR administrators will be
responsible for keeping users, groups, and passwords consistent
between all OSCAR nodes.

% LocalWords:  geiselha Blah Kerberos NIS passwd adduser deluser
