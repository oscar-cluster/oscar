% -*- latex -*-
%
% $Id: requirements.tex,v 1.2 2001/08/14 23:24:03 jsquyres Exp $
%
% $COPYRIGHT$
%

\section{Requirements Specification}

After version 1.0 was released, it became more apparent to the project
developers that in order to eliminate the problems associated with the
first release, the second generation OSCAR needed to be redesigned
from scratch. Accordingly, a meeting was held in April 2001 with the
agenda focusing on identifying the major problems in version 1.0 and
coming up with solutions to those problems for version 2.0, with the
final goal being a complete design architecture for version 2.0. The
following sections detail the problems to be resolved and the
solutions generated for the second generation OSCAR project.

\subsection{Architecture and Distribution Independence}

One of the most highly visible problems in the first version was a
lack of support for varying cluster node architectures and Linux
software distributions.

Part of the problem lay in the fact that this was the first attempt to
do a tightly integrated clustering solution using the "best-known
practices" software. In order to produce a functioning solution, all
development was done for only one architecture (Intel x86) and one
Linux distribution (RedHat 6.2). The result was that attempts to use
OSCAR 1.0 with other architectures were futile, and using other Linux
distributions required much work by users to port the OSCAR code to
the new distribution, unless the users were lucky enough to try a
distribution very similar to RedHat 6.2.

Another contributor to the architecture/distribution problem was the
underlying software upon which OSCAR heavily relied, that being the
Linux Utility for cluster Installs (LUI). When LUI was selected as the
cluster installation tool for OSCAR 1.0, it was very much in infancy,
and as such was also heavily architecture and distribution
specific. As time has progressed, it was decided that LUI must become
architecture and distribution independent, a task that has also
generated a complete redesign and rewrite.

For the second generation OSCAR, we will once again be relying heavily
on the LUI technology for installing cluster nodes.  However, the LUI
project has joined forces with SystemImager to become the new System
Installation Suite (SIS), consisting of SystemInstaller,
SystemConfigurator, and SystemImager. As SIS plans on supporting
multiple architectures and distributions, using SIS in OSCAR 2.0 will
greatly help in furthering the architecture/distribution independence
problem. At this point however, SIS only supports the x86 and PowerPC
architectures, due to a lack of availability of other hardware for the
project developers. During the April meeting, it was decided that
OSCAR 2.0 will be made to initially work with the following Linux
distributions: RedHat, MSC Linux, TurboLinux, Mandrake, and Debian.

\subsection{OS Installation vs.\ Cluster Software Installation}

In  feedback  from  users  of  the first  release,  much  trouble  was
encountered just getting the operating system installed on the cluster
nodes. Most of  the confusion stemmed from a  lack of familiarity with
LUI.  Many of  the  users (and  even  some OSCAR  core group  members)
commented that  it would be  nice if there  was a way to  separate the
installation  of  the  operating  system  on cluster  nodes  from  the
installation and configuration  of the cluster software. A  few of the
benefits for doing so include:

\begin{enumerate}
\item The ability to support multiple methods for installing the OS on
  nodes (e.g., LUI, RedHat KickStart, from the distribution CD), as
  users may be more comfortable/familiar with a method not provided by
  OSCAR.
  
\item Easier to support site differences such as non-private clusters
  and special purpose nodes.
  
\item The ability to turn workstations with an operating system
  already installed into a cluster without reinstalling the OS.
\end{enumerate}

Therefore, in OSCAR 2.0 it was decided that OS installation would be a
user selectable option in the OSCAR Wizard.

The one main disadvantage to the separation is that the OS
installation is now just an option for OSCAR users, which means that
OSCAR cannot rely on the OS installer to install any of the cluster
software. However, given the new cluster software management scheme
discussed in Section~\ref{sec:reqs-software-management} below, this
should not be a problem. Another minor issue with supporting multiple
installation methods is that the installer will be required to provide
certain information about the cluster nodes for use by OSCAR in
installing and configuring the cluster software.

\subsection{Software Management}
\label{sec:reqs-software-management}

One of the goals for the next generation of OSCAR is to include a
wider variety of cluster software and to allow users to select which
packages they wish to install on their cluster. As such, a mechanism
must be in place for OSCAR 2.0 that allows new packages to be easily
added to the OSCAR bundle without requiring any changes to the
infrastructure. Naturally, the solution to this requirement lends
itself towards an API that new and existing packages must conform to
in order to be integrated into OSCAR. In addition, the mechanism must
support a installation scheme that is general enough for all possible
software packages, since the OS installation can no longer be used for
software installs. The package management API was designed accordingly
and specifies the functionalities that each package integrater must
provide. These basic functionalities include installation,
uninstallation, and configuration. The OSCAR package management API is
discussed in more detail in
Section~\ref{sec:design-software-package-mgmt}
(page~\pageref{sec:design-software-package-mgmt}).

\subsection{Cluster Management}

Another problem area voiced by users of the first version was that
there was no documentation for how to add nodes to a cluster after the
original installation. The reason there was no documentation was that
we did not support this operation in 1.0, as the process was too
strictly defined and supporting such an operation would have required
much change to the existing infrastructure.

However, this is obviously a great need by many people, so it has been
added as a requirement for version 2.0 that users will be able to add
and remove nodes at will from their cluster, and OSCAR will be
responsible for doing any associated software configuration updates
using the configure functionality provided by each package.

% LocalWords:  Exp RedHat LUI SystemImager SystemInstaller SystemConfigurator
% LocalWords:  PowerPC MSC TurboLinux Debian vs KickStart CD API integrater
% LocalWords:  uninstallation
