% -*- latex -*-
%
% $Id: ramdisk.tex,v 1.1 2001/12/13 22:14:12 mchasal Exp $
%
% $COPYRIGHT$
%

\section{Generating Initial Ramdisks}
\label{app:ramdisk}

You will need to create an initial ramdisk to support the hardware of
your client, including any SCSI devices and network cards. If your
client and server machines have identical hardware and the kernel
running on the server is the same as the one to be installed on the
clients, you may skip this step and OSCAR will automatically create
and allocate the ramdisk resource. If your client machines do not
contain SCSI, you may be able to get away with not creating an initial
ramdisk and just creating a \file{/etc/modules.conf} source resource
that contains an entry for you client machine's network adapter. If
you are unsure of the format for the \file{/etc/modules.conf} file,
you should probably just build the initial ramdisk with support for
your clients' network adapter as described below.

To create an initial ramdisk for your client machines, you will use
the \cmd{mkinitrd} command. Using the command, you can create a
ramdisk to support any special hardware your clients may contain. If
the client machines contain SCSI disks, you will need to build support
for the SCSI adapter into your ramdisk. In addition, you should build
in support for your client's ethernet adapter.

Before creating the initial ramdisk, you should be aware of some
important caveats. The first caveat is that \msg{the ramdisk you
  create must match the kernel to be installed on the clients}. As LUI
automatically installs the appropriate kernel (UP/SMP) based upon the
number of processors in the client machine, you should create a
ramdisk that matches this kernel. The second caveat is that \msg{in
  order to build a ramdisk for a particular kernel version, the
  kernel's associated modules must be located on the server in
  \file{/lib/modules/kernel-version}}.

Now that you are aware of the caveats described above, you are ready
to build an initial ramdisk for your clients. A typical command using
the kernel currently running on your server is as follows:

\begin{verbatim}
  /sbin/mkinitrd -v --with=eth-module client-initrd.img `uname -r`
\end{verbatim}

In the above command, ``\file{eth-module}'' is the name of the module
for the client's ethernet adapter, e.g., \file{eepro100},
``\file{client-initrd.img}'' is the name of the ramdisk to create, and
``\cmd{uname -r}'' returns the version of the currently running
kernel. The above command also assumes the client nodes use the same
disk adapter(s) as the server system. If alternate adapters are used,
specify them before the ethernet adapter with additional
``\cmd{--with}'' arguments.

If you defined custom kernel and system map resources, then be sure to
specify the appropriate kernel version as the last argument. For
example, the command

\begin{verbatim}
  /sbin/mkinitrd -v --with=aic7xxx --with=eepro100 \
     client-initrd.img 2.2.17
\end{verbatim}

\noindent will create an initial ramdisk with support for the AIC7xxx
series of Adaptec SCSI adapters and the Intel EtherExpress Pro 100
ethernet adapter using the modules located in
\file{/lib/modules/2.2.17}.

For additional information on how to use \cmd{mkinitrd}, see its man
page, i.e., ``\cmd{man mkinitrd}''.

