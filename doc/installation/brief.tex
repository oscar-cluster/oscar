% -*- latex -*-
%
% $Id: brief.tex,v 1.2 2001/12/17 22:41:58 mchasal Exp $
%
% $COPYRIGHT$
%

\begchange
\section{Quick Start Installation of Homogenous Clusters on a Private Subnet}

%%%%%%%%%%%%%%%%%%%%%%%%%%%%%%%%%%%%%%%%%%%%%%%%%%%%%%%%%%%%%%%%%%%%%%%%%%

\subsection{Why you shouldn't do a Quick Start install}

If you meet the following criteria and are very brave, you can 
try to install your cluster using the brief, mysterious,
cryptic, terse, and obscure documentation in this section.  
Otherwise, please do the right thing and read and use the 
detailed installation procedure section of this installation guide.
If you don't understand any of the following criteria, 
then you probably don't meet them:

\begin{enumerate}
\item You hate to read manuals, documentation, or other printed matter.
\item You are a unix guru. (You can uset \cmd{netcfg} or an editor to
  configure network interfaces.)
\item Your cluster client machines are on a private subnet.
\item Your cluster server machine has two network interfaces, one public,
  and one connected to the cluster client machines.
\item Your cluster server and client machines contain identical hardware.
\item Server's root (\file{/}) filesystem is on the same disk type (IDE
  or SCSI) as the clients will use for their root filesystem.
\item The default RPM list supplied by OSCAR
  (\file{OSCAR-\oscarversion/oscarResources/sample.rpmlist}) is
  acceptable for your clients.
\item You are using the RedHat 7.1 distribution.
\item You know how to install the RedHat 7.1 distribution on a machine.
\item You've never read completely through the installation instructions
  of anything, ever. 
\end{enumerate}

\subsection{Quick installation procedures}

Note: All actions specified herein should be performed by the
\user{root} user on the service machine unless noted otherwise.

%%\subsubsection{Install Linux on your server machine}

Install (or already have installed) Linux on your server machine.
The only requirements for your Linux installation are:

\begin{enumerate}
\item Some X environment such as GNOME or KDE must be installed.
\item Networking must be set up and working on the public interface.
  (Do yourself a favor and install some type of network security if
  your system is exposed to the general internet.)
\item The second network interface for the private cluster network
  must be installed.
\end{enumerate}

%%\subsubsection{Get and unpack the OSCAR software}

Download and unpack OSCAR with these commands:

\begin{verbatim}
  # cd /root
  # ncftp ftp.sourceforge.net
  ncftp / > cd pub/sourceforge/oscar
  ncftp /pub/sourceforge/oscar > get oscar-1.1.1.tgz
  ncftp /pub/sourceforge/oscar > quit
  # tar -zxf oscar-1.1.1.tgz
\end{verbatim}
    
%%\subsubsection{Configure the second ethernet adapter} 

Configure the second (private) cluster network adapter using the linux
\cmd{/usr/sbin/netcfg} command or your favorite editor. Set the interface 
IP address to 10.0.0.250, set the interface configuration protocol
to ``none'', and set the interface to activate at boot time.
Then reboot your machine. and make sure that the private cluster
interface is properly setup and activated. 

%%\subsubsection{Edit \file{/etc/hosts}}

Edit the \file{/etc/hosts} file to make sure it has an entry for
the private network IP address of your server machine, and add entries
for the IP addresses of your client machines.

\noindent{\bf IMPORTANT:} the value returned by the \cmd{hostname}
command has to be included in the \file{/etc/hosts} file {\bf ONCE and
  ONLY ONCE} on the line that has the private network IP address of
the server machine.  If the \cmd{hostname} command returns ``envy''
and you have two client machines named node1 and node2, and PUBLICIP
is the public IP address of the server, the sequence of commands might
look like this:

\begin{verbatim}
  # hostname
  envy
  # vi /etc/hosts
  # cat /etc/hosts
  127.0.0.1       localhost.localdomain   localhost
  PUBLICIP        envy.domain.name
  10.0.0.250      envy
  10.0.0.1        node1
  10.0.0.2        node2
\end{verbatim}

%%\subsubsection{Start the OSCAR configuration wizard}

To start the OSCAR configuration wizard, in the X environment do
the following command:

\begin{verbatim}
  cd /root/OSCAR-1.1.1
  ./install_cluster ethernet-device
\end{verbatim}
  
In the above command, substitute the device name 
(e.g., \file{eth1})
for your server's private network ethernet adapter. After the OSCAR 
wizard successfully completes some startup commands, it will display 
a list of steps to be done.

%%\subsubsection{Define the server}

Press the Step 1 button of the wizard entitled \button{Define the Server}. 
In the dialog box that is displayed, fill in these fields:

\begin{enumerate}
\item Name: envy (the name of the server machine. This should be the
  same name returned by the \cmd{hostname} command, and the
  name in the \file{/etc/hosts} file name for the private cluster network
  IP address of the server machine.
\item IP Address: 10.0.0.250 
  (the private cluster network address of the server machine)
\item Netmask: 255.255.255.0 (in most cases)
\end{enumerate}

Then press the \button{Apply} button. Ignore any warning
messages about the nfsd service and the exporting of the \file{/usr}
and \file{/tftpboot} directories that are displayed in the output.
After you see the ``successfully created the machine object named envy''
line and the output stops, press the \button{Close} button.

%%\subsubsection{Collect client MAC addresses} 

If you already know the MAC addresses of your client machines, edit
the file \file{/etc/MAC.info} to look something like this:

\begin{verbatim}
  node1   12:33:41:5c:29:01
  node2   12:33:41:8a:25:19
\end{verbatim}

The first field of each line is ignored, all that matters is the MAC address.
If you don't know the MAC addresses for your client machines, use the
OSCAR wizard to collect them like this:

\begin{enumerate}
\item Press the button of the wizard entitled 
  \button{Collect Client MAC Addresses}. The OSCAR MAC address collection
  utility dialog box will be displayed.
\item To start collecting the MAC address for another client node, 
  press the \button{Collect} button.
\item Network boot the client.
\item After the client MAC address is collected, a line with 
  ``found mac address'' will be displayed.
\item If another client's MAC address needs to be collected,
  go back to step 2.
\item When you have collected the
  addresses for all your client nodes, press \button{Done}.
\end{enumerate}

%%\subsubsection{Define your client machines} 

Press the Step 3 button of the wizard entitled \button{Define the
  Client Machines}. In the dialog box that is displayed, enter the
  following information:

\begin{enumerate}
\item Starting IP Address: 10.0.0.1 (or whatever the first client
  machine private IP address is)
\item Netmask: 255.255.255.0 (for most private networks)
\item Default Route(optional): 10.0.0.250 (the server's private network 
  IP address)
\end{enumerate}
  
When finished entering information, press the \button{Apply} button.
When the message ``Successfully created the machine object LASTCLIENT''
is displayed (where LASTCLIENT is the name of the last client machine),
press the \button{Close} button.

%%\subsubsection{Prepare for Client Installs}

Press the Step 6 button of the wizard entitled 
\button{Prepare for Client Installs}. This will
run the pre client installation server configuration script.
When this is completed, the last line of the output should
be ``Begin booting client nodes''.

%%\subsubsection{Client Installations}

Network boot the client machines. They will automatically be
installed and configured. 
See Appendix~\ref{app:net-boot-client-nodes} for instructions on
network booting clients.

%%\subsubsection{Check completion status of nodes}

For each client, a log is kept detailing the progress of its
installation. The log files for all clients are kept on the server in
\file{/tftpboot/lim/log}. When a client installation completes, the
last line in the log for that client will read ``\msgout{installation
  is now complete, time to reboot!}'' Depending on the capabilities of
your server and the number of simultaneous client installations, a
client could take anywhere from five minutes to over an hour to
complete its installation.
  
%%\subsubsection{Reboot the client nodes}

After confirming that a client has completed its installation, you
should reboot the node from its hard drive. 

%%\subsubsection{Check network connectivity to client nodes}

In order to perform the final cluster configuration, the server must
be able to communicate with the client nodes over the network. If a
client's ethernet adapter is not properly configured upon boot,
however, the server will not be able to communicate with the client. A
quick and easy way to confirm network connectivity is to do the
following (assuming OSCAR installed in \file{/root}):

\begin{verbatim}
  cd /root/OSCAR-1.1.1/scripts
  ./ping_clients
\end{verbatim}

The above commands will run the \cmd{ping\_clients} script, which will
attempt to ping each defined client and will print a message stating
success or failure. Once all the clients have been
installed, rebooted, and their network connections have been
confirmed, you may proceed with the next step.

%%\subsubsection{Complete the cluster configuration}

Press the Step 7 button of the wizard entitled \button{Complete
  Cluster Setup}.  This will run the \file{post\_install} script.

%%\subsubsection{Check for successful completion}

In the output window for the above step, search for a message stating
``\msgout{Congratulations, your cluster is now ready for use.}''
Provided along with OSCAR is a simple test to make sure the key
cluster components (PBS, MPI, and PVM) are functioning properly. For
information on installing and running the software, see the
\file{oscar\_testing} document in the \file{docs} subdirectory.

\endchange
