% -*- latex -*-
%
% $Id: netboot.tex,v 1.2 2001/12/17 03:32:16 mchasal Exp $
%
% $COPYRIGHT$
%

\section{Network Booting Client Nodes}
\label{app:net-boot-client-nodes}

There are two methods available for network booting your client nodes.
The first is to use the Preboot eXecution Environment (PXE) network
boot option in the client's BIOS, if available. If the option is not
available, you will need to create a network boot floppy disk using
the SystemImager boot package. Each method is described below.

\begin{enumerate}
\item \msg{Network booting using PXE.} To use this method, your client
  machines' BIOS and network adapter will need to support PXE version
  2.0 or later. The PXE specification is available at
  \url{http://developer.intel.com/ial/wfm/tools/pxepdk20/index.htm}.
  Earlier versions may work, but experience has shown that versions
  earlier than 2.0 are unreliable. As BIOS designs vary, there is not
  a standard procedure for network booting client nodes using PXE.
  More often than not, the option is presented in one of two ways.

  \begin{enumerate}
  \item The first is that the option can be specified in the BIOS boot
    order list. If presented in the boot order list, you will need to
    set the client to have network boot as the first boot device. In
    addition, when you have completed the client installation,
    remember to reset the BIOS and remove network boot from the boot
    list so that the client will not attempt to do the installation
    again.
    
    b)The second is that the user must watch the output of the client
    node while booting and press a specified key such as ``N'' at the
    appropriate time. In this case, you will need to do so for each
    client as it boots.
  \end{enumerate}
  
\item \msg{Network booting using a SystemImager boot floppy.} The SystemImager
  boot package is provided with OSCAR just in case your machines do not
  have a BIOS network boot option.
  You can create a boot floppy through the \file{oscar\_wizard} on the 
  \button{Setup Networking} panel or by using the file{mkautoinstalldiskette}
  command.
\end{enumerate}

Once you have created the SystemImager boot floppy,
set your client's BIOS to boot from the floppy
drive. Insert the floppy and boot the machine to start the network
boot. Check the output for errors to make sure your network boot
floppy is working properly. Remember to remove the floppy when you
reboot the clients after installation.
