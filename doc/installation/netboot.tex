% -*- latex -*-
%
% $Id: netboot.tex,v 1.5 2002/07/20 17:20:46 jsquyres Exp $
%
% $COPYRIGHT$
%

\section{Network Booting Client Nodes}
\label{app:net-boot-client-nodes}

There are two methods available for network booting your client nodes.
The first is to use the Preboot eXecution Environment (PXE) network
boot option in the client's BIOS, if available. If the option is not
available, you will need to create a network boot CD disk using
the SystemImager boot package or use an Etherboot disk. Each method 
is described below.

\begin{enumerate}
\item \msg{Network booting using PXE.} To use this method, the BIOS
  and network adapter on each of the client nodes will need to support
  PXE version 2.0 or later. The PXE specification is available at
  \url{http://developer.intel.com/ial/wfm/tools/pxepdk20/}.  Earlier
  versions may work, but experience has shown that versions earlier
  than 2.0 are unreliable. As BIOS designs vary, there is not a
  standard procedure for network booting client nodes using PXE.  More
  often than not, the option is presented in one of two ways.

  \begin{enumerate}
  \item The first is that the option can be specified in the BIOS boot
    order list. If presented in the boot order list, you will need to
    set the client to have network boot as the first boot device. In
    addition, when you have completed the client installation,
    remember to reset the BIOS and remove network boot from the boot
    list so that the client will boot from its local hard drive and
    will not attempt to do the installation again.
    
  \item The second is that the user must watch the output of the
    client node while booting and press a specified key such as ``N''
    at the appropriate time. In this case, you will need to do so for
    each client as it boots.
  \end{enumerate}
  
\item \msg{Network booting using a SystemImager boot CD.} The
  SystemImager boot package is provided with OSCAR just in case your
  machines do not have a BIOS network boot option.  You can create a
  boot CD through the OSCAR GUI installation wizard on the
  \button{Setup Networking} panel or by using the
  \file{mkautoinstallCD} command.
  
  Once you have created the SystemImager boot CD, set your
  client's BIOS to boot from the CD drive. Insert the CD and
  boot the machine to start the network boot. Check the output for
  errors to make sure your network boot CD is working properly.
  Remember to remove the CD when you reboot the clients after
  installation.

\item \msg{Using an Etherboot disk} Etherboot is a software package
  for creating ROM images.  This type of image is what drives the PXE
  network boot process described above.  However, the Etherboot
  package (\url{http://www.etherboot.org/}) can also be used to create
  bootable flopy diskettes that mimic the PXE functionality of many
  network cards.  This is useful for both older systems, and because
  booting off a diskette is sometimes easier than fiddling around with
  BIOS settings.

  A users manual with installation instructions can be found on the
  project's website (\url{http://www.etherboot.org/}). This tool is
  not supported by the OSCAR team directly, but is very handy.

\end{enumerate}

% LocalWords:  Exp
