% -*- latex -*-
%
% $Id: troubleshooting.tex,v 1.9 2002/06/09 17:20:45 jsquyres Exp $
%
% $COPYRIGHT$
%

\section{Troubleshooting}
\label{app:troubleshooting}

%%%%%%%%%%%%%%%%%%%%%%%%%%%%%%%%%%%%%%%%%%%%%%%%%%%%%%%%%%%%%%%%%%%%%%%%%%

\subsection{An overview of \cmd{switcher}}
\label{app:switcher-overview}

\begchange

\begin{discuss}
  This entire section should probably go in a User's Manual.
\end{discuss}

Experience has shown that requiring untrained users to manually edit
their ``dot'' files (e.g., \file{\$HOME/.bashrc},
\file{\$HOME/.login}, \file{\$HOME/.logout}, etc.) can result in
damaged user environments.  Side effects of damaged user environments
include:

\begin{itemize}
\item Lost and/or corrupted work
\item Severe frustration / dented furniture
\item Spending large amounts of time debugging ``dot'' files, both by
  the user and the system administrator
\end{itemize}

The OSCAR \cmd{switcher} package is an attempt to provide a simple
mechanism to allow users to manipulate their environment.  The
\cmd{switcher} package provides a convenient command-line interface to
manipulate the inclusion of packages in a user's environment.  Users
are not required to manually edit their ``dot'' files, nor are they
required to know what the inclusion of a given package in the
environment entails.\footnote{Note, however, that it was a requirement
  for the OSCAR \cmd{switcher} package that advanced users should not
  be precluded -- in any way -- from either not using \cmd{switcher},
  or otherwise satisfying their own advanced requirements without
  interference from \cmd{switcher}.}  For example, if a user specifies
that they want LAM/MPI in their environment, \cmd{switcher} will
automatically add the appropriate entries to the \file{\$PATH} and
\file{\$MANPATH} environment variables.

Finally, the OSCAR \cmd{switcher} package provides a two-level set of
defaults: a system-level default and a user-level default.  User-level
defaults (if provided) override corresponding system-level defaults.
This allows a system administrator to (for example) specify which MPI
implementation users should have in their environment by setting the
system-level default.  Specific users, however, may decide that they
want a different implementation in their environment and set their
personal user-level default.

Note, however, that {\em \cmd{switcher} does not change the
  environment of the shell from which it was invoked.}  This is a
critical fact to remember when administrating your personal
environment or the cluster.  While this may seem inconvenient at
first, \cmd{switcher} was specifically designed this way for two
reasons:

\begin{enumerate}
\item If a user inadvertantly damages their environment using
  \cmd{switcher}, there is still [potentially] a shell with an
  undamaged environment (i.e., the one that invoked \cmd{switcher})
  that can be used to fix the problem.
  
\item The \cmd{switcher} package uses the \cmd{modules} package for
  most of the actual environment manipulation (see
  \url{http://modules.sourceforge.net/}).  The \cmd{modules} package
  can be used directly by users (or scripts) who wish to manipulate
  their current environment.
\end{enumerate}

The OSCAR \cmd{switcher} package contains two sub-packages:
\cmd{modules} and \cmd{env-switcher}.  The \cmd{modules} package can
be used by itself (usually for advanced users).  The
\cmd{env-switcher} package provides a persistent \cmd{modules}-based
environment.

\subsubsection{The \cmd{modules} package}

The \cmd{modules} package (see \url{http://modules.sourceforge.net/})
provides an elegant solution for individual packages to install (and
uninstall) themselves from the current environment.  Each OSCAR
package can provide a modulefile that will set (or unset) relevant
environment variables, create (or destroy) shell aliases, etc.

An OSCAR-ized \cmd{modules} RPM is installed during the OSCAR
installation process.  Installation of this RPM has the following
notable effects:

\begin{itemize}
\item Every user shell will be setup for modules -- notably, the
  commands ``{\cmd module}'' and ``{\cmd man module}'' will work as
  expected.

\item Guarantee the execution of all modulefiles in a specific
  directory for every shell invocation (including corner cases such as
  non-interactive remote shell invocation by \cmd{rsh}/\cmd{rsh}).
\end{itemize}

Most users will not use any \cmd{modules} commands directly -- they
will only use the \cmd{env-switcher} package.  However, the
\cmd{modules} package can be used directly by advanced users (and
scripts).

\subsubsection{The \cmd{env-switcher} package}

The \cmd{env-switcher} package provides a persistent
\cmd{modulues}-based environment.  That is, \cmd{env-switcher} ensures
to load a consistent set of modules for each shell invocation
(including corner cases such as non-interactive remote shells via
\cmd{rsh}/\cmd{ssh}).  \cmd{env-switcher} is what allows users to
manipulate their environment by using a simple command line interface
-- not by editing ``dot'' files.

It is important to note that {\em using the \cmd{switcher} command
  alters the environment of all {\bf future} shells / user
  environments.  \cmd{switcher} does not change the environment of the
  shell from which it was invoked.}  This may seem seem inconvenient
at first, but was done deliberately.  See the rationale provided at
the beginning of this section for the reasons why.

\cmd{env-switcher} manipulates three different kinds of entities:
names, attributes, and values.  ``Names'' are best thought of as
``packages''.  In OSCAR, for example, an obvious name that is used
frequently is ``mpi''.  Each name can have zero or more attributes.
An attribute, if defined, must have a single value.  An attribute
specifies something about a given name by having an assigned value.
There are a few built-in attributes with special meanings (any other
attribute will be ignored by \cmd{env-switcher}, and can therefore be
used to cache arbitrary values).  ``default'' is probably the
most-commonly used attribute -- its value specifies which package will
be loaded.

For example, setting the ``default'' attribute on the ``mpi'' name to
a given value will control which MPI implementation is loaded into the
environment.

\cmd{env-switcher} operates at two different levels: system-level and
user-level.  The system-level names, attributes, and values are stored
in a central location.  User-level names, attributes, and values are
stored in each user's \file{\$HOME} directory.

When \cmd{env-switcher} looks up entity that it manipulates (for
example, to determine the value of the ``default'' attribute on the
``mpi'' name), it attempts to resolves the value in a specific
sequence:

\begin{enumerate}
\item Look for a ``default'' attribute value on the ``mpi'' name in
  the user-level defaults
  
\item Look for a ``default'' attribute value on the ``global'' name in
  the user-level defaults
  
\item Look for a ``default'' attribute value on the ``mpi'' name in
  the system-level defaults
  
\item Look for a ``default'' attribute value on the ``global'' name in
  the system-level defaults
\end{enumerate}

In this way, a four-tiered set of defaults can be effected: specific
user-level, general user-level, specific system-level, and general
system-level.  

Using that terminology, the most command \cmd{env-switcher} commands
that users will invoke are:

\begin{enumerate}
\item \cmd{switcher --list}
  
  List all available names.

\item \cmd{switcher <name> --list}
  
  List all attributes that have values for the name \cmd{<name>}.

\item \cmd{switcher <name> = <value> [--system]} 
  
  A shortcut nomenclature to set the ``default'' attribute on
  \cmd{<name>} equal to the value \cmd{<value>}.  Note that the
  special value ``\cmd{none}'' will remove the value from a given
  attribute.  If the \cmd{--system} parameter is used, the change will
  affect the system-level defaults; otherwise, the user's personal
  user-level defaults are changed.

\item \cmd{switcher <name> --show}

  Show the all attribute / value pairs for the name \cmd{<name>}.  The
  values shown will be for attributes that have a resolvable value
  (using the resolution sequence described above).  Hence, this output
  may vary from user to user for a given \cmd{<name>} depending on the
  values of user-level defaults.

\end{enumerate}

Appendix~\ref{app:which-mpi-to-use} shows an example scenario using
the \cmd{switcher} command detailing how to change which MPI
implementation is used, both at the system-level and user-level.

See the man page for \cmd{switcher(1)} and the output of \cmd{switcher
  --help} for more details on the \cmd{switcher} command.

\endchange

%%%%%%%%%%%%%%%%%%%%%%%%%%%%%%%%%%%%%%%%%%%%%%%%%%%%%%%%%%%%%%%%%%%%%%%%%%

\subsection{Which MPI do you want to use?}
\label{app:which-mpi-to-use}

\begchange

\begin{discuss}
  This entire section should probably go in a User's Manual.
\end{discuss}

Starting with the OSCAR 1.3 series, there is a generalized mechanism
to both set a system-level default MPI implementation, and also to
allow users to override the system-level default with their own choice
of MPI implementation.

This allows multiple MPI implementations to be installed on an OSCAR
cluster (e.g., LAM/MPI and MPICH), yet still provide unambiguous MPI
implementation selection such that ``\cmd{mpicc foo.c -o foo}'' will
give deterministic results.

\subsubsection{Setting the system-level default}

The system-level default MPI implementation can be set in two ways:

\begin{enumerate}
\item During the OSCAR installation, the GUI will prompt asking which
  MPI should be the system-level default.  This will set the default
  for all users on the system who do not provide their own individual
  MPI settings.

\item As root, execute the command:

\begin{verbatim}
  % switcher mpi --list
\end{verbatim}

   This will list all the MPI implementations available.  To set the
   system-level default, execute the command:

\begin{verbatim}
  % switcher mpi = name --system
\end{verbatim}
   
   where ``name'' is one of the names from the output of the
   \cmd{--list} command.
\end{enumerate}

{\bf NOTE:} Using the \cmd{switcher} command to change the default MPI
implementation will modify the \cmd{PATH} and \cmd{MANPATH} for all
{\em future} shell invocations -- it does {\em not} change the
environment of the shell in which it was invoked.  For example:

\begin{verbatim}
  % which mpicc
  /opt/lam-1.2.3/bin/mpicc
  % switcher mpi = mpich-4.5.6 --system
  % which mpicc
  /opt/lam-1.2.3/bin/mpicc
  % csh
  % which mpicc
  /opt/mpich-4.5.6/bin/mpicc
\end{verbatim}

\subsubsection{Setting the user-level default}

Setting a user-level default is essentially the same as setting the
system-level default, except without the \cmd{--system} argument.
This will set the user-level default instead of the system-level
default.  Using the special name \cmd{none} will remove the user-level
default and revert the user to the system-level default.

\begin{verbatim}
  # Set the user's default, overriding the system default:
  % switcher mpi = lam-1.2.3
  # Remove the user's default, and return to whatever the system
  # default is:
  % switcher mpi = none
\end{verbatim}

{\bf WARNING: The \cmd{switcher} command must be used with care!}  It
immediately affects the environment of all future shell invocations
(including the environment of scripts).  To get a full list of options
available, read the \cmd{switcher(1)} man page, and/or run
\cmd{switcher --help}.  Also see Appendix~\ref{app:switcher-overview},
``An overview of \cmd{switcher}''.

\endchange

%%%%%%%%%%%%%%%%%%%%%%%%%%%%%%%%%%%%%%%%%%%%%%%%%%%%%%%%%%%%%%%%%%%%%%%%%%

\subsection{Managing machines and images}
\label{app:troubleshooting-machines-images}

During the life of your cluster, you may want to delete unused
machines or images, create new images, or change the image that a
client uses.  Currently OSCAR doesn't have a direct interface to do
this, but you can use the SIS commands directly. Here are some useful
examples:

\begin{itemize}
\item To list all defined machines, run:
\begin{verbatim}
        mksimachine --List
\end{verbatim}
\item To list all defined images, run:
\begin{verbatim}
        mksiimage --List
\end{verbatim}
\item To delete an image, run:
\begin{verbatim}
        mksiimage --Delete --name <imagename>
\end{verbatim}
\item To delete a machine, run:
\begin{verbatim}
        mksimachine --Delete --name <machinename>
\end{verbatim}
\item To delete all machines, run:
\begin{verbatim}
        mksimachine --Delete --all
\end{verbatim}
\item To change which image a machine will install, run:
\begin{verbatim}
        mksimachine --Update --name <machinename> --image <imagename>
\end{verbatim}
\end{itemize}

There is also a SIS gui that is availble. Start it by running
\file{tksis}. It doesn't yet support the update function, but it can
make the other operations easier.

More details on these commands can be obtained from their respective
man pages.

%%%%%%%%%%%%%%%%%%%%%%%%%%%%%%%%%%%%%%%%%%%%%%%%%%%%%%%%%%%%%%%%%%%%%%%%%%

\subsection{Known Problems and Solutions}
\label{app:troubleshooting-known-problems}

\subsubsection{Client nodes fail to network boot}
\label{app:troubleshooting-known-problems-dhcp}

There are two causes to this problem. The first is that the DHCP
server is not running on the server machine, which probably means the
\file{/etc/dhcpd.conf} file format is invalid.  Check to see if it is
running by running the command ``\cmd{service dhcpd status}'' in the
terminal.  If no output is returned, the DHCP server is not running.
See the problem solution for ``DHCP server not running'' below. If the
DHCP server is running, the client probably timed out when trying to
download its configuration file. This may happen when a client is
requesting files from the server while multiple installs are taking
place on other clients. If this is the case, just try the network boot
again when the server is less busy. Occasionally, restarting the inet
daemon also helps with this problem as it forces tftp to restart as
well. To restart the daemon, issue the following command:

\begin{verbatim}
  service xinetd restart
\end{verbatim}

\subsubsection{DHCP server not running}

Run the command ``\cmd{service dhcpd start}'' from the terminal and
observe the output. If there are error messages, the DHCP
configuration is probably invalid. A few common errors are documented
below. For other error messages, see the \file{dhcpd.conf} man page.

\begin{enumerate}
\item If the error message produced reads something like
  ``\msgout{Can't open lease database}'', you need to manually create
  the DHCP leases database, \file{/var/lib/dhcp/dhcpd.leases}, by
  issuing the following command in a terminal:

\begin{verbatim}
  touch /var/lib/dhcp/dhcpd.leases
\end{verbatim}
  
\item If the error message produced reads something like ``\msg{Please
    write a subnet declaration for the network segment to which
    interface ethx is attached}'', you need to manually edit the DHCP
  configuration file, \file{/etc/dhcpd.conf}, in order to try to get
  it valid. A valid configuration file will have at least one subnet
  stanza for each of your network adapters. To fix this, enter an
  empty stanza for the interface mentioned in the error message, which
  should look like the following:

\begin{verbatim}
  subnet subnet-number netmask subnet-mask { }
\end{verbatim}
  
  The subnet number and netmask you should use in the above command
  are the one's associated with the network interface mentioned in the
  error message.
\end{enumerate}

\subsubsection{PBS is not working}

\begin{discuss}
  This has already been solved, and should be removed, right?
\end{discuss}

The PBS configuration done by OSCAR did not complete successfully and
requires some manual tweaking. Issue the following commands to
configure the server and scheduler:

% We have to use \tt instead of {verbatim} because we need to use
% \oscarversion inside.  This makes it somewhat painful -- much less
% easy than {verbatim}.
\vspace{11pt}
{\tt
  service pbs\_server start \\
\indent  service maui start \\
\indent  cd /root/oscar-\oscarversion/pbs/config \\
\indent  /usr/local/pbs/bin/qmgr < pbs\_server.conf
}
\vspace{11pt}

Replace ``\file{/root}'' with the directory into which you unpacked
OSCAR in the change directory command above.

\subsubsection{Uni-processor P4 nodes fail to boot}

\begin{discuss}
  This has already been solved, and should be removed, right?
\end{discuss}

This problem has occured on some newer machines after the 'oscar\_wizard' 
has run and the nodes have rsync'd their files to the local harddrive. 
The nodes then restart and begin to boot the newly installed kernel
and hang with a message similar to the following,
\begin{verbatim}
       Getting VERSION: 0
       Getting VERSION: ff00ff
       enabled ExtINT on CPU#0
       ESR value before enabling vector: 00000000
       ESR value after enabling vector: 00000000
       calibrating APIC timer ...
       ..... CPU clock speed is 1995.0407 Mhz.
       ..... host bus clock speed is 0.0000 Mhz.
       cpu: 0, clocks: 0, slice: 0
       _
\end{verbatim}
The cursur just blinks here forever.

This problem occurs because the SMP kernel is installed and the
machine needs a standard uni-processor kernel.  Most machines will boot
normally with the SMP kernel but a small number exhibit this issue.
Currently, the simplest fix is to rebuild the image with the
uni-processor kernel (e.g. \emph{kernel-2.4.2-2}) in the rpmlist 
(e.g. \emph{oscarsamples/sample.rpmlist}) and re-install the failing node.  

We are currently investigating the problem further.  If you are experiencing 
this problem, please check the OSCAR web page at
{\tt http://oscar.sourceforge.net/} for the latest information and solutions
to this problem.

%%%A workable fix for this would be to do the following steps to
%%%install the uni-processor kernel (\emph{kernel-2.4.2-2.i386.rpm}) into 
%%%an existing image named \emph{oscarimage}.  After that you can repeat
%%%the installation of the nodes, using either network boot (PXE) or a
%%%boot floppy.   This will do a complete re-install of the node.
%%%\begin{verbatim}
%%%       [root@oscar /root]# cp /tftpboot/rpm/kernel-2.4.2-2.i386.rpm \
%%%       /var/lib/systemimager/images/oscarimage/tmp
%%%       [root@oscar /root]# chroot /var/lib/systemimager/oscarimage
%%%       [root@oscar /]# rpm -ivh /tmp/kernel-2.4.2-2.i386.rpm 
%%%       [root@oscar /]# rm /tmp/kernel-2.4.2-2.i386.rpm
%%%       [root@oscar /]# cd boot
%%%       [root@oscar boot]# ln -sf kernel.h-2.4.2  kernel.h
%%%       [root@oscar boot]# ln -sf module-info-2.4.2-2  module-info
%%%       [root@oscar boot]# ln -sf System.map-2.4.2-2  System.map
%%%       [root@oscar boot]# ln -sf vmlinuz-2.4.2-2  vmlinuz
%%%       [root@oscar /]# exit
%%%       [root@oscar /root]#
%%%\end{verbatim}
% extra dir listing i took out of the above.
%       [root@oscar boot]# ls *2.4.2*
%       kernel.h-2.4.2          System.map-2.4.2-2     vmlinux-2.4.2-2smp
%       module-info-2.4.2-2     System.map-2.4.2-2smp  vmlinuz-2.4.2-2
%       module-info-2.4.2-2smp  vmlinux-2.4.2-2        vmlinuz-2.4.2-2smp
%Now, you need backup and edit the \emph{oscarimage.master} script so
%that it skips the partitioning of the disk and just goes onto the
%rsync'ing of the files.  NOTE: THIS IS THE SECTION THAT I NEED TO 
%FINISH UP B/C I'VE NOT TESTED IT YET ... THEREFORE I'M NOT GOING TO
%INCLUDE IT HERE.


\subsection{What to do about unknown problems?}

For help in solving problems not covered by this HowTo, send a
detailed message describing the problem to the OSCAR users mailing
list at \email{oscar-users@lists.sourceforge.net}. You may also wish
to visit the OSCAR web site, \url{http://oscar.sourceforge.net/}, for
updates on newly found and resolved problems.

\subsection{Starting over -- installing OSCAR again}

\begin{discuss}
  The \cmd{start\_over} script isn't interactive, is it?
\end{discuss}

If you feel that you want to start the cluster installation process
over from scratch in order to recover from irresolvable errors, you
can do so with the \cmd{start\_over} script located in the
\file{scripts} subdirectory. This script is interactive, and will
prompt you when removing components installed by OSCAR that you may
not want to remove.

\begchange

It is important to note that \cmd{start\_over} is {\em not} an
uninstaller.  That is, \cmd{start\_over} does {\em not} guarantee to
return the head node to the start that it was in before OSCAR was
installed.  It does a ``best attempt'' to do so, but the only
guarantee that it provides is that the head node will be suitable for
OSCAR re-installation.  For example, the RedHat 7.x series ships with
a LAM/MPI RPM.  The OSCAR install process removes this RedHat-default
RPM and installs a custom OSCAR-ized LAM/MPI RPM.  The
\cmd{start\_over} script only removes the OSCAR-ized LAM/MPI RPM -- it
does not re-install the RedHat-default LAM/MPI RPI.

Another important fact to note is that because of the environment
manipulation that was performed via \cmd{switcher} from the previous
OSCAR install, {\em it is necessary to re-install OSCAR from a shell
  that was not tainted by the previous OSCAR installation}.
Specifically, the \cmd{start\_over} script can remove most files and
packages that were installed by OSCAR, but it cannot chase down and
patch up any currently-running user environments that were tainted by
the OSCAR environment manipulation packages.

Ensuring to have an untainted environment can be done in one of two
ways:

\begin{enumerate}
\item After running \cmd{start\_over}, completely logout and log back
  in again before re-installing.  {\em Simply launching a new shell
    may not be sufficient} (e.g., if the parent environment was
  tainted by the previous OSCAR install).  This will completely erase
  the previous OSCAR installation's effect on the environment in all
  user shells, and establish a set of new, untainted user
  environments.
  
\item Use a shell that was established {\em before} the previous OSCAR
  installation was established.  Although perhaps not entirely
  intuitive, this may include the shell was that initially used to
  install the previous OSCAR installation (see
  Appendix~\ref{app:switcher-overview} for the rationale as to why
  this may be true).
\end{enumerate}

\endchange

% LocalWords:  Exp
