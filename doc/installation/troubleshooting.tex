% -*- latex -*-
%
% $Id: troubleshooting.tex,v 1.12 2002/07/15 17:02:28 loann09 Exp $
%
% $COPYRIGHT$
%

\section{Troubleshooting}
\label{app:troubleshooting}

%%%%%%%%%%%%%%%%%%%%%%%%%%%%%%%%%%%%%%%%%%%%%%%%%%%%%%%%%%%%%%%%%%%%%%%%%%

\subsection{Managing machines and images}
\label{app:troubleshooting-machines-images}

During the life of your cluster, you may want to delete unused
machines or images, create new images, or change the image that a
client uses.  Currently OSCAR doesn't have a direct interface to do
this, but you can use the SIS commands directly. Here are some useful
examples:

\begin{itemize}
\item To list all defined machines, run:
\begin{verbatim}
        mksimachine --List
\end{verbatim}
\item To list all defined images, run:
\begin{verbatim}
        mksiimage --List
\end{verbatim}
\item To delete an image, run:
\begin{verbatim}
        mksiimage --Delete --name <imagename>
\end{verbatim}
\item To delete a machine, run:
\begin{verbatim}
        mksimachine --Delete --name <machinename>
\end{verbatim}
\item To delete all machines, run:
\begin{verbatim}
        mksimachine --Delete --all
\end{verbatim}
\item To change which image a machine will install, run:
\begin{verbatim}
        mksimachine --Update --name <machinename> --image <imagename>
\end{verbatim}
\end{itemize}

There is also a SIS gui that is availble. Start it by running
\file{tksis}. It doesn't yet support the update function, but it can
make the other operations easier.

More details on these commands can be obtained from their respective
man pages.

%%%%%%%%%%%%%%%%%%%%%%%%%%%%%%%%%%%%%%%%%%%%%%%%%%%%%%%%%%%%%%%%%%%%%%%%%%

\subsection{Known Problems and Solutions}
\label{app:troubleshooting-known-problems}

\subsubsection{Client nodes fail to network boot}
\label{app:troubleshooting-known-problems-dhcp}

There are two causes to this problem. The first is that the DHCP
server is not running on the server machine, which probably means the
\file{/etc/dhcpd.conf} file format is invalid.  Check to see if it is
running by running the command ``\cmd{service dhcpd status}'' in the
terminal.  If no output is returned, the DHCP server is not running.
See the problem solution for ``DHCP server not running'' below. If the
DHCP server is running, the client probably timed out when trying to
download its configuration file. This may happen when a client is
requesting files from the server while multiple installs are taking
place on other clients. If this is the case, just try the network boot
again when the server is less busy. Occasionally, restarting the inet
daemon also helps with this problem as it forces tftp to restart as
well. To restart the daemon, issue the following command:

\begin{verbatim}
  service xinetd restart
\end{verbatim}

\subsubsection{DHCP server not running}

Run the command ``\cmd{service dhcpd start}'' from the terminal and
observe the output. If there are error messages, the DHCP
configuration is probably invalid. A few common errors are documented
below. For other error messages, see the \file{dhcpd.conf} man page.

\begin{enumerate}
\item If the error message produced reads something like
  ``\msgout{Can't open lease database}'', you need to manually create
  the DHCP leases database, \file{/var/lib/dhcp/dhcpd.leases}, by
  issuing the following command in a terminal:

\begin{verbatim}
  touch /var/lib/dhcp/dhcpd.leases
\end{verbatim}
  
\item If the error message produced reads something like ``\msg{Please
    write a subnet declaration for the network segment to which
    interface ethx is attached}'', you need to manually edit the DHCP
  configuration file, \file{/etc/dhcpd.conf}, in order to try to get
  it valid. A valid configuration file will have at least one subnet
  stanza for each of your network adapters. To fix this, enter an
  empty stanza for the interface mentioned in the error message, which
  should look like the following:

\begin{verbatim}
  subnet subnet-number netmask subnet-mask { }
\end{verbatim}
  
  The subnet number and netmask you should use in the above command
  are the one's associated with the network interface mentioned in the
  error message.
\end{enumerate}

\subsubsection{PBS is not working}

%\begin{discuss}
%  This has already been solved, and should be removed, right?
%\end{discuss}

The PBS configuration done by OSCAR did not complete successfully and
requires some manual tweaking. Issue the following commands to
configure the server and scheduler:

% We have to use \tt instead of {verbatim} because we need to use
% \oscarversion inside.  This makes it somewhat painful -- much less
% easy than {verbatim}.
\vspace{11pt}
{\tt
  service pbs\_server start \\
\indent  service maui start \\
\indent  cd /root/oscar-\oscarversion/pbs/config \\
\indent  /usr/local/pbs/bin/qmgr < pbs\_server.conf
}
\vspace{11pt}

Replace ``\file{/root}'' with the directory into which you unpacked
OSCAR in the change directory command above.

\subsubsection{Uni-processor P4 nodes fail to boot}

\begin{discuss}
  This has already been solved, and should be removed, right?
\end{discuss}

This problem has occured on some newer machines after the 'oscar\_wizard' 
has run and the nodes have rsync'd their files to the local harddrive. 
The nodes then restart and begin to boot the newly installed kernel
and hang with a message similar to the following,
\begin{verbatim}
       Getting VERSION: 0
       Getting VERSION: ff00ff
       enabled ExtINT on CPU#0
       ESR value before enabling vector: 00000000
       ESR value after enabling vector: 00000000
       calibrating APIC timer ...
       ..... CPU clock speed is 1995.0407 Mhz.
       ..... host bus clock speed is 0.0000 Mhz.
       cpu: 0, clocks: 0, slice: 0
       _
\end{verbatim}
The cursur just blinks here forever.

This problem occurs because the SMP kernel is installed and the
machine needs a standard uni-processor kernel.  Most machines will boot
normally with the SMP kernel but a small number exhibit this issue.
Currently, the simplest fix is to rebuild the image with the
uni-processor kernel (e.g. \emph{kernel-2.4.2-2}) in the rpmlist 
(e.g. \emph{oscarsamples/sample.rpmlist}) and re-install the failing node.  

We are currently investigating the problem further.  If you are experiencing 
this problem, please check the OSCAR web page at
{\tt http://oscar.sourceforge.net/} for the latest information and solutions
to this problem.

%%%A workable fix for this would be to do the following steps to
%%%install the uni-processor kernel (\emph{kernel-2.4.2-2.i386.rpm}) into 
%%%an existing image named \emph{oscarimage}.  After that you can repeat
%%%the installation of the nodes, using either network boot (PXE) or a
%%%boot floppy.   This will do a complete re-install of the node.
%%%\begin{verbatim}
%%%       [root@oscar /root]# cp /tftpboot/rpm/kernel-2.4.2-2.i386.rpm \
%%%       /var/lib/systemimager/images/oscarimage/tmp
%%%       [root@oscar /root]# chroot /var/lib/systemimager/oscarimage
%%%       [root@oscar /]# rpm -ivh /tmp/kernel-2.4.2-2.i386.rpm 
%%%       [root@oscar /]# rm /tmp/kernel-2.4.2-2.i386.rpm
%%%       [root@oscar /]# cd boot
%%%       [root@oscar boot]# ln -sf kernel.h-2.4.2  kernel.h
%%%       [root@oscar boot]# ln -sf module-info-2.4.2-2  module-info
%%%       [root@oscar boot]# ln -sf System.map-2.4.2-2  System.map
%%%       [root@oscar boot]# ln -sf vmlinuz-2.4.2-2  vmlinuz
%%%       [root@oscar /]# exit
%%%       [root@oscar /root]#
%%%\end{verbatim}
% extra dir listing i took out of the above.
%       [root@oscar boot]# ls *2.4.2*
%       kernel.h-2.4.2          System.map-2.4.2-2     vmlinux-2.4.2-2smp
%       module-info-2.4.2-2     System.map-2.4.2-2smp  vmlinuz-2.4.2-2
%       module-info-2.4.2-2smp  vmlinux-2.4.2-2        vmlinuz-2.4.2-2smp
%Now, you need backup and edit the \emph{oscarimage.master} script so
%that it skips the partitioning of the disk and just goes onto the
%rsync'ing of the files.  NOTE: THIS IS THE SECTION THAT I NEED TO 
%FINISH UP B/C I'VE NOT TESTED IT YET ... THEREFORE I'M NOT GOING TO
%INCLUDE IT HERE.


\subsection{What to do about unknown problems?}

For help in solving problems not covered by this HowTo, send a
detailed message describing the problem to the OSCAR users mailing
list at \email{oscar-users@lists.sourceforge.net}. You may also wish
to visit the OSCAR web site, \url{http://oscar.sourceforge.net/}, for
updates on newly found and resolved problems.

\subsection{Starting over -- installing OSCAR again}

\begin{discuss}
  The \cmd{start\_over} script isn't interactive, is it?
\end{discuss}

If you feel that you want to start the cluster installation process
over from scratch in order to recover from irresolvable errors, you
can do so with the \cmd{start\_over} script located in the
\file{scripts} subdirectory. This script is interactive, and will
prompt you when removing components installed by OSCAR that you may
not want to remove.

\begchange

It is important to note that \cmd{start\_over} is {\em not} an
uninstaller.  That is, \cmd{start\_over} does {\em not} guarantee to
return the head node to the start that it was in before OSCAR was
installed.  It does a ``best attempt'' to do so, but the only
guarantee that it provides is that the head node will be suitable for
OSCAR re-installation.  For example, the RedHat 7.x series ships with
a LAM/MPI RPM.  The OSCAR install process removes this RedHat-default
RPM and installs a custom OSCAR-ized LAM/MPI RPM.  The
\cmd{start\_over} script only removes the OSCAR-ized LAM/MPI RPM -- it
does not re-install the RedHat-default LAM/MPI RPI.

Another important fact to note is that because of the environment
manipulation that was performed via \cmd{switcher} from the previous
OSCAR install, {\em it is necessary to re-install OSCAR from a shell
  that was not tainted by the previous OSCAR installation}.
Specifically, the \cmd{start\_over} script can remove most files and
packages that were installed by OSCAR, but it cannot chase down and
patch up any currently-running user environments that were tainted by
the OSCAR environment manipulation packages.

Ensuring to have an untainted environment can be done in one of two
ways:

\begin{enumerate}
\item After running \cmd{start\_over}, completely logout and log back
  in again before re-installing.  {\em Simply launching a new shell
    may not be sufficient} (e.g., if the parent environment was
  tainted by the previous OSCAR install).  This will completely erase
  the previous OSCAR installation's effect on the environment in all
  user shells, and establish a set of new, untainted user
  environments.
  
\item Use a shell that was established {\em before} the previous OSCAR
  installation was established.  Although perhaps not entirely
  intuitive, this may include the shell was that initially used to
  install the previous OSCAR installation.
\end{enumerate}

\endchange

% LocalWords:  Exp
