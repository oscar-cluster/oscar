% -*- latex -*-
%
% $Id: sis.tex,v 1.1 2001/12/13 22:14:12 mchasal Exp $
%
% $COPYRIGHT$
%

\section{Overview of SIS}

The first question you may have is what is SIS. The System Installation
Suite (SIS) is a cluster installation tool developed by the collaboration
of the IBM Linux Technology Center and the SystemImager team.
The main reason that SIS was chosen as the
installation mechanism was that it does not require that client nodes
already have Linux installed.
SIS also has many other distinguishing features that make it the mechanism
of choice. The most highly used quality of SIS in the OSCAR install is
the cluster information database that it maintains. The database
contains all the information on each node needed to both install and
configure the cluster. A second desirable quality is that SIS makes
use of the Red Hat Package Manager (RPM) standard for software
installation, which simplifies software installation
tremendously. Another quality, which up to this point has not been
taken advantage of in OSCAR, is the heterogeneous nature of SIS,
allowing clients to contain not only heterogeneous hardware, but
heterogeneous software as well. An obvious application of the future
use of this quality is in the installation of heterogeneous clusters,
allowing certain clients to be used for specialized purposes.

In order to understand some of the steps in the upcoming install, you
will need knowledge of the main concepts used within SIS. The first
concept is that of an image. In SIS, an image is defined for use by
the cluster nodes. This image is a copy of the operating system files 
stored on the server. The client nodes install by replicating this image
to their local disk partitions. Another important concept from SIS it the
client definition.  A SIS client is  defined for each of your cluster nodes. 
These client definitions keep track of the pertinent information about each
client.
The server machine is responsible for creating the cluster
information database and for servicing client installation requests.
The information that is stored for  each client includes;

\begin{itemize}
        \item IP information like hostname, IPaddress, route.
        \item Image name.
\end{itemize}

Each of these pieces of information will be discussed further as part of the
detailed install procedure.

For additional information on the concepts in SIS and how to use it,
you should refer to the SIS manpage. In addition, you can visit the SIS web site at
\url{http://sisuite.org} for recent updates.

