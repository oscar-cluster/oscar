% -*- latex -*-
%
% $Id: outline.tex,v 1.8 2002/09/11 07:18:56 jsquyres Exp $
%
% $COPYRIGHT$
%

\section{Outline of Cluster Installation Procedure}
\label{sec:outline}

The following outline is a general description of how an OSCAR cluster
is installed.  Users do not manually perform many of the steps in this
outline -- they are automatically executed as part of the OSCAR
installation process.

\begin{enumerate}
\item Server Installation and Configuration
  \begin{enumerate}
  \item install Linux on the server node
  \item get an OSCAR distribution package
  \item configure the ethernet adapter for cluster
  \item copy Linux distribution RPMs from CDs to \file{/tftpboot/rpm}
  \item get updated distribution-specific RPMs
  \item copy updated RPMs to \file{/tftpboot/rpm}
  \end{enumerate}
  
\item Initial OSCAR Server Configuration
  \begin{enumerate}
  \item create OSCAR directories
  \item install OSCAR-specific software
    \begin{itemize}
    \item C3
    \item LAM/MPI
    \item MPICH
    \item OPIUM
    \item OpenPBS
    \item Maui
    \item pfilter
    \item PVM
    \item SIS
    \item Switcher
    \item Various services:
      \begin{enumerate}
      \item Network File System (NFS)
      \item Dynamic Host Configuration Protocol (DHCP)
      \item rsync
      \item OpenSSH
      \end{enumerate}
    \end{itemize}
  \item update some system files
  \item update system startup scripts
  \item start/restart affected services
  \end{enumerate}
  
\item Cluster Definition
  \begin{enumerate}
  \item build and customize image
    \begin{enumerate}
    \item install Linux distribution RPMs
    \item install OSCAR RPMs
    \item customize and copy system and user files
    \item setup NFS mount for \file{/home}
    \item generate SSH keys
    \end{enumerate}
  \item define disk partitioning and filesystems
  \item define clients
    \begin{enumerate}
    \item update hosts files on server and in images
    \end{enumerate}
  \item collect client MAC addresses
  \item setup remote booting (network or diskette)
  \end{enumerate}
  
\item Client Installations
  \begin{enumerate}
  \item boot clients to start install
  \item reboot each client when finished installing
  \end{enumerate}
\item Complete cluster setup
\item Test cluster
  \begin{enumerate}
  \item test PBS
  \item test OpenSSH
  \item test MPICH
  \item test LAM/MPI
  \item test PVM
  \item test HDF5
  \end{enumerate}
\item \emph{Cluster setup complete!}
\end{enumerate}

% LocalWords:  Exp
