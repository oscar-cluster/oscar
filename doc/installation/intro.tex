% -*- latex -*-
%
% $Id: intro.tex,v 1.1 2001/12/13 22:14:12 mchasal Exp $
%
% $COPYRIGHT$
%

\section{Introduction}

The OSCAR cluster installation HowTo is provided as an installation
guide to users, as well as a detailed explanation of what is happening
as you install. This document does not describe what OSCAR is however.
For an overview of OSCAR and the intentions behind it see the
\file{oscar\_introduction} document located in the docs subdirectory.
A list of software and hardware requirements for OSCAR can be found in
the \file{oscar\_requirements} document as well. 
\begchange
There are both ``Quick Start'' and detailed installation sections in
this guide. Due to the complicated nature of putting together a 
high-performance cluster, it is strongly suggested that you read this 
document through, without skipping any sections, and then use the
detailed installation procedure to install your OSCAR cluster.
Novice users will be comforted to know that anyone who has installed
and used Linux can successfully navigate through the OSCAR cluster install.
If you are too impatient to read manuals, and your cluster meets
certain requirements, feel free to try your luck with the ``Quick Start''
installation.
\endchange

Let's start with a few basic terms and concepts, so that everyone is
starting out on the same level. The most important is the term
\term{cluster}, which when mentioned herein refers to a group of
individual computers bundled together using hardware and software in
order to make them work as a single machine. Each individual machine
of a cluster is referred to as a \term{node}. Within the OSCAR cluster
to be installed, there are two types of nodes, \term{server} and
\term{client}. A \term{server} node is responsible for servicing the
requests of \term{client} nodes.  A \term{client} node is dedicated to
computation.  The OSCAR cluster to be installed will consist of one
server node and a number of client nodes, where all the client nodes
have homogeneous hardware.  The software contained within OSCAR does
support doing multiple cluster installs from the same server, but no
documentation is provided on how to do so. In addition, OSCAR does not
support installation of additional client nodes after the initial
cluster installation is performed, although this functionality is
planned for later releases.

The rest of this document is organized as follows.  \begchange First,
a ``Quick Start''section is provided for people who hate manuals.  The
``Quick Start'' section is {\bf not} for the Linux or clustering
novice. Then, an overview is given for the installation software used
in OSCAR, known as SIS.
%
\begchange
%
Third, 
%
\endchange
%
an outline is given of the entire cluster install procedure, so that
users have a general understanding of what they will be doing.  Next,
the cluster installation procedure is presented in much detail.  The
level of detail lies somewhere between ``the install will now update
some files'' and ``the install will now replace the string `xyz' with
`abc' in file \file{some\_file}.'' The reasoning behind providing this
level of detail is to allow users to fully understand what it takes to
put together a cluster, as well as to allow them to help troubleshoot
any problems, should some arise. Last, but certainly not least, are
the appendices.  Appendix~\ref{app:net-boot-client-nodes} covers the
topic of network booting client nodes, which is so important that it
deserved its own section.  Appendix~\ref{app:ramdisk} gives
instructions for creating initial ramdisks.
Appendix~\ref{app:client-install} provides curious users an overview
of what really happens during a client install.
%
\begchange
%
Appendix~\ref{app:troubleshooting} tackles troubleshooting, providing
fixes to known problems and where to find help for unknown problems,
as well as telling how to start over with a new OSCAR install.  And
finally, Appendix~\ref{app:security} covers some security aspects of a
linux cluster.
%
\endchange
