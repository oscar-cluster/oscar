% -*- latex -*-
%
% Copyright (c) 2002 The Trustees of Indiana University.
%                    All rights reserved.
%
% This file is part of the OSCAR software package.  For license
% information, see the COPYING file in the top level directory of the
% OSCAR source distribution.
%
% $Id: intro.tex,v 1.21 2003/02/24 23:30:41 naughtont Exp $
%
% $COPYRIGHT$
%

\section{Introduction}
OSCAR version \oscarversion\ is a snapshot of the best known methods
for building, programming, and using clusters. It consists of a
fully integrated and easy to install software bundle designed for
high performance cluster computing (HPC). Everything needed to
install, build, maintain, and use a modest sized Linux cluster is
included in the suite, making it unnecessary to download or even
install any individual software packages on your cluster.

OSCAR is the first project by the Open Cluster Group. For more
information on the group and its projects, visit its website
\url{http://www.OpenClusterGroup.org/}.

This document provides a step-by-step installation guide for system
administrators, as well as a detailed explanation of what is
happening as you install.  Note that this installation guide is
specific to OSCAR version \oscarversion.

%%%%%%%%%%%%%%%%%%%%%%%%%%%%%%%%%%%%%%%%%%%%%%%%%%%%%%%%%%%%%%%%%%%%%%%%%%

\subsection{Latest Documentation}

Please be sure that you have the latest version of this document.  It
is possible (and probable!) that newer versions of this document were
released on the main OSCAR web site after the software was released.
You are {\em strongly} encouraged to check
\url{http://oscar.sourceforge.net/} for the latest version of these
instructions before proceeding.  Document versions can be compared by
checking their version number and date on the cover page.

%%%%%%%%%%%%%%%%%%%%%%%%%%%%%%%%%%%%%%%%%%%%%%%%%%%%%%%%%%%%%%%%%%%%%%%%%%

\subsection{Terminology}

A common term used in this document is \term{cluster}, which refers to
a group of individual computers bundled together using hardware and
software in order to make them work as a single machine.

Each individual machine of a cluster is referred to as a \term{node}.
Within the OSCAR cluster to be installed, there are two types of
nodes: \term{server} and \term{client}. A \term{server} node is
responsible for servicing the requests of \term{client} nodes.  A
\term{client} node is dedicated to computation.

An OSCAR cluster consists of one server node and one or more client
nodes, where all the client nodes [currently] must have homogeneous
hardware.  The software contained within OSCAR does support doing
multiple cluster installs from the same server, but that process is
outside the scope of this guide.

An \term{OSCAR package} is a set of files that is used to install a
software package in an OSCAR cluster.  An OSCAR package can be as
simple as a single RPM file, or it can be more complex, perhaps
including a mixture of RPM and other auxiliary configuration /
installation files.  OSCAR packages provide the majority of
functionality in OSCAR clusters.

OSCAR packages fall into one of three categories:

\begin{itemize}
\item \term{Core packages} are required for the operation of OSCAR
  itself (mostly involved with the installer).

\item \term{Included packages} are shipped in the official OSCAR
  distribution.  These are usually authored and/or packaged by OSCAR
  developers, and have some degree of official testing before
  release.

\item \term{Third party packages} are not included in the official
  OSCAR distribution; they are ``add-ons'' that can be unpacked in the
  OSCAR tree, and therefore installed using the OSCAR installation
  framework.
\end{itemize}

%%%%%%%%%%%%%%%%%%%%%%%%%%%%%%%%%%%%%%%%%%%%%%%%%%%%%%%%%%%%%%%%%%%%%%%%%%

\subsection{Supported Distributions}

OSCAR has been tested to work with several distributions.
Table~\ref{tab:oscar-distro-support} lists each distribution and
version and specifies the level of support for each. In order to
ensure a successful installation, most users should stick to a
distribution that is listed as \emph{Fully supported}.

% -*- latex -*-
%
% $Id: supported.tex,v 1.6 2003/08/13 22:11:39 naughtont Exp $
%
% $COPYRIGHT$
%

\begin{table}[htbp]
  \begin{center}
    \begin{tabular}{|l|c|p{3in}|}
      \hline
      \multicolumn{1}{|c|}{Distribution and Release} &
      \multicolumn{1}{|c|}{Architecture} &
      \multicolumn{1}{|c|}{Status} \\
      \hline
      \hline
      Red Hat 9 & x86 &Fully supported \\
%
      Red Hat Enterprise Linux 3 & x86 & Fully supported \\
%
      Red Hat Enterprise Linux 3 & ia64 & Fully supported \\
%
\hline
%
      Fedora Core 2 & x86 & Fully supported \\
%
\hline
%
      Mandrakelinux 10.0 & x86 & Fully supported \\
%
\hline
%
    \end{tabular}
    \caption{OSCAR supported distributions}
    \label{tab:oscar-distro-support}
  \end{center}
\end{table}


%%%%%%%%%%%%%%%%%%%%%%%%%%%%%%%%%%%%%%%%%%%%%%%%%%%%%%%%%%%%%%%%%%%%%%%%%%

\subsection{Minimum System Requirements}
\label{sec:intro-min-sys}


The following is a list of minimum system requirements for the OSCAR
server node:

\begin{itemize}
\item CPU of i586 or above
\item A network interface card that supports a TCP/IP stack
\item If your OSCAR server node is going to be the router between a
  public network and the cluster nodes, you will need a second
  network interface card that supports a TCP/IP stack
\item At least 4GB total free space -- 2GB under \file{/} and 2GB
  under \file{/var}
\item An installed version of Linux, preferably a {\em Fully
    supported} distribution from Table~\ref{tab:oscar-distro-support}
\end{itemize}

\noindent The following is a list of minimum system requirements for
the OSCAR client nodes:

\begin{itemize}
\item CPU of i586 or above
\item A disk on each client node, at least 2GB in size (OSCAR will
  format the disks during the installation)
\item A network interface card that supports a TCP/IP
  stack\footnote{Beware of certain models of 3COM cards -- not all
    models of 3COM cards are supported by the installation Linux
    kernel that is shipped with OSCAR.  See the OSCAR web site for
    more information. \label{foot:3com-warning}}
\item Same Linux distribution and version as the server node
\item All clients must have the same architecture (e.g., ia32 vs.\
  ia64)
\item Monitors and keyboards may be helpful, but are not required
\item Floppy or PXE enabled BIOS
\end{itemize}


%%%%%%%%%%%%%%%%%%%%%%%%%%%%%%%%%%%%%%%%%%%%%%%%%%%%%%%%%%%%%%%%%%%%%%%%%%

\subsection{Document Organization}

Due to the complicated nature of putting together a high-performance
cluster, it is strongly suggested that even experienced administrators
read this document through, without skipping any sections, and then
use the detailed installation procedure to install your OSCAR cluster.
Novice users will be comforted to know that anyone who has installed
and used Linux can successfully navigate through the OSCAR cluster
install.

The rest of this document is organized as follows.
%
First, Section~\ref{sec:download} tells how to obtain an OSCAR version
\oscarversion\ distribution package.
%
Next, the ``Release Notes'' section (Section~\ref{sec:release-notes})
that applies to OSCAR version \oscarversion\ contains some
requirements and update issues that need to be resolved before the
install.
%
\detailed{Section~\ref{sec:sis} provides an overview for the System
  Installation Suite software package used in OSCAR to perform the
  bulk of the cluster installation.}
%
Section~\ref{sec:detail} details the cluster installation procedure
(the level of detail lies somewhere between ``the install will now
update some files'' and ``the install will now replace the string
`xyz' with `abc' in file \file{some\_file}.'')
%
\detailed{

  Finally, Section~\ref{sec:pkg-specific-notes} contains system
  administration notes about several of the individual packages that
  are installed by OSCAR.  {\bf\em This section is a ``must read'' for
    all OSCAR system administrators.}

  Appendix~\ref{app:net-boot-client-nodes} covers the topic of network
  booting client nodes, which is so important that it deserved its own
  section.
%
  Appendix~\ref{app:client-install} provides curious users an overview
  of what really happens during a client install.
%
  Appendix~\ref{app:no_dhcp} discusses how to install an OSCAR cluster
  without a DHCP server.
%
  Appendix~\ref{app:security} covers a primer of some security aspects
  of a Linux cluster.  Although not intended to be a comprehensive
  description of cluster security, it is a good overview for those who
  know relatively little about system administration and security.
%
  Finally, Appendix~\ref{app:screen-by-screen} is a screen-by-screen
  walk through of a typical OSCAR installation.

}

More information is available on the OSCAR web site and archives of
the various OSCAR mailing lists.  If you have a question that cannot
be answered by this document (including answers to common installation
problems), be sure to visit the OSCAR web site:

\vspace{11pt}
\centerline{\url{http://oscar.sourceforge.net/}}

% LocalWords:  Exp
