% -*- latex -*-
%
% $Id: intro.tex,v 1.15 2002/07/22 04:26:10 jsquyres Exp $
%
% $COPYRIGHT$
%

\section{Introduction}

The OSCAR (Open Source Cluster Application Resource) software package
is intended to simplify the complex tasks required to install a
cluster.  While the usual intended use for OSCAR clusters is for
high-performance computing (HPC), OSCAR clusters can be used for any
cluster-enabled kinds of applications.  Note that since OSCAR is aimed
towards HPC, several HPC-related packages are installed by default,
such as popular MPI implementations, PVM, PBS, etc.

This document provides a step-by-step installation guide for system
administrators, as well as a detailed explanation of what is happening
as you install.  Note that this installation guide is specific to
OSCAR version \oscarversion.  

%%%%%%%%%%%%%%%%%%%%%%%%%%%%%%%%%%%%%%%%%%%%%%%%%%%%%%%%%%%%%%%%%%%%%%%%%%

\subsection{Terminology}

Let's start with a few basic terms and concepts, so that everyone is
starting out on the same level. The most important is the term
\term{cluster}, which when mentioned herein refers to a group of
individual computers bundled together using hardware and software in
order to make them work as a single machine. 

Each individual machine of a cluster is referred to as a \term{node}.
Within the OSCAR cluster to be installed, there are two types of
nodes: \term{server} and \term{client}. A \term{server} node is
responsible for servicing the requests of \term{client} nodes.  A
\term{client} node is dedicated to computation.  

An OSCAR cluster consists of one server node and one or more client
nodes, where all the client nodes [currently] must have homogeneous
hardware.  The software contained within OSCAR does support doing
multiple cluster installs from the same server, but no documentation
is provided on how to do so.

%%%%%%%%%%%%%%%%%%%%%%%%%%%%%%%%%%%%%%%%%%%%%%%%%%%%%%%%%%%%%%%%%%%%%%%%%%

\subsection{Supported Distributions}

OSCAR has been tested to work with several distributions.
Table~\ref{tab:oscar-distro-support} lists each distribution and
version and specifies the level of support for each. In order to
ensure a successful installation, most users should stick to a
distribution that is listed as \emph{Fully supported}.

% -*- latex -*-
%
% $Id: supported.tex,v 1.6 2003/08/13 22:11:39 naughtont Exp $
%
% $COPYRIGHT$
%

\begin{table}[htbp]
  \begin{center}
    \begin{tabular}{|l|c|p{3in}|}
      \hline
      \multicolumn{1}{|c|}{Distribution and Release} &
      \multicolumn{1}{|c|}{Architecture} &
      \multicolumn{1}{|c|}{Status} \\
      \hline
      \hline
      Red Hat 9 & x86 &Fully supported \\
%
      Red Hat Enterprise Linux 3 & x86 & Fully supported \\
%
      Red Hat Enterprise Linux 3 & ia64 & Fully supported \\
%
\hline
%
      Fedora Core 2 & x86 & Fully supported \\
%
\hline
%
      Mandrakelinux 10.0 & x86 & Fully supported \\
%
\hline
%
    \end{tabular}
    \caption{OSCAR supported distributions}
    \label{tab:oscar-distro-support}
  \end{center}
\end{table}


%%%%%%%%%%%%%%%%%%%%%%%%%%%%%%%%%%%%%%%%%%%%%%%%%%%%%%%%%%%%%%%%%%%%%%%%%%

\subsection{Minimum System Requirements}
\label{sec:intro-min-sys}

\begchange

The following is a list of minimum system requirements for the OSCAR
server node:

\begin{itemize}
\item CPU of i586 or above
\item A network interface card that supports a TCP/IP stack
\item If your OSCAR server node is going to be the router between a
  public network and the cluster nodes, you will need a second
  network interface card that supports a TCP/IP stack
\item At least 4GB total free space -- 2GB under \file{/} and 2GB
  under \file{/var}
\item An installed version of Linux, preferably a {\em Fully
    supported} distribution from Table~\ref{tab:oscar-distro-support}
\end{itemize}

The following is a list of minimum system requirements for the OSCAR
client nodes:

\begin{itemize}
\item CPU of i586 or above
\item A disk on each client node, at least 2GB in size (OSCAR will
  format the disks during the installation)
\item A network interface card that supports a TCP/IP
  stack\footnote{Beware of 3COM cards.  See the OSCAR web site for
  more information.}
\item Same Linux distribution and version as the server node
\item All clients must have the same architecture (ia32 vs.\ ia64)
\item Monitors and keyboards may be helpful, but are not required
\end{itemize}

\endchange

%%%%%%%%%%%%%%%%%%%%%%%%%%%%%%%%%%%%%%%%%%%%%%%%%%%%%%%%%%%%%%%%%%%%%%%%%%

\subsection{Document Organization}

There are both ``Quick Start'' and detailed installation sections in
this guide. Due to the complicated nature of putting together a
high-performance cluster, it is strongly suggested that you read this
document through, without skipping any sections, and then use the
detailed installation procedure to install your OSCAR cluster.  Novice
users will be comforted to know that anyone who has installed and used
Linux can successfully navigate through the OSCAR cluster install.  If
you are too impatient to read manuals, and your cluster meets certain
requirements, feel free to try your luck with the ``Quick Start''
installation.

The rest of this document is organized as follows.  
%
First, Section~\ref{sec:download} tells how to obtain an OSCAR version
\oscarversion\ distribution package.
%
Next, a ``Release Notes'' section (Section~\ref{sec:release-notes})
that applies to OSCAR version \oscarversion\ and contains some
requirements and update issues that need to be resolved before the
install.
%
Section~\ref{sec:quick-start} provides a ``Quick Start'' installation
guide, which gives a brief overview of the entire installation
process.  It is intended for experienced users who do not need the
step-by-step instructions for a complete install.  The ``Quick Start''
section is {\bf not} for the Linux or clustering novice.
%
Section~\ref{sec:sis} provides an overview is given for the
installation software used in OSCAR, known as SIS.  
%
An outline is provided in Section~\ref{sec:outline} of the entire
cluster install procedure, so that users have a general understanding
of what they will be doing.
%
Finally, Section~\ref{sec:detail} details the cluster installation
procedure.  The level of detail lies somewhere between ``the install
will now update some files'' and ``the install will now replace the
string `xyz' with `abc' in file \file{some\_file}.'' The reasoning
behind providing this level of detail is to allow users to fully
understand what it takes to put together a cluster, as well as to
allow them to help troubleshoot any problems, should any arise.

Appendix~\ref{app:net-boot-client-nodes} covers the topic of network
booting client nodes, which is so important that it deserved its own
section.  
%
Appendix~\ref{app:client-install} provides curious users an overview
of what really happens during a client install.
%
Appendix~\ref{app:security} covers a primer of some security aspects
of a Linux cluster.  Although not intended to be a comprehensive
description of cluster security, it is a good overview for those who
know relatively little about system administration and security.
%
Finally, Appendix~\ref{app:screen-by-screen} is a screen-by-screen
walk through of a typical OSCAR installation.

More information is available on the OSCAR web site and archives of
the various OSCAR mailing lists.  If you have a question that cannot
be answered by this document (including answers to common installation
problems), be sure to visit the OSCAR web site:

\vspace{11pt}
\centerline{\url{http://oscar.sourceforge.net/}}

% LocalWords:  Exp
