% -*- latex -*-
%
% $Id: client-install.tex,v 1.1 2001/12/13 22:14:12 mchasal Exp $
%
% $COPYRIGHT$
%

\section{What Happens During Client Installation}
\label{app:client-install}

Once the client is network booted, it either boots off the autoinstall
diskette you created or uses PXE to network boot and loads the install 
kernel. It then broadcasts a BOOTP/DHCP request
to obtain the IP address associated with its MAC address. The DHCP
server provides the IP information and the client looks for its auto-install
script in \file{/var/lib/systemimager/scripts/}. The script is named <nodename>.sh and
is a symbolic link to the script for the desired image.
The auto-install script is the installation workhorse, and does the
following:

\begin{enumerate}
\item partitions the disk as specified in the image in 
        \file{<imagedir>etc/systemimager/partitionschemes}.

\item mounts the newly created partitions on \file{/a}
  
\item chroots to \file{/a} and uses rsync to bring over all 
        the files in the image.

\item calls systemconfigurator to customize the image to the client's
        particular hardware and configurate.

\item unmounts \file{/a}
\end{enumerate}

Once clone completes, the client will either reboot, halt or beep as 
specified when defining the image.

