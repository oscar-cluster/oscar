% -*- latex -*-
%
% $Id: release-notes.tex,v 1.23 2002/09/11 07:20:19 jsquyres Exp $
%
% $COPYRIGHT$
%

\section{Release Notes}
\label{sec:release-notes}

The following release notes apply to OSCAR version \oscarversion:

\begchange

\begin{itemize}
\item All nodes must have a hostname other than ``{\tt localhost}''
  that does not contain any underscores (``{\tt \_}'').
  
\item A domain name must be specified for the client nodes when
  defining them.
  
\item The current version of C3 that is shipped with OSCAR requires
  Python 2.x.  The Python 2.x RPMs that are provided in OSCAR are
  re-built to work on RedHat 7.1 as well as RedHat 7.2.  However,
  RedHat 7.2 ships with Python 2.x, so measures were taken such that
  the OSCAR-shipped Python 2.x RPM will not be installed over the
  default RedHat 7.x RPM (specifically, the RPM release number has
  been decremented).  Due to some distribution portability issues,
  OSCAR currently installs a ``compatibility''
  (\file{python2-compat-1.0-1}) RPM to resolve the Python2
  prerequisite that is slightly different across different Linux
  distributions.  Also see the file
  \file{packages/c3/RPMS/NOTE.python2}.

\item In some cases, the test window that is opened from the 
  OSCAR wizard may close suddenly when there is a test failure. If
  this happens, run the test script, \file{testing/test\_cluster\_as\_root},
  manually in a shell window to diagnose the problem.
\end{itemize}

\endchange

%%%%%%%%%%%%%%%%%%%%%%%%%%%%%%%%%%%%%%%%%%%%%%%%%%%%%%%%%%%%%%%%%%%%%%%%%%

\subsection{Mandrake 8.2 Notes}
\label{subsec:mdk82notes}

\begin{itemize}
\item This is the first release of OSCAR to support this distribution.
  OSCAR \oscarversion\- has been tested and successfully installed on
  this distribution (see Table~\ref{tab:oscar-distro-support}); prior
  versions of OSCAR will not work on any flavors of Mandrake without
  significant modifications.
  
\item If installing the third-party package Ganglia, the
  \rpmname{libpng} RPM version 1.0.8-2mdk must be manually installed
  before invoking the OSCAR install wizard.  This RPM can be
  found in the Ganglia distribution package in the following location:\\
  \file{packages/ganglia/extras/libpng-1.0.8-2mdk.i586.rpm}
  
% TJN: relates to server_prep: unmunge_pathenv()
\item \user{root}'s default shell configuration files hardcode the
  value for {\tt PATH} environment variable, regardless of what is
  added via {\tt profile.d/} startup scripts.  This effects various
  OSCAR installed components, which are installed into locations such
  as \file{/opt}.  As such, the OSCAR installer comments out these
  lines in following files:

  \begin{itemize}
  \item \file{/root/.bashrc}
  \item \file{/root/.cshrc}
  \item \file{/root/.tcshrc}
  \end{itemize}

\end{itemize}

%%%%%%%%%%%%%%%%%%%%%%%%%%%%%%%%%%%%%%%%%%%%%%%%%%%%%%%%%%%%%%%%%%%%%%%%%%

\subsection{IA64 and Other Bleeding Edge Systems Notes}
\label{subsec:ia64notes}

\begin{itemize}
  
\item Itanium is supported on RedHat 7.2.  Itanium 2 is not yet
  supported.\footnote{Preliminary indications are that Itanium 2
    support is possible, even if it is not formally supported.  The
    main issues for OSCAR are that at the time of this writing,
    Itanium 2 hardware post-dates the available binary versions of
    operating systems, and SIS does not properly pick up modules for
    the netbooted kernel.  The solution is for the user to compile
    their own kernel.  This is essentially a ``do it yourself'' kind
    of process, and not recomended for novices.  Please consult the
    OSCAR user's mailing list for more information.}

\item The OSCAR autoinstall diskette will probably not work on Itanium
  systems, but they mostly have support for network booting.
  
\item Some Itanium 1 or 2 systems are supported by OSCAR under
  RedHat's Linux 7.2 release for Itanium.  Itanium systems that can
  use the standard RedHat Itanium kernel for booting and network
  access should work with OSCAR.
  
\item If an Itanium (or an IA32 based) system has new or unknown
  hardware that is not supported by the stock kernel in the base
  operating system release, it will not work with standard OSCAR,
  since OSCAR uses the kernel RPMS supplied with the vendors release
  to boot and load the compute nodes.  If you have a custom compiled
  kernel/module set that works for your hardware, you should look at
  the NCSA utilities at \url{http://oscar.ncsa.uiuc.edu/} to
  enable you to use them in OSCAR.
\end{itemize}

% LocalWords:  tex Exp
