% -*- latex -*-
%
% Copyright (c) 2002 The Trustees of Indiana University.
%                    All rights reserved.
%
% This file is part of the OSCAR software package.  For license
% information, see the COPYING file in the top level directory of the
% OSCAR source distribution.
%
% $Id: release-notes.tex,v 1.44 2003/11/26 23:53:35 bernardli Exp $
%

\section{Release Notes}
\label{sec:release-notes}

The following release notes apply to OSCAR version \oscarversion.
%%%%%%%%%%%%%%%%%%%%%%%%%%%%%%%%%%%%%%%%%%%%%%%%%%%%%%%%%%%%%%%%%%%%%%%%%%
%%%%%%%%%%%%%%%%%%%%%%%%%%%%%%%%%%%%%%%%%%%%%%%%%%%%%%%%%%%%%%%%%%%%%%%%%%
\subsection{Release Features}
\label{subsec:release-features}

\begin{itemize}

\item Red Hat Linux 9, Fedora Core 2 and Mandriva Linux 10.0 support on x86.
\item Red Hat Enterprise Linux (RHEL) 3 support on Itanium and x86.
\item New RPM dependency finder helps build the server (DepMan/PackMan).
\item Ganglia now included in the default package set.
\item Torque now included/OpenPBS is now an optional package.
\item Multiple bug fixes and Wizard improvements.
\item Updated user interface (updated/improved wizard).

\end{itemize}
%%%%%%%%%%%%%%%%%%%%%%%%%%%%%%%%%%%%%%%%%%%%%%%%%%%%%%%%%%%%%%%%%%%%%%%%%%
%%%%%%%%%%%%%%%%%%%%%%%%%%%%%%%%%%%%%%%%%%%%%%%%%%%%%%%%%%%%%%%%%%%%%%%%%%

\subsection{Notes for All Systems}
\label{subsec:release-notes}

\begin{itemize}

\item There is currently a known issue with the Delete Nodes functionality.
  You may notice error messages like the following when you delete the
  last node of a cluster:

  \begin{verbatim}
    /opt/oscar/packages/sis/scripts/post_clients: illegal or
    non-understood where string <nics.driver_module = >
    /opt/oscar/packages/sis/scripts/post_clients: INTERNAL
    ODA ERROR - failing to convert where expression
    <nics.driver_module = > in oda::delete_records - at
    least one record not deleted in table <nics> in database <oscar>
    /opt/oscar/packages/sis/scripts/post_clients: illegal or
    non-understood where string <nics.gateway = >
    /opt/oscar/packages/sis/scripts/post_clients: INTERNAL
    ODA ERROR - failing to convert where expression
    <nics.gateway = > in oda::delete_records - at least one
    record not deleted in table <nics> in database <oscar>
    /opt/oscar/packages/sis/scripts/post_clients: illegal or
    non-understood where string <nics.assignment_method = >
  \end{verbatim}

  This is not `harmful' but will leave the ODA database inconsistent
  with the SIS database.  The information on network interfaces is
  currently not used therefore it is not really damaging to your
  installation.  This will be fixed in the next release.

\item In Step 8 \button{Test Cluster Setup}, some tests may fail for the first
  time but subsequent re-runs of this step will indicate that all the
  tests succeeded.  This is a known issue and will be fixed for the next
  release.

\item Each package in OSCAR has its own installation and release
  notes.  \detailed{See Section~\ref{sec:pkg-specific-notes} for
  additional release notes.} \quick{See the full Installation Guide
  for these notes.}

\begchange
\item All nodes must have a hostname other than ``{\tt localhost}''
  that does not contain any underscores (``{\tt \_}'') or periods ``{\tt
  .}''.  Some distributions complicate this by putting a line such
  as as the following in /etc/hosts:
  \begin{verbatim}
    127.0.0.1   localhost.localdomain   localhost
    yourhostname.yourdomain    yourhostname
  \end{verbatim}
  If this occurs the file should be separated as follows:
  \begin{verbatim}
  127.0.0.1 localhost.localdomain localhost
  192.168.0.1 yourhostname.yourdomain yourhostname
  \end{verbatim}
\endchange

\item A domain name must be specified for the client nodes when
  defining them.

%TJN: KEEP THIS (if we keep the python2-compat rpms under c3?)
\item Due to some distribution portability issues, OSCAR currently installs
  a ``compatibility''  (\file{python2\--compat\--1.0-1}) RPM to resolve the
  Python2 prerequisite that is slightly different across different Linux
  distributions.  Also see the file \file{packages/c3/RPMS/NOTE.python2}.

\item In some cases, the test window that is opened from the
  OSCAR wizard may close suddenly when there is a test failure. If
  this happens, run the test script, \file{testing/test\_cluster},
  manually in a shell window to diagnose the problem.

\item Although OSCAR can be installed on pre-existing server nodes, it
  is typically easiest to use a machine that has a new, fresh install
  of a distribution listed in Table~\ref{tab:oscar-distro-support}
  {\em with no updates installed}.  If the updates are installed,
  there may be conflicts in RPM requirements.  It is recommended to
  install RedHat updates {\em after} the initial OSCAR installation has
  completed.

\item The following benign warning messages will appear multiple times
  during the OSCAR installation process:

\begin{verbatim}
  awk: cmd. line:2: fatal: cannot open file `/etc/fstab'
    for reading (No such file or directory)

  rsync_stub_dir: no such variable at ...

  Use of uninitialized value in pattern match (m//) at
  /usr/lib/perl5/site_perl/oda.pm ...
\end{verbatim}

  It is safe to ignore these messages.

\item The OSCAR installer will install the MySQL package on the server
  node if it is not already installed.  A random password will be automatically
\begchange
  generated for the oscar user to access the oscar database.  This
  password will be stored in the file \file{/etc/odapw}.  It should
  not be needed by other users.
\endchange

\item The OSCAR installer GUI provides little protection for user
  mistakes.  If the user executes steps out of order, or provides
  erroneous input, Bad Things may happen.  Users are strongly
  encouraged to closely follow the instructions provided in this
  document.

\item The OSCAR installer GUI currently does not support deleting a
  node and adding the same node back {\em in the same session}.  If
  you wish to delete a node and then add it back, you must delete the
  node, close the OSCAR installer GUI, launch the OSCAR installer GUI
  again, and then add the node.

\item If \cmd{ssh} produces warnings when logging into the compute
  nodes from the OSCAR head node, the C3 tools (e.g., \cmd{cexec}) may
  experience difficulties.  For example, if you use \cmd{ssh} to login
  in to the OSCAR head node from a terminal that does not support X
  windows and then try to run \cmd{cexec}, you might see a warning
  message in the \cmd{cexec} output:

\begin{verbatim}
  Warning: No xauth data; using fake authentication data for
  X11 forwarding.
\end{verbatim}

  Although this is only a warning message from \cmd{ssh}, \cmd{cexec}
  may interpret it as a fatal error, and not run across all cluster
  nodes properly (e.g., the \button{Install/Uninstall Packages} button
  will likely not work properly).

  Note that this is actually an \cmd{ssh} problem, not a C3 problem.
  As such, you need to eliminate any warning messages from ssh (more
  specifically, eliminate any output from \file{stderr}).  In the
  example above, you can tell the C3 tools to use the ``\cmd{-x}''
  switch to \cmd{ssh} in order to disable X forwarding:

\begin{verbatim}
  # export C3_RSH='ssh -x'
  # cexec uptime
\end{verbatim}

  The warnings about \cmd{xauth} should no longer appear (and the
  \button{Install/Uninstall Packages} button should work properly).

\item The \button{Cancel} button in the \button{Install/Uninstall
    Package} step does not work properly; if any packages are selected
  to be installed or uninstalled, clicking the \button{Cancel} button
  still triggers the execution of the package installer/uninstaller.
  This will be fixed in a future release.  The same behavior occurs if
  you close the window via the window manager's ``close''
  functionality.

  Note that if you do not select any additional packages to
  install/uninstall, nothing will run (as expected).

\item The SIS multicast facility (Flamethrower) is ``experimentally''
  supported.  If you are having problems with multicast and would like
  to experiment please check the {\tt oscar-users} and/or
  {\tt sisuite-users} mailing lists for tips.

\begchange
\item In the \button{Setup Networking} step, MACs can
  only be saved in /root which is the default directory.

\item The man pages for the Torque package will not be available in
  a default installation because /opt/pbs/man is missing from
  MANPATH.  They should appear if this is added by hand eg:

\begin{verbatim}
  # export MANPATH=$MANPATH:/opt/pbs/man
\end{verbatim}
\endchange


\item \emph{FutureWarning} message during APItests on Python2.3 based systems.
   The following is a warning message about the for the version of
   TwistedMatrix used by the APItest tool.  It is only a warning and can be
   ignored.
   \begin{small}
   \begin{verbatim}
   Running Installation tests for pvm
   /usr/lib/python2.3/site-packages/twisted/internet/defer.py:398:
   FutureWarning: hex()/oct() of negative int will return
   a signed string in Python 2.4 and up return "<%s at %s>" 
   % (cname, hex(id(self)))
   \end{verbatim}
   \end{small}

\item PVM Installation (APItest) tests showing {\tt FAIL}.  In some cases,
  namely the first pass through the OSCAR Wizard to install the system (from
  original shell without PVM installed) may show output similar to the
  following:
  \begin{small}
  \begin{verbatim}
     [PASS]       2005-04-08T15:57:32Z   pvmd-path-ls.apt
     [FAIL]       2005-04-08T15:57:32Z   envvar-pvm_arch.apt
     [FAIL]       2005-04-08T15:57:32Z   envvar-pvm_root.apt
     [FAIL]       2005-04-08T15:57:32Z   pvmd-path-which.apt
     [PASS]       2005-04-08T15:57:33Z   modulecmd-path-ls.apt
     [PASS]       2005-04-08T15:57:33Z   pvm-module-list.apt
     [PASS]       2005-04-08T15:57:33Z   pvm-module-show-pvm_rsh.apt
     [PASS]       2005-04-08T15:57:33Z   pvm-module-show-pvm_arch.apt
     [PASS]       2005-04-08T15:57:34Z   pvm-module-show-pvm_root.apt
  \end{verbatim}
  \end{small}

  The three tests marked as {\tt FAIL} (with all others showing {\tt PASS})
  can typically be ignored.  These three tests check PVM setting using the
  same environment as the OSCAR Wizard.  If it the Wizard is being run from
  the same shell that was used during the initial OSCAR install (i.e., the
  shell used to install PVM via the Wizard) then the PVM setting are absent
  from the environment and therefore will not appear in the child process
  (APItests).  The fact that the other tests show {\tt PASS} indicates that
  starting a new shell should update the environment and all tests should
  pass.  To confirm this:
  (1) Exit the OSCAR Wizard, (2) Start a new shell,
  (3) Start the OSCAR Wizard and (4) Re-run the Test Cluster step
  (all tests should show PASS)


\end{itemize}

%%%%%%%%%%%%%%%%%%%%%%%%%%%%%%%%%%%%%%%%%%%%%%%%%%%%%%%%%%%%%%%%%%%%%%%%%%
%%%%%%%%%%%%%%%%%%%%%%%%%%%%%%%%%%%%%%%%%%%%%%%%%%%%%%%%%%%%%%%%%%%%%%%%%%

\subsection{Red Hat Linux 9 Notes}
\label{subsec:rh90notes}

There are a few issues that may crop up when using OSCAR on Red Hat 9.
The following items highlight these issues.

\begin{itemize}

\item Deselecting Pfilter causes the image creation to fail.  This is
  due to a dependency with IPtables and when Pfilter is not selected
  the IPtables RPM is not listed in the node (image) rpmlist.   The simple
  fix is to add ``iptables'' to the Red Hat 9 rpmlist if you are not
  installing Pfilter on the compute nodes.

\item The RPM system has been updated with this Red Hat release.  The
 OSCAR install process will likely display several warnings due to unsigned
 RPMS.  These warnings can be ignored.

%TJN: These are the notes Jeff posted for the "Re: mksiimage hang problems"
%  thread on oscar-devel (2003-07-23)
\item In some OSCAR pre-release testing, RPM would hang during the
  building of a client image (Section~\ref{det:build-client-image}).
  This is a documented bug in the version of RPM that ships with
  Redhat 9; it is not a problem with OSCAR.  The procedure
  that was used to remedy this situation is outlined below (excerpts
  taken from \url{http://www.rpm.org/hintskinks/repairdb-2003-06/}):
        \begin{itemize}
        \item If RPM hangs at any point (e.g., building the client
          image) -- first ensure that it really has hung and just
          isn't taking a long, long time to complete.  Typical
          indications that it has genuinely hung include: the disk is
          not running and load goes down to 0 (or nearly 0) and stays
          there.

        \item Then do a \cmd{ps} and find the PID of the \cmd{rpm}
          process:
\begin{verbatim}
  # ps -eadf | grep rpm | grep -v grep
  ...output...
  # kill <PID_of_RPM>
\end{verbatim}

        \item This will probably not kill the process (it's likely to
          be in a state where it is ignoring signals), but it should
          be tried anyway -- this would allow \cmd{rpm} to exit
          cleanly.  If \cmd{rpm} does exit cleanly, jump down to the
          last step in this procedure.

        \item If \cmd{rpm} does not exit within a short period of
          time, \cmd{kill -9 <PID\_of\_RPM>}.  This guarantees that
          \cmd{rpm} will not exit cleanly, but in this case, it's ok.
          Now, do the following:
                        \begin{enumerate}
                        \item Save a copy of the RPM database (just to
                          be safe):
\begin{verbatim}
  # cd /var/lib
  # tar zcvf /tmp/rpmdb.tar.gz rpm
\end{verbatim}

                        \item Delete any existing RPM database locks:
\begin{verbatim}
  # cd /var/lib/rpm
  # rm -f __db*
\end{verbatim}

                        \item Rebuild the RPM database:
\begin{verbatim}
  # rpm -vv --rebuilddb
\end{verbatim}
                        \end{enumerate}

                      \item Now re-run the OSCAR step that hung.  If
                        RPM hangs again, repeat these steps to un-hang
                        it.  Testing has shown that it may be
                        necessary to repeat these steps multiple times
                        in order to get a successful RPM run.
        \end{itemize}

\end{itemize}

%%%%%%%%%%%%%%%%%%%%%%%%%%%%%%%%%%%%%%%%%%%%%%%%%%%%%%%%%%%%%%%%%%%%%%%%%%
%%%%%%%%%%%%%%%%%%%%%%%%%%%%%%%%%%%%%%%%%%%%%%%%%%%%%%%%%%%%%%%%%%%%%%%%%%

\subsection{Red Hat Enterprise Linux 3 Notes}
\label{subsec:rhel3notes}

Currently we only support Red Hat Enterprise Linux 3 Update 2 and Update
3.  Gold and Update 4 are not supported.

If you are installing a version of Red Hat Enterprise Linux 3 that does
not provide a MySQL Server RPM (eg. WS), please refer to the first item 'mysql-server'.

If you are installing on Red Hat Enterprise Linux 3 Update 2, please refer to
the second item 'Red Hat Enterprise Linux 3 (Update 2) rpmlist'.

The rest of the notes are only relevant if you are installing on ia64
(Itanium) hardware.

\begin{itemize}

\item mysql-server

\emph{BEFORE YOU BEGIN}: you will need to obtain a mysql-server RPM and put it into

  \file{/tftpboot/rpm}

The easiest way is to get the SRPM and rebuild it\begchange (i386
users should replace ia64 with i386)\endchange,

\begin{verbatim}
  rpmbuild --rebuild mysql-3.23.58-1.src.rpm
  cp /usr/src/redhat/RPMS/ia64/mysql-server-3.23.58-1.ia64.rpm /tftpboot/rpm
\end{verbatim}

MySQL v3.23.58-1 is the version that came with Red Hat Enterprise Linux 3, Update 3; if you cannot
find that particular version, please make sure that you copy all the
rebuilt MySQL RPMs (including the server RPM).  For example, if the
version you found is v3.23.58-2.3:

\begin{verbatim}
  rpmbuild --rebuild mysql-3.23.58-2.3.src.rpm
  cp /usr/src/redhat/RPMS/ia64/mysql*-3.23.58-2.3.ia64.rpm /tftpboot/rpm
\end{verbatim}

Note: The key here is to keep the MySQL versions consistent.

\item Red Hat Enterprise Linux 3 (Update 2) rpmlist

\emph{IN OSCAR WIZARD STEP 4}: if you're running Update 2, you will
need to manually select the correct rpmlist before generating the client image.
Select the appropriate file for your architecture

  \file{/opt/oscar/oscarsample/redhat-3asU2-i386.rpmlist}
  \ \\  % Force newline
or
  \ \\  % Force newline
  \file{/opt/oscar/oscarsample/redhat-3asU2-ia64.rpmlist}


\item \file{systemconfig.conf} on ia64

\emph{AFTER OSCAR WIZARD STEP 4}: you will need to add an INITRD entry to the image
file

  \file{/var/lib/systemimager/images/oscarimage/etc/systemconfig/systemconfig.conf}

After the modification, the kernel section should look like this:

\begin{verbatim}
  [KERNEL0]
    PATH = /boot/efi//EFI/redhat/vmlinuz-2.4.21-20.EL
    INITRD = /boot/efi//EFI/redhat/initrd-2.4.21-20.EL.img
    LABEL = 2.4.21-20.EL
\end{verbatim}

where 2.4.21-20.EL is the kernel for Red Hat Enterprise Linux 3 (Update 3)
you should substitute your kernel version if you're not running Update 3.

{\bf Note: Note carefully the \emph{DOUBLE SLASH} in the PATH and INITRD lines!}


\item SCSI and network on ia64

\emph{AFTER OSCAR WIZARD STEP 4}: you may also need to add a Hardware section with
SCSI and network drivers.  In the image file

  \file{/var/lib/systemimager/images/oscarimage/etc/systemconfig/systemconfig.conf}

For an Intel SR870BH2, the hardware section of this file would look like:

\begin{verbatim}
  [HARDWARE]
    ORDER = e1000 e1000 mptscsih mptbase scsi_mod
\end{verbatim}


\item USB on ia64

\emph{AFTER OSCAR WIZARD STEP 4}: if you need to use the keyboard on a USB-only
system, like the Intel SR870BH2, you need to add the USB controller to the
image file

  \file{/var/lib/systemimager/images/oscarimage/etc/modules.conf}

For example,

\begin{verbatim}
  echo alias usb-controller usb-uhci >> \
    /var/lib/systemimager/images/oscarimage/etc/modules.conf
\end{verbatim}

\end{itemize}

%%%%%%%%%%%%%%%%%%%%%%%%%%%%%%%%%%%%%%%%%%%%%%%%%%%%%%%%%%%%%%%%%%%%%%%%%%
%%%%%%%%%%%%%%%%%%%%%%%%%%%%%%%%%%%%%%%%%%%%%%%%%%%%%%%%%%%%%%%%%%%%%%%%%%

\subsection{Fedora Core 2 Notes}
\label{subsec:fc2notes}

\begin{itemize}

\item SIS currently does not fully support 2.6 kernel, the way we make it
  work right now is to run a script which generate \file{/etc/modprobe.conf}
  from \file{/etc/modules.conf} - this does not work 100\% of the time.  If
  you can deploy an image to your client nodes but are having problems
  booting, chances are this is the issue.  To get around this problem, copy a
  working \file{modprobe.conf} for your client nodes to the image directory
  \file{/var\-/lib\-/systemimager\-/images\-/oscarimage\-/etc/}.  Re-image the
  nodes and it should work.  One quick way of getting the correct
  \file{modprobe.conf} is to boot a node with the Fedora Core 2 CD 1
  and run the Rescue mode.  You will then be able to find
  \file{modprobe.conf} in \-/tmp.

\end{itemize}

%%%%%%%%%%%%%%%%%%%%%%%%%%%%%%%%%%%%%%%%%%%%%%%%%%%%%%%%%%%%%%%%%%%%%%%%%%
%%%%%%%%%%%%%%%%%%%%%%%%%%%%%%%%%%%%%%%%%%%%%%%%%%%%%%%%%%%%%%%%%%%%%%%%%%

\subsection{Mandriva Linux 10.0 Notes}
\label{subsec:mdk10notes}

\begin{itemize}

\item The version of Mandriva Linux supported is 10.0 Official which is a
3CD set.  While it may be possible to install OSCAR with 10.0 Community,
it is not officially supported.

\item During Step 3, \button{Install OSCAR Server Packages}, the RPM package
\file{kernel-enterprise} will be selected to be installed on the headnode.
As a result, this kernel will be used upon subsequent reboot.

\item While Mandriva Linux 10.0 supports 2.4 kernel, OSCAR only supports
2.6 kernel - this is explicitly specified in
\file{oscarsamples/mandrake-10.0-i386.rpmlist}.

\item The \file{tftp-server} RPM which came with Mandriva Linux 10.0 expects the
pxelinux boot files to reside in \file{/var/lib/tftpboot} but OSCAR puts
them in \file{/tftpboot}.  To get around this issue a symbolic link is
created from \file{/var/lib/tftpboot} to \file{/tftpboot}.

\item During Step 8, \button{Test Cluster Setup}, some APItests may fail due to
the OSCAR Wizard not having the correct environment variables set.  If you
encounter this problem, simply quit the Wizard, open another terminal,
bring up the Wizard and then re-run the tests.  All tests should pass if
the cluster has been configured correctly.

\end{itemize}
