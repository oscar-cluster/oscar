% -*- latex -*-
%
% $Id: release-notes.tex,v 1.15 2002/08/31 16:29:36 jsquyres Exp $
%
% $COPYRIGHT$
%

\section{Release Notes}
\label{sec:release-notes}

The following release notes apply to OSCAR version \oscarversion:

\begchange

\begin{itemize}
\item All nodes must have a hostname other than ``{\tt localhost}''
  that does not contain any underscores (``{\tt \_}'').
  
\item A domain name must be specified for the client nodes when
  defining them.
  
\item The current version of C3 that is shipped with OSCAR requires
  Python 2.x.  The Python 2.x RPMs that are provided in OSCAR are
  re-built to work on RedHat 7.1 as well as RedHat 7.2.  However,
  RedHat 7.2 ships with Python 2.x, so measures were taken such that
  the OSCAR-shipped Python 2.x RPM will not be installed over the
  default RedHat 7.x RPM (specifically, the RPM release number has
  been decremented).  Due to some distribution portability issues,
  OSCAR currently installs a ``compatibility''
  (\file{python2-compat-1.0-1}) RPM to resolve the Python2
  prerequisite that is slightly different across different Linux
  distributions.  Also see the file
  \file{packages/c3/RPMS/NOTE.python2}.

\item In some cases, the test window that is opened from the 
  OSCAR wizard may close suddenly when there is a test failure. If
  this happens, run the test script, \file{testing/test\_cluster\_as\_root},
  manually in a shell window to diagnose the problem.
\end{itemize}

\endchange

%%%%%%%%%%%%%%%%%%%%%%%%%%%%%%%%%%%%%%%%%%%%%%%%%%%%%%%%%%%%%%%%%%%%%%%%%%

\subsection{Mandrake 8.2 Notes}
\label{subsec:mdk82notes}

\begin{itemize}
\item This is the first release of OSCAR to support this distribution.
  OSCAR \oscarversion\- has been tested and successfully installed on
  this distribution (see Table~\ref{tab:oscar-distro-support}); prior
  versions of OSCAR will not work on any flavors of Mandrake without
  significant modifications.
  
\item Copy only the original Mandrake 8.2 packages (i.e., those from
  the CDs or from you local mirror but not the updates package from
  Mandrake Update).  Indeed, because of some naming convention problem
  with mandrake RPMs, updated RPMs with the same version will not be
  selected and you will not be able to pass step 1 (server install).
  
  \begin{discuss}
    This needs to be fixed before 1.4 soup.
  \end{discuss}
  
\item Copy the {\tt libcap} package from RedHat from this site: \\
  \file{ftp://ftp.rpmfind.net/linux/redhat/7.3/en/os/i386/RedHat/RPMS/libcap-1.10-8.i386.rpm}
  to your {\tt /tftpboot/rpm} directory. For the moment, this package
  is necessary, and only RedHat is providing it.
  
\item To build your image, select the
  \file{oscarsamples/Mandrake-8.2-i386.rpmlist} (for a complete client
  with X) or \file{oscarsamples/Mandrake-8.2-noX-i386.rpmlist} (for a
  smaller client with no X) as your RPM list.  Note that the former (X
  client) will be selected by default.

\item Ganglia : The web client is working if you change the line \\
  \verb+ $rrdtool = "/opt/rrdtool-1.0.35/bin/rrdtool";+ \\
  in the file \\
  \file{/var/www/html/ganglia/ganglia.php}\\ 
  to \verb+ $rrdtool = "/usr/bin/rrdtool"+.
  
  \begin{discuss}
    This needs to be fixed before 1.4 soup.
  \end{discuss}

\item Once the server has been installed, you can update your packages
  by using the {\tt MandrakeUpdate} or the {\tt urpmi --auto-select}
  method. It is recommended as some important packages have been
  upgraded (apache and openssl namely !).  However, in order to start
  the oscar wizard again, you need to remove all the upgraded packages
  from {\tt share/serverlists/Mandrake-8.2-i386.rpmlist}.  At the time
  of this writing, delete the lines containing the folowing package
  name : {\tt apache}, {\tt apache-common}, {\tt apache-conf}, {\tt
    apache-modules}, {\tt openssl}.
  
  \begin{discuss}
    What about using the \rpmname{autoupdater} package that is
    installed by OSCAR?  Does that work, or no?
  \end{discuss}

\end{itemize}

%%%%%%%%%%%%%%%%%%%%%%%%%%%%%%%%%%%%%%%%%%%%%%%%%%%%%%%%%%%%%%%%%%%%%%%%%%

\subsection{IA64 and Other Bleeding Edge Systems Notes}
\label{subsec:ia64notes}

\begin{itemize}
  
\item Itanium is supported on RedHat 7.2.  Itanium 2 is not yet
  supported.

  \begin{discuss}
    Is this still true?
  \end{discuss}
  
\item The OSCAR autoinstall diskette will probably not work on Itanium
  systems, but they mostly have support for network booting.
  
\item Some Itanium 1 or 2 systems are supported by OSCAR under
  RedHat's Linux 7.2 release for Itanium.  Itanium systems that can
  use the standard RedHat Itanium kernel for booting and network
  access should work with OSCAR.
  
\item If an Itanium (or an IA32 based) system has new or unknown
  hardware that is not supported by the stock kernel in the base
  operating system release, it will not work with standard OSCAR,
  since OSCAR uses the kernel RPMS supplied with the vendors release
  to boot and load the compute nodes.  If you have a custom compiled
  kernel/module set that works for your hardware, you should look at
  the NCSA utilities at \url{http://oscar.ncsa.uiuc.edu/} to
  enable you to use them in OSCAR.
\end{itemize}





% LocalWords:  tex Exp
